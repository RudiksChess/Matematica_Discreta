\documentclass[a4paper,12pt]{article}
\usepackage[top = 2.5cm, bottom = 2.5cm, left = 2.5cm, right = 2.5cm]{geometry}
\usepackage[T1]{fontenc}
\usepackage[utf8]{inputenc}
\usepackage{multirow} 
\usepackage{booktabs} 
\usepackage{graphicx}
\usepackage{tikz}
\usepackage[spanish]{babel}
\usepackage{setspace}
\setlength{\parindent}{0in}
\usepackage{float}
\usepackage{fancyhdr}
\usepackage{amsmath}
\usepackage{amssymb}
\usepackage{amsthm}
\usepackage{natbib}
\usepackage{graphicx}
\usepackage{subcaption}
\usepackage{booktabs}
\usepackage{etoolbox}
\usepackage{apalike}
\usepackage{minibox}
\usepackage{hyperref}
\usepackage{xcolor}
\usepackage{tcolorbox}
\AtBeginEnvironment{align}{\setcounter{equation}{0}}
\newenvironment{solution}
  {\renewcommand\qedsymbol{$\square$}\begin{proof}[\textcolor{blue}{Solución}]}
  {\end{proof}}
\usepackage{enumitem}
\pagestyle{fancy}

\fancyhf{}

\lhead{\footnotesize Matemática Discreta - }
\rhead{\footnotesize  Rudik Roberto Rompich}
\cfoot{\footnotesize \thepage}

\begin{document}
    \thispagestyle{empty} 
    \begin{tabular}{p{15.5cm}}
    \begin{tabbing}
    \textbf{Universidad del Valle de Guatemala} \\
    Departamento de Matemática\\
    Licenciatura en Matemática Aplicada\\\\
   \textbf{Estudiante:} Rudik Roberto Rompich\\
   \textbf{E-mail:} \textcolor{blue}{ \href{mailto:rom19857@uvg.edu.gt}{rom19857@uvg.edu.gt}}\\
   \textbf{Carné:} 19857
    \end{tabbing}
    \begin{center}
        MM2015 - Matemática Discreta - Catedrático: Mario Castillo\\
        \today
    \end{center}\\
    \hline
    \\
    \end{tabular} 
    \vspace*{0.3cm} 
    \begin{center} 
    {\Large \bf Tarea 7
} 
        \vspace{2mm}
    \end{center}
    \vspace{0.4cm}
    %---------------------------
%\begin{tcolorbox}[colback=gray!15,colframe=black!1!black,title=A nice heading]
%\end{tcolorbox}

%\fbox{lol}

%---------------------------

\section{Problema 1}
\begin{tcolorbox}[colback=gray!15,colframe=black!1!black,title=Teorema 1 - Sección 4.4 de \cite{rosen2012discrete}]
	If $a$ and $m$ are relatively prime integers and $m > 1$, then an inverse of a modulo $m$ exists. Furthermore, this inverse is unique modulo $m$. (That is, there is a unique positive integer a less than m that is an inverse of a modulo $m $and every other inverse of a modulo $m$ is congruent to a modulo $m$.)
	\end{tcolorbox}
Utilice el algoritmo euclidiano para encontrar un entero $a^{-1}$ en $\mathbb{Z}_n$ tal que $a^{-1}\cdot a \equiv 1$ (mod $n$), en donde:
\begin{enumerate}
	\item $a\equiv 5$ (mod 13) 
	\begin{solution}
		Procedemos a calcular el mcd(13,5) con el algoritmo de Euclides: 
		\begin{align*}
			13 &= 2*5+3\\
			5  &= 1*3 + 2\\ 
			3  &= 1*2 +1\\
			2 &= 2*1+0
		\end{align*}
	Por lo tanto, el mdc(13,5) = 1. Entonces, el \textbf{teorema 1} nos afirma que existe un inverso de 5 en módulo 13. Nótese que por el procedimiento anterior, se propone encontrar los coeficientes de Bézout, yendo hacia atrás:
\begin{align*}
	1	&=	(1 * 3) + (-1 * 2)\\
	&=	(-1 * 5) + (2 * 3)\\
	& =	(2 * 13) + (-5 * 5)\\
	&=	(-5 * 5) + (2 * 13).
\end{align*}
Por lo tanto, los coeficientes son -5  y 13. Es decir que $a^{-1}\equiv -5$ mod (13). Comprobando: 
$$a^{-1}\cdot a = -5\cdot 5 = -25 \equiv 1, \quad (\text{mod 13}).$$
	\end{solution}




	\item  $a\equiv 13$ (mod 19)
	\begin{solution}
		Procedemos a calcular el mcd(19,13) con el algoritmo de Euclides: 
		\begin{align*}
			19 &= 1*13 + 6\\
			13 &= 2*6 + 1\\
			6 &= 6*1 + 0
		\end{align*}
		Por lo tanto, el mdc(19,13) = 1. Entonces, el \textbf{teorema 1} nos afirma que existe un inverso de 13 en módulo 19. Nótese que por el procedimiento anterior, se propone encontrar los coeficientes de Bézout, yendo hacia atrás:
		\begin{align*}
			1	&=	(1 * 13) + (-2 * 6)\\
			&=	(-2 * 19) + (3 * 13)\\
			&=	(3 * 13) + (-2 * 19)
		\end{align*}
		Por lo tanto, los coeficientes son 3  y -2. Es decir que $a^{-1}\equiv 3$ mod (19). Comprobando: 
		$$a^{-1}\cdot a = 3\cdot 13 = 39 \equiv 1, \quad (\text{mod 19}).$$
	\end{solution}
\end{enumerate}
El entero $a^{-1}$ se conoce como el \textit{inverso multiplicativo} de $a$ (mod $n$).

\section{Problema 2}

¿Para qué valores de $m > 0$ tiene solución la ecuación $30x + 14y = m$?

\begin{solution}
	Procedemos a encontrar el mcd(30,14): 
	\begin{align*}
		30 &= 2*14 + 2\\
		14 &= 7*2 + 0
	\end{align*}
Es decir que el mcd(30,14) = 2. Por lo tanto, cualquier valor 2$m$> 0, tendrá solución la ecuación. 
\end{solution}

\section{Problema 3}

Resuelva las siguientes congruencias:
\begin{enumerate}
	\item $66x \equiv 42$ (mod 168)
	\begin{solution}
		Procedemos a encontrar la inversa de $a\equiv 66 $ (mod 168). Es neceario calcular el mcd(168,66):
		\begin{align*}
			168 &= 2*66 + 36\\
			66 &= 1*36 + 30\\
			36 &= 1*30 + 6\\
			30 &= 5*6 + 0
		\end{align*}
	Entonces, el mdc(168,66)=6.  Por lo tanto, 6 es múltiplo de 42; por lo que la ecuación sí tiene solución. Ahora tenemos la ecuación diofántica:
	$$168x+66y=42$$ 
	Nótese que por el procedimiento anterior, se propone encontrar los coeficientes de Bézout, yendo hacia atrás:
	\begin{align*}
6	&=	(1 * 36) + (-1 * 30)\\
&=	(-1 * 66) + (2 * 36)\\
&=	(2 * 168) + (-5 * 66)\\
&=	(-5 * 66) + (2 * 168)
	\end{align*}
Nótese que sí multiplcamos por 7, entonces tenemos 2 soluciones: 
$$x_0=14 \qquad y \qquad y_0=-35.$$
Para encontar las soluciones generales, tenemos las siguientes ecuaciones demostradas en clase: 
\begin{center}
	\fbox{$s=-{(\lambda a)\over \text{mcd}(a,b)}$ y $t={(\lambda b)\over \text{mcd}(a,b)}$ }
\end{center}
Por lo tanto: 
$$s=-\frac{\lambda 168}{6}=-28\lambda \qquad y \qquad t=\frac{\lambda 66}{6}=11\lambda.$$
Por lo tanto, la solución: 
$$x=14+11\lambda \qquad y \qquad  y= -35-28\lambda.$$	
	Las soluciones particulares son triviales.
	\end{solution}
	\item $21x \equiv 18$ (mod 30)
	
		\begin{solution}
		Procedemos a encontrar la inversa de $a\equiv 21 $ (mod 30). Es necesario calcular el mcd(30,21):
		\begin{align*}
			30 &= 1*21 + 9\\
			21 &= 2*9 + 3\\
			9 &= 3*3 + 0
		\end{align*}
		Entonces, el mdc(30,21)=3.  Por lo tanto, 3 es múltiplo de 18; por lo que la ecuación sí tiene solución. Ahora tenemos la ecuación diofántica:
		$$30x+21y=18.$$ 
		Nótese que por el procedimiento anterior, se propone encontrar los coeficientes de Bézout, yendo hacia atrás:
		\begin{align*}
3	&=	(1 * 21) + (-2 * 9)\\
&=	(-2 * 30) + (3 * 21)\\
&=	(3 * 21) + (-2 * 30)
		\end{align*}
		Nótese que sí multiplcamos por 6, entonces tenemos 2 soluciones: 
		$$x_0=-12 \qquad y \qquad y_0=18$$
		Para encontar las soluciones generales, tenemos las siguientes ecuaciones demostradas en clase: 
		\begin{center}
			\fbox{$s=-{(\lambda a)\over \text{mcd}(a,b)}$ y $t={(\lambda b)\over \text{mcd}(a,b)}$ }
		\end{center}
		Por lo tanto: 
		$$s=-\frac{\lambda 30}{3}=-10\lambda \qquad y \qquad t=\frac{\lambda 21}{3}=7\lambda.$$
		Por lo tanto, la solución: 
		$$x=-12+7\lambda \qquad y \qquad  y= 18-10\lambda.$$	
			Las soluciones particulares son triviales.
	\end{solution}
	\item $35x \equiv 42$ (mod 49)
		\begin{solution}
		Procedemos a encontrar la inversa de $a\equiv 35 $ (mod 49). Es necesario calcular el mcd(49,35):
		\begin{align*}
49 &= 1*35 + 14\\
35 &= 2*14 + 7\\
14 &= 2*7 + 0
		\end{align*}
		Entonces, el mdc(49,35)=7.  Por lo tanto, 7 es múltiplo de 42; por lo que la ecuación sí tiene solución. Ahora tenemos la ecuación diofántica:
		$$49x+35y=42.$$ 
		Nótese que por el procedimiento anterior, se propone encontrar los coeficientes de Bézout, yendo hacia atrás:
		\begin{align*}
7	&=	(1 * 35) + (-2 * 14)\\
&=	(-2 * 49) + (3 * 35)\\
&=	(3 * 35) + (-2 * 49)
		\end{align*}
		Nótese que sí multiplcamos por 6, entonces tenemos 2 soluciones: 
		$$x_0=-12\qquad y \qquad y_0=18.$$
		Para encontar las soluciones generales, tenemos las siguientes ecuaciones demostradas en clase: 
		\begin{center}
			\fbox{$s=-{(\lambda a)\over \text{mcd}(a,b)}$ y $t={(\lambda b)\over \text{mcd}(a,b)}$ }
		\end{center}
		Por lo tanto: 
		$$s=-\frac{\lambda49}{7}=-7\lambda \qquad y \qquad t=\frac{\lambda 35}{7}=5\lambda.$$
		Por lo tanto, la solución: 
		$$x=-12+5\lambda \qquad y \qquad  y= 18-7\lambda.$$	
	Las soluciones particulares son triviales.
	\end{solution}
\end{enumerate}

\section{Problema 4}
\begin{tcolorbox}[colback=gray!15,colframe=black!1!black,title=Teorema 2 (Teorema chino del resto) de \cite{rosen2012discrete} de la sección 4.4]
	 Let $m_{1}, m_{2}, \ldots, m_{n}$ be pairwise relatively prime positive integers greater than one and $a_{1}, a_{2}, \ldots, a_{n}$ arbitrary integers. Then the system

	\begin{align*}
		x \equiv & a_{1}\quad \left(\bmod  \ m_{1}\right), \\
		x  \equiv & a_{2}\quad \left(\bmod  \ m_{2}\right), \\
		\vdots\\
		x \equiv & a_{n}\quad \left(\bmod \ m_{n}\right)
	\end{align*}
	
	has a unique solution modulo $m=m_{1} m_{2} \cdots m_{n} .$ (That is, there is a solution $x$ with $0 \leq x<m$, and all other solutions are congruent modulo $m$ to this solution.)
	\end{tcolorbox}
Una banda de 17 piratas se reúne para repartirse un cofre con más de 100 monedas de oro, sobrando 1 moneda después del reparto. En la consiguiente pelea, muere un pirata y vuelve a hacerse el reparto sobrando de nuevo 1 moneda. ¿Cuál es el menor número de monedas que puede contener el cofre?
\begin{solution}
	Sea $x$ el número de monedas del cofre. De la primera y segunda condición, tenemos
	$$\begin{cases}
		x \equiv 1 & (\bmod \ 17 )\\
	    x \equiv 1 & (\bmod \  16)
	\end{cases}$$
Entonces, aplicando el \textbf{teorema chino del resto}:
$$m=m_1\cdot m_2 = 17*16=272  \textcolor{blue}{+1}=273.$$
Se le suma +1, para que la condición tenga sentido y obtengamos el número menor de monedas que puede contener el cofre. 
\end{solution}

Supongamos que la solución anterior es el número real de monedas en el cofre y que la historia continúa: siempre que sobran monedas en el reparto, hay una pelea y muere un pirata. ¿Cuántos piratas quedarán vivos cuando en el reparto no sobre ninguna moneda?
\begin{solution}
	Trivial. Usando la función la función $\bmod$, tenemos, para el número menor de monedas (273): 
	\begin{align*}
		273\equiv & 1 \ (\bmod 17) \\
		273\equiv & 1 \ (\bmod 16) \\
		273\equiv & 3 \ (\bmod 15) \\
		273\equiv & 7 \ (\bmod 14) \\
		273\equiv & 0 \ (\bmod 13) \\
	\end{align*}
Por lo tanto, quedarán vivos 13 piratas cuando ya no sobre ninguna moneda.
\end{solution}

%---------------------------
\bibliographystyle{apalike}
\bibliography{sample.bib}

\end{document}