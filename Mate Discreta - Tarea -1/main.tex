\documentclass[a4paper,12pt]{article}
\usepackage[top = 2.5cm, bottom = 2.5cm, left = 2.5cm, right = 2.5cm]{geometry}
% Unfortunately, LaTeX has a hard time interpreting German Umlaute. The following two lines and packages should help. If it doesn't work for you please let me know.
\usepackage[T1]{fontenc}
\usepackage[utf8]{inputenc}
% The following two packages - multirow and booktabs - are needed to create nice looking tables.
\usepackage{multirow} % Multirow is for tables with multiple rows within one cell.
\usepackage{booktabs} % For even nicer tables.
% As we usually want to include some plots (.pdf files) we need a package for that.
\usepackage{graphicx}
% The default setting of LaTeX is to indent new paragraphs. This is useful for articles. But not really nice for homework problem sets. The following command sets the indent to 0.
\usepackage[spanish]{babel}
\usepackage{setspace}
\setlength{\parindent}{0in}
% Package to place figures where you want them.
\usepackage{float}
% The fancyhdr package let's us create nice headers.
\usepackage{fancyhdr}
\usepackage{amsmath}
\usepackage{amssymb}
\usepackage{natbib}
\usepackage{graphicx}
\usepackage{subcaption}
\usepackage{booktabs}
\usepackage{etoolbox}
\usepackage{amsthm}
\newenvironment{solution}
  {\renewcommand\qedsymbol{$\blacksquare$}\begin{proof}[Solución]}
  {\end{proof}}
\pagestyle{fancy}

\fancyhf{}

\lhead{\footnotesize Tarea 1}
\rhead{\footnotesize  Rompich}
\cfoot{\footnotesize \thepage}



\begin{document}
    \thispagestyle{empty} % This command disables the header on the first page.

    \begin{tabular}{p{15.5cm}} % This is a simple tabular environment to align your text nicely
    \begin{tabbing}
    Universidad del Valle de Guatemala \\ 4 de febrero de 2021  \\
    Rudik R. Rompich   - Carné: 19857\\
    \end{tabbing}
    Matemática Discreta 1 - MM2015 - Mario Castillo \\
    \hline % \hline produces horizontal lines.
    \\
    \end{tabular} % Our tabular environment ends here.
    \vspace*{0.3cm} % Now we want to add some vertical space in between the line and our title.
    \begin{center} % Everything within the center environment is centered.
    {\Large \bf Tarea 1 
} % <---- Don't forget to put in the right number
        \vspace{2mm}
    \end{center}
    \vspace{0.4cm}

\section{Ejercicios del tema 1}


\textbf{Ejercicio 7}
Sean $\mathrm{p}, \mathrm{q}, \mathrm{r}, \mathrm{s}$ cuatro proposiciones simples cuyos valores son:\newline\newline 
p verdadera\newline
q verdadera\newline
r verdadera\newline
s falsa.\newline
Diga cuáles de las proposiciones compuestas que aparecen a continuación son verdaderas:
\begin{enumerate}
    \item $(\sim p \rightarrow q) \rightarrow(s \rightarrow r)$.
    \begin{align}
    \intertext{Solución:}
    (\sim V\to V)&\to (F\to V)\\
    (F\to V)&\to (F\to V)\\
    (V)&\to (V)\\
    &V\\ \therefore \text{ es verdadera.}
    \end{align}
    \item $(p \rightarrow q) \rightarrow[(q \rightarrow r) \rightarrow(r \rightarrow s)]$
    \begin{align}
        \intertext{Solución:}
        (V \rightarrow V) &\rightarrow[(V \rightarrow V) \rightarrow(V \rightarrow F)]\\
        (V \rightarrow V) &\rightarrow[(V)\to(F)]\\
        (V)&\to [F]\\
        & F\\
        \therefore \text{ es falsa.}
    \end{align}
    \item $\mathrm{p} \rightarrow[\mathrm{q} \rightarrow(\mathrm{r} \rightarrow \mathrm{s})]$
    \begin{align}
        \intertext{Solución:}
        \mathrm{V} &\rightarrow[\mathrm{V} \rightarrow(\mathrm{V} \rightarrow \mathrm{F})]\\
        V &\to [V\to F]\\
        V&\to [F]\\
        & F\\
        \therefore \text{es falsa}
    \end{align}
    \item  $\mathrm{p}$ y $\mathrm{q} \leftrightarrow \mathrm{r}$ y $ \sim \mathrm{s}$
    \begin{align}
        \intertext{Solución:}
        \intertext{Cambiamos la notación a una más cómoda:}
        p\land q\leftrightarrow r\land\sim s
        \intertext{Entonces tenemos:}
        V\land V&\leftrightarrow V\land\sim F\\
        V\land V&\leftrightarrow V\land V\\
        V &\leftrightarrow V\\
        & V\\
        \therefore \text{verdadera}
    \end{align}
    \item $(p \leftrightarrow q) \rightarrow(s \leftrightarrow r)$
    \begin{align}
        \intertext{Solución:}
        (V\leftrightarrow V)&\to (F\leftrightarrow V)\\
        V&\to F\\
        &F\\
        \therefore falsa. 
    \end{align}
\end{enumerate}

\textbf{Ejercicio 9}
Sea $p$ una proposición tal que para cualquier proposición $q$, es verdadera la proposición $p\lor q$. ¿Qué puede decir acerca del valor de verdad de $p$?

\begin{solution}
Debido a la naturaleza de la operación $p\lor q$; solamente es necesario que exista una proposición $p$ o un $q$ que sea verdadera para que se cumpla la operación mencionada. Entonces, se puede afirmar que por lo menos $p$ es verdadera y en caso $p$ no es verdadera, entonces $q$ es verdadera; cumpliendo $p\lor q$. 
\end{solution}


\textbf{Ejercicio 11}
Hacer la tabla de verdad para cada una de las proposiciones siguientes:
\begin{itemize}
    \item $[(p \vee \sim q) \wedge(\sim p \wedge q) \rightarrow(\sim p \wedge \sim q)]$ \begin{table}[H]
\begin{tabular}{@{}c|c|c|c|c|c|c@{}}
\toprule
$p$ & $q$ & $(p\lor \neg q)$ & $(\neg p \land q)$ & $(\neg p \land \neg q)$ & $(p\lor \neg q) \land (\neg p \land q)$ & $(p\lor \neg q) \land (\neg p \land q)\implies (\neg p \land \neg q) $ \\ \midrule
T   & T   & T                & F                  & F                       & F                                       & T                                                                      \\
T   & F   & T                & F                  & F                       & F                                       & T                                                                      \\
F   & T   & F                & T                  & F                       & F                                       & T                                                                      \\
F   & F   & T                & F                  & T                       & F                                       & T                                                                      \\ \bottomrule
\end{tabular}
\end{table}
    \item $[(\sim p \vee q) \wedge(p \wedge q)] \rightarrow(p \wedge \sim q)$
\begin{table}[H]
\begin{tabular}{@{}c|c|c|c|c|c|c@{}}
\toprule
$p$ & $q$ & $(\neg p \lor q)$ & $(p\land q)$ & $(p\land \neg q)$ & $(\neg p \lor q)\land (p\land q)$ & $\implies$ \\ \midrule
T   & T   & T                 & T            & F                 & T                                 & F          \\
T   & F   & F                 & F            & T                 & F                                 & T          \\
F   & T   & T                 & F            & F                 & F                                 & T          \\
F   & F   & T                 & F            & F                 & F                                 & T          \\ \bottomrule
\end{tabular}
\end{table}    
    
    \item $(\mathrm{p} \wedge \sim \mathrm{q}) \leftrightarrow(\sim \mathrm{p} \wedge \mathrm{q})$
    % Please add the following required packages to your document preamble:
% \usepackage{booktabs}
\begin{table}[H]
\begin{tabular}{@{}c|c|c|c|c@{}}
\toprule
$p$ & $q$ & $(p\land \neg q)$ & $(\neg p \land q)$ & $\leftrightarrow$ \\ \midrule
T   & T   & F                 & F                  & T                 \\
T   & F   & T                 & F                  & F                 \\
F   & T   & F                 & T                  & F                 \\
F   & F   & F                 & F                  & T                 \\ \bottomrule
\end{tabular}
\end{table}
\end{itemize}


\textbf{Ejercicio 22}
Sean los conjuntos:\\
$A=\{4,2,3\}$\\
$\mathrm{B}=\left\{\mathrm{x} \mid \mathrm{x}^{2}=4\right.$ y $\mathrm{x}$ positivo $\} = \{2\}$\\
$\mathrm{C}=\{\mathrm{x} \mid \mathrm{x}$ es par $\}=\{0,2,4,6,8,…\}$\\
$\mathrm{D}=\left\{\mathrm{x} \mid \mathrm{x}^{2}-6 \mathrm{x}+8=0\right\}=\{2,4\}$\\
Completar las siguientes proposiciones escribiendo los simbolos $\subset, \supset 6 \phi$ entre cada par de conjuntos:

\begin{enumerate}
\item $A\supset B$
\item $A\not\subset C$
\item $B\subset C$
\item $A\supset D$
\item $B\subset D$
\item $C \supset D$
\end{enumerate}

\textbf{Ejercicio 25}
Sean los conjuntos:\\
$A=\{x \mid x \in Z, x>8$ y $x<16\}$\\
$\mathrm{B}=\{\mathrm{x} \mid \mathrm{x} \in \mathrm{Z}, \mathrm{x}$ positivo par y $\mathrm{x} \leqslant 12\}$\\
$\mathrm{C}=\{\mathrm{x} \mid \mathrm{x} \in \mathrm{Z}, \mathrm{x}$ múltiplo de 3 y $5<\mathrm{x}<20\}$\\
Encontrar:\newline\newline 
a) $A \cup(B \cap C)=\{6,9,10,11,12,13,14,15\}$.\\
b) $A-B = \{9,11,13,14,15\}$.\\
c) $(B-A)-C = \{2,4,8\}$.\\
d) $(\mathrm{A} \cup \mathrm{B}) \cap(\mathrm{A} \cup \mathrm{C})=\{6,9,10,12,13,14,15\}$\newline\newline 

\textbf{Ejercicio 26}
Demostrar, por doble contención, que para conjuntos $A, B, C,$ cualesquiera:\\
a) $A \cap(B \cup C)=(A \cap B) \cup(A \cap C)$.
\begin{align}
    \intertext{$\subset$}
    A\cap (B\cup C)\\
    x\in A \land x\in (B\cup C)\\
    x\in A\land [x\in B\lor x\in C]\\
    [x\in A\land x\in B]\lor [x\in A\land x\in C]\\
    A \cap(B \cup C)\subset(A \cap B) \cup(A \cap C)
    \intertext{$\supset$}
    x\in (A\cap B)\lor x\in (A\cap C)\\
    [x\in A\land x\in B]\lor[x\in A\land x\in C]\\
    x\in A\land [x\in B\lor x\in C]\\
    x\in A\land x\in (B\cup C))\\
    A \cap B) \cup(A \cap C)\subset A \cap(B \cup C)
\end{align}

b) $A \cap B=\left(A^{\circ} \cup B^{c}\right)^{\circ}$.
\begin{align}
    \intertext{$\subset$}
    x\in A\land x\in B\\
    x\not\in A^c\land x\not\in B^c\\
    x\not\in (A^c\land B^c)\\
    x\in (A^c\land B^c)^c\\
    A \cap B\subset \left(A^{\circ} \cup B^{c}\right)^{\circ}
    \intertext{$\supset$}
    x\in (A^c \cup B^c)^c\\
    x\not\in (A^c\cup B^c)\\
    x\not\in A^c\land x\not\in B^c\\
    x\in A\land x\in B\\
    x\in (A\land B)\\
    \left(A^{\circ} \cup B^{c}\right)^{\circ}\subset A \cap B
\end{align}
c) $A-B=A \cap B^{c}$
\begin{align}
    \intertext{$\subset$}
    x\in A\land x\not\in B\\
    x\in A\land x\in B^c\\
    x\in (A\land B^c)\\
    A-B\subset A \cap B^{c}
    \intertext{$\supset$}
    x\in A\cap B^c\\
    x\in A\land x\in B^c\\
    x\in A\land x\not\in B\\
    x\in(A-B)
\end{align}
d) $(\mathrm{A} \cup \mathrm{B})-\mathrm{C}=(\mathrm{A}-\mathrm{C}) \\\cup(\mathrm{B}-\mathrm{C})$
\begin{align}
    \intertext{$\subset$}
    x\in A\cup B \land x\not\in C\\
    [x\in A\lor x\in B]\land x\not\in C\\
    [x\not\in C\land x\in A]\lor [x\in B\land x\not\in C]\\
    (\mathrm{A} \cup \mathrm{B})-\mathrm{C}\subset(\mathrm{A}-\mathrm{C}) \\\cup(\mathrm{B}-\mathrm{C})
    \intertext{$\supset$}
    [x\in A\land x\not\in C]\land [x\in B\land x\not\in C]\\
    x\not\in C\land [x\in A\lor x\in B]\\
    x\in A\cup B\land x\not\in C\\
    (\mathrm{A}-\mathrm{C}) \\\cup(\mathrm{B}-\mathrm{C})\subset (\mathrm{A} \cup \mathrm{B})-\mathrm{C}
\end{align}
e) $\left(A^{\circ}\right)^{\circ}=A$.
\begin{align}
    \intertext{$\subset$}
    x\in (A^c)^c\\
    x\not\in A^c\\
    x\in A\\
    \left(A^{c}\right)^{c}\subset A
    \intertext{$\supset$}
    x\in A\\
    x\not\in A^c\\
    x\in (A^c)^c\\
    A\subset\left(A^{c}\right)^{c}
\end{align}



\section{Hoja de trabajo 1}

\textbf{Ejercicio 3}
Dados los conjuntos:\\
$\mathrm{U}=\{1,2,3,4,5,6,7,8,9\}$\\
$A=\{1,2,3,4\}$\\
$\mathrm{B}=\{3,4\}$\\
$\mathrm{C}=\{3,5,6,7\}$\\
$\mathrm{D}=\{8,9\}$\\
Expresar en forma enumerativa los conjuntos:\newline\newline
i) $A \cup B = \{1,2,3,4\}$.\\
ii) $A \cup D=\{1,2,3,4,8,9\}$.\\
iii) $A \cap B = \{3,4\}$.\\
iv) $A \cap C = \{3\}$.\\
v) $A \cap D = \emptyset$.\\
vi) $A-B = \{1,2\}$.\\
vii) $A-C=\{1,2,4\}$.\\
viii) $A-D=\{1,2,3,4\}$\\
ix) $\mathrm{C}-\mathrm{A}= \{5,6,7\}$.\\
x) $A^{c}=(-\infty,1)\cup (4,\infty)$.\\
xii ) $\mathrm{C}^{c}=(-\infty,3)\cup (3,5)\cup (7,\infty) $.\\
xiii) $\mathrm{D}^{\mathrm{c}}= (-\infty,8)\cup (9,\infty)$.\newline\newline 

\textbf{Ejercicio 4}
$A-B= A\cap B^c$
\begin{align}
    \intertext{$\subset$}
    x\in A \land x\not\in B\\
    x\in A\land x\in B^c\\
    x\in (A\land B^c)\\
    A-B\subset A\cap B^c
    \intertext{$\supset$}
    x\in (A\cap B^c)\\
    x\in A\land x\in B^c\\
    x\in A\land x\not\in B\\
    x\in (A-B)\\
    A\cap B^c\subset A-B
\end{align}
\textbf{Ejercicio 16}\\
 La proposición $\mathrm{A} \rightarrow \mathrm{B}$ es una implicación.\newline \newline 
\textbf{Ejercicio 17}
En $A\to B$, a $A$ se le llama hipótesis de la implicación y a $B$ se le llama tesis. \newline\newline 
\textbf{Ejercicio 18}\\
Cuál es la única distribución de valores que hace falsa la proposición $A \rightarrow B$. La única distribución consiste en una hipótesis verdadera, pero tiene un tesis falsa. \newline\newline 
\textbf{Ejercicio 19}\\
¿Qué es una tautología?
 
Es una proposición lógica que siempre es verdadera; no importando sus operaciones.

\end{document}