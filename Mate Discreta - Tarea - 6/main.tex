\documentclass[a4paper,12pt]{article}
\usepackage[top = 2.5cm, bottom = 2.5cm, left = 2.5cm, right = 2.5cm]{geometry}
\usepackage[T1]{fontenc}
\usepackage[utf8]{inputenc}
\usepackage{multirow} 
\usepackage{booktabs} 
\usepackage{graphicx}
\usepackage{tikz}
\usepackage[spanish]{babel}
\usepackage{setspace}
\setlength{\parindent}{0in}
\usepackage{float}
\usepackage{fancyhdr}
\usepackage{amsmath}
\usepackage{amssymb}
\usepackage{amsthm}
\usepackage{natbib}
\usepackage{graphicx}
\usepackage{subcaption}
\usepackage{booktabs}
\usepackage{etoolbox}
\usepackage{apalike}
\usepackage{minibox}
\usepackage{hyperref}
\usepackage{xcolor}
\usepackage{tcolorbox}
\AtBeginEnvironment{align}{\setcounter{equation}{0}}
\newenvironment{solution}
  {\renewcommand\qedsymbol{$\square$}\begin{proof}[\textcolor{blue}{Solución}]}
  {\end{proof}}
\usepackage{enumitem}
\pagestyle{fancy}

\fancyhf{}

\lhead{\footnotesize Matemática Discreta - }
\rhead{\footnotesize  Rudik Roberto Rompich}
\cfoot{\footnotesize \thepage}

\begin{document}
    \thispagestyle{empty} 
    \begin{tabular}{p{15.5cm}}
    \begin{tabbing}
    \textbf{Universidad del Valle de Guatemala} \\
    Departamento de Matemática\\
    Licenciatura en Matemática Aplicada\\\\
   \textbf{Estudiante:} Rudik Roberto Rompich\\
   \textbf{E-mail:} \textcolor{blue}{ \href{mailto:rom19857@uvg.edu.gt}{rom19857@uvg.edu.gt}}\\
   \textbf{Carné:} 19857
    \end{tabbing}
    \begin{center}
        MM2015 - Matemática Discreta - Catedrático: Mario Castillo\\
        \today
    \end{center}\\
    \hline
    \\
    \end{tabular} 
    \vspace*{0.3cm} 
    \begin{center} 
    {\Large \bf Tarea 6
} 
        \vspace{2mm}
    \end{center}
    \vspace{0.4cm}
    %---------------------------
%\begin{tcolorbox}[colback=gray!15,colframe=black!1!black,title=A nice heading]
%\end{tcolorbox}

%\fbox{lol}

%---------------------------

\section{Problema 1.}  Demuestre las siguientes propiedades de la divisibilidad para enteros $a, b \& c$.
\begin{enumerate}[label=\alph*.]
\item Si $\ a|b$ y $ b| c$, entonces $a \mid c$.
\begin{solution}
	content...
\end{solution}
\item  Si $a \mid b$ y $a \mid c$, entonces $a \mid \operatorname{mcd}(b, c)$
\begin{solution}
	content...
\end{solution}
\item  Si $a \mid c$ y $b \mid c$, entonces $\operatorname{mcm}(a, b) \mid c$.
\begin{solution}
	content...
\end{solution}
\end{enumerate}

\section{Problema 2.} Pruebe que $n^{5}-n$ siempre es divisible por 30, para todo entero positivo $n$.
\begin{solution}
	content...
\end{solution}
\section{Problema 3.} Encuentre el máximo común divisor de los siguientes pares de números:
\begin{enumerate}[label=\alph*.]
	\item 542 y 234
	\begin{solution}
		content...
	\end{solution}
	\item  9652 y 252
	\begin{solution}
		content...
	\end{solution}
	\item 8374 y 24517
	\begin{solution}
		content...
	\end{solution}
	\item 24537 y 1387
	\begin{solution}
		content...
	\end{solution}
\end{enumerate}

\section{Problema 4.} Encuentre el mínimo común múltiplo de los siguientes pares de números:
\begin{enumerate}[label=\alph*.]
	\item $234 \mathrm{y} 12$
	\begin{solution}
		content...
	\end{solution}
	\item 142 y 742
	\begin{solution}
		content...
	\end{solution}
	\item $17 \mathrm{y} 141$
	\begin{solution}
		content...
	\end{solution}
	\item 35 y 24
	\begin{solution}
		content...
	\end{solution}
\end{enumerate}

\section{Problema 5.} Encuentre todos los enteros $x \& y$, tales que:
\begin{enumerate}[label=\alph*.]
	\item $93 x+81 y=3$
	\begin{solution}
		content...
	\end{solution}
	\item $43 x+128 y=1$
	\begin{solution}
		content...
	\end{solution}
\end{enumerate}

\section{Problema 6.} Encuentre enteros positivos $x \& y$ que satisfagan las condiciones indicadas:
\begin{enumerate}[label=\alph*.]
	\item $x+y=150, \operatorname{mcd}(x, y)=30$
	\begin{solution}
		content...
	\end{solution}
	\item $x y=8400, \operatorname{mcd}(x, y)=20$
	\begin{solution}
		content...
	\end{solution}
\end{enumerate}

\section{Problema 7.} Calcule $\operatorname{mcd}(203,91,77)$.
\begin{solution}
	content...
\end{solution}

%---------------------------
%\bibliographystyle{apalike}
%\bibliography{sample.bib}

\end{document}