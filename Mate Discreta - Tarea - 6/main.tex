\documentclass[a4paper,12pt]{article}
\usepackage[top = 2.5cm, bottom = 2.5cm, left = 2.5cm, right = 2.5cm]{geometry}
\usepackage[T1]{fontenc}
\usepackage[utf8]{inputenc}
\usepackage{multirow} 
\usepackage{booktabs} 
\usepackage{graphicx}
\usepackage{tikz}
\usepackage[spanish]{babel}
\usepackage{setspace}
\setlength{\parindent}{0in}
\usepackage{float}
\usepackage{fancyhdr}
\usepackage{amsmath}
\usepackage{amssymb}
\usepackage{amsthm}
\usepackage{natbib}
\usepackage{graphicx}
\usepackage{subcaption}
\usepackage{booktabs}
\usepackage{etoolbox}
\usepackage{apalike}
\usepackage{minibox}
\usepackage{hyperref}
\usepackage{xcolor}
\usepackage{tcolorbox}
\AtBeginEnvironment{align}{\setcounter{equation}{0}}
\newenvironment{solution}
  {\renewcommand\qedsymbol{$\square$}\begin{proof}[\textcolor{blue}{Solución}]}
  {\end{proof}}
\usepackage{enumitem}
\pagestyle{fancy}

\fancyhf{}

\lhead{\footnotesize Matemática Discreta - }
\rhead{\footnotesize  Rudik Roberto Rompich}
\cfoot{\footnotesize \thepage}

\begin{document}
    \thispagestyle{empty} 
    \begin{tabular}{p{15.5cm}}
    \begin{tabbing}
    \textbf{Universidad del Valle de Guatemala} \\
    Departamento de Matemática\\
    Licenciatura en Matemática Aplicada\\\\
   \textbf{Estudiante:} Rudik Roberto Rompich\\
   \textbf{E-mail:} \textcolor{blue}{ \href{mailto:rom19857@uvg.edu.gt}{rom19857@uvg.edu.gt}}\\
   \textbf{Carné:} 19857
    \end{tabbing}
    \begin{center}
        MM2015 - Matemática Discreta - Catedrático: Mario Castillo\\
        \today
    \end{center}\\
    \hline
    \\
    \end{tabular} 
    \vspace*{0.3cm} 
    \begin{center} 
    {\Large \bf Tarea 6
} 
        \vspace{2mm}
    \end{center}
    \vspace{0.4cm}
    %---------------------------
%\begin{tcolorbox}[colback=gray!15,colframe=black!1!black,title=A nice heading]
%\end{tcolorbox}

%\fbox{lol}

%---------------------------


\section{Problema 1.}  Demuestre las siguientes propiedades de la divisibilidad para enteros $a, b$ \& $c$.
\begin{enumerate}[label=\alph*.]
\item Si $\ a|b$ y $ b| c$, entonces $a \mid c$.
\begin{solution}
	Sean $m,n\in\mathbb{Z}$. Por hipótesis tenemos: 
	$$b=ma\qquad \text{ y } \qquad c = nb.$$
 	Ahora bien, debemos probar
 	$$c=ka,\qquad k\in\mathbb{Z}.$$ 
En donde $$c=nb=nma=ka. $$
$\therefore \ $  $a$ divide a $c$ $\implies c | a$. 
 \end{solution}

\item  Si $a \mid b$ y $a \mid c$, entonces $a \mid \operatorname{mcd}(b, c)$

\begin{solution}
	Sean $n,m\in\mathbb{Z}$.  Por hipótesis tenemos: 
	$$b=ma \qquad \text{ y } \qquad c=na.$$
	
	Además, por el teorema de Bézout, el mcd$(b,c)$ se puede expresar como $\text{mcd}(b,c)=k_1b+k_2c$. Ahora bien, debemos probar
	$$k_1b+k_2c=ka, \qquad k\in\mathbb{Z}.$$
	En donde
	$$k_1b+k_2c=k_1(ma)+k_2(na)=(k_1m+k_2n)a=ka.$$
	$\therefore \ a$ divide a $\text{mcd}(b,c) \implies a |\text{mcd}(b,c) $. 
\end{solution}


\item  Si $a \mid c$ y $b \mid c$, entonces $\operatorname{mcm}(a, b) \mid c$.
\begin{solution}
	Sean $m,n\in\mathbb{Z}$.  Por hipótesis tenemos: 
	$$c=ma\implies a=\frac{c}{m} \qquad \text{ y }\qquad  c=nb\implies b=\frac{c}{n}.$$	
	Además, por el teorema de 5 de la sección 4 de \cite{rosen2012discrete}, sabemos  
	$$\text{mcm}(a,b)=\frac{ab}{\text{mcd}(a,b)}.$$
	Por el teorema de Bézout, mcd$(a,b)=sa+tb$. Ahora bien, debemos probar
	$$c=k\frac{ab}{\text{mcd}(a,b)}, \quad k:=(sn+tm) \in\mathbb{Z}. $$
	Entonces 
	$$k\frac{ab}{sa+tb}=k\frac{\left(\frac{c}{m}\right)\left(\frac{c}{n}\right)}{s\left(\frac{c}{m}\right)+t\left(\frac{c}{n}\right)}=k\frac{\frac{c^2}{mn}}{\frac{scn+tcm}{mn}}=k\frac{c^2}{c(sn+tm)}=k\frac{c}{sn+tm}=c.$$
	Por lo tanto $\text{mcm}(a,b)$ divide a $c$. $\implies\text{mcm}(a,b)|c$.
\end{solution}
\end{enumerate}

\section{Problema 2.} Pruebe que $n^{5}-n$ siempre es divisible por 30, para todo entero positivo $n$.
\begin{tcolorbox}[colback=gray!15,colframe=black!1!black,title=FERMAT’S LITTLE THEOREM]
	If $p$ is prime and $a$ is an integer not divisible by $p$, then
	$a^{p-1} \equiv 1$ (mod $p$).\newline \newline 
	Furthermore, for every integer a we have $a^p \equiv a$ (mod $p$).
	\end{tcolorbox}
\begin{solution}
	Trivial. A probar: 
	$$n^5-n=30c, \qquad c\in\mathbb{Z}.$$
	
	En módulo 5 la expresión sería
	$$n^5-n = 0.$$
	Entonces, por el  teorema pequeño de Fermat 
	$$n^5\equiv n \qquad \text{mod 5}.$$
\end{solution}


\section{Problema 3.} Encuentre el máximo común divisor de los siguientes pares de números:
\begin{enumerate}[label=\alph*.]
	\item 542 y 234
	\begin{solution}
	Por el algoritmo de Euclides: 
	\begin{align*}
		  542 &= 2*234+74   \\
		 234 &= 3*74+12\\
		 74 &= 6*12+2\\
		 12 &= 6*2+0
	\end{align*}
$\therefore \ $ mcd$(542,234)$=2. 
\end{solution}
	\item  9652 y 252
	\begin{solution}
		\begin{align*}
			 9652 &= 252*38 + 76\\
			252 &= 76*3 + 24\\
			76 &= 24*3 + 4\\
			24 &= 4*6 + 0
		\end{align*}
	$\therefore \ $ mcd$(9652,252)$=4. 
	\end{solution}
	\item 8374 y 24517
	\begin{solution}
		\begin{align*}
			 24517 &= 8374*2 + 7769\\
			8374 &= 7769*1 + 605\\
			7769 &= 605*12 + 509\\
			605 &= 509*1 + 96\\
			509 &= 96*5 + 29\\
			96 &= 29*3 + 9\\
			29 &= 9*3 + 2\\
			9 &= 2*4 + 1\\
			2 &= 1*2 + 0
		\end{align*}
		$\therefore \ $ mcd$(8374,24517)$=1. 
	\end{solution}
	\item 24537 y 1387
	\begin{solution}
		\begin{align*}
			 24537 &= 1387*17 + 958\\
			1387 &= 958*1 + 429\\
			958 &= 429*2 + 100\\
			429 &= 100*4 + 29\\
			100 &= 29*3 + 13\\
			29 &= 13*2 + 3\\
			13 &= 3*4 + 1\\
			3 &= 1*3 + 0
		\end{align*}
	$\therefore \ $ mcd$(24537,1387)$=1. 
	\end{solution}
\end{enumerate}

\section{Problema 4.} Encuentre el mínimo común múltiplo de los siguientes pares de números:
\begin{enumerate}[label=\alph*.]
	\item $234$ y $12$
	\begin{solution}
		Comenzamos calculando el mcd:
		\begin{align*}
			 234 &= 12*19 + 6\\
			12    &= 6*2 + 0
		\end{align*}
	Implica que 
	$$mcm(234,12)=\frac{234*12}{mcd(234,12)}=\frac{234*12}{6}=234*2=468.$$
	\end{solution}
	\item 142 y 742
	\begin{solution}
			Comenzamos calculando el mcd:
		\begin{align*}
			 742 &= 142*5 + 32\\
			142 &= 32*4 + 14\\
			32 &= 14*2 + 4\\
			14 &= 4*3 + 2\\
			4 &= 2*2 + 0
		\end{align*}
	Implica que
	$$mcm(742,142)=\frac{742*142}{mcd(742,142)}=\frac{742*142}{2}=742*71=52682.$$
	\end{solution}
	\item $17 $ y $141$
	\begin{solution}
			Comenzamos calculando el mcd:
		\begin{align*}
			 141 &= 17*8 + 5\\
			17 &= 5*3 + 2\\
			5 &= 2*2 + 1\\
			2 &= 1*2 + 0
		\end{align*}
	Implica que
	$$mcm(141,17)=\frac{141*17}{mcd(141,17)}=\frac{141*17}{1}=2397.$$
	\end{solution}
	\item 35 y 24
	\begin{solution}
			Comenzamos calculando el mcd:
		\begin{align*}
			 35 &= 24*1 + 11\\
			24 &= 11*2 + 2\\
			11 &= 2*5 + 1\\
			2 &= 1*2 + 0
		\end{align*}
	Implica que
	$$mcm(35,24)=\frac{35*24}{mcd(35,24)}=\frac{35*24}{1}=840.$$
	\end{solution}
\end{enumerate}

\section{Problema 5.} Encuentre todos los enteros $x$ \& $y$, tales que:
\begin{enumerate}[label=\alph*.]
	\item $93 x+81 y=3$
	\begin{solution}
		Se determina si la ecuación diofántica tiene soluciones. Calculamos el mcd(93,81):
		\begin{align*}
			 93 &= 81*1 + 12\\
			81 &= 12*6 + 9\\
			12 &= 9*1 + 3\\
			9 &= 3*3 + 0\\
		\end{align*}
	Concluimos que mcd(93,81)=3 (múltiplo de 3), por lo tanto, sí tiene solución la ecuación diofántica. Ahora, calculamos una solución particular: 
	\begin{align*}
		3	&=	(1 * 12) + (-1 * 9)\\
		&=	(-1 * 81) + (7 * 12)\\
		&=	(7 * 93) + (-8 * 81)\\
		&=	(-8 * 81) + (7 * 93)
	\end{align*}
Es decir:
$$X_0=7, Y_0=-8$$
	
	 Quiere decir que la ecuación tiene soluciones infinitas, y por lo tanto se pueden usar las siguientes fórmulas encontradas en clase: 
	$$t=\frac{\lambda b}{\text{mcd}(a,b)}=\frac{\lambda 81}{3}=27\lambda \quad\text{ y }\quad s=-\frac{\lambda 93}{3}=-31\lambda. $$
	Entonces, la solución es:
	$$\begin{cases}
		x &= 7+27\lambda\\
		y &= -8-31\lambda
	\end{cases}$$
	\end{solution}
	\item $43 x+128 y=1$
	\begin{solution}
		Se determina si la ecuación diofántica tiene soluciones. Calculamos el mcd(128,43):
		\begin{align*}
			 128 = 43*2 + 42\\
			43 = 42*1 + 1\\
			42 = 1*42 + 0\\
		\end{align*}
	Es decir:
	$$X_0=3,Y_0=-1.$$
		Concluimos que mcd(128,43)=1 (múltiplo de 1), por lo tanto, sí tiene solución la ecuación diofántica. Ahora, calculamos la solución particular: 
		\begin{align*}
			1	&=	(1 * 43) + (-1 * 42)\\
			&=	(-1 * 128) + (3 * 43)
		\end{align*}
		
		
		Quiere decir que la ecuación tiene soluciones infinitas, y por lo tanto se pueden usar las siguientes fórmulas encontradas en clase: 
		$$t=\frac{\lambda b}{\text{mcd}(a,b)}=\frac{\lambda 128}{1} \quad\text{ y }\quad s=-\frac{\lambda 43}{1}. $$
		Entonces, la solución es:
		$$\begin{cases}
			x &= 3+128\lambda\\
			y &= -1-43\lambda
		\end{cases}$$
	\end{solution}
\end{enumerate}

\section{Problema 6.} Encuentre enteros positivos $x $ \& $y$ que satisfagan las condiciones indicadas:
\begin{enumerate}[label=\alph*.]
	\item $x+y=150, \operatorname{mcd}(x, y)=30$
	\begin{solution}
		Usamos el teorema de Bézout:
		$$\begin{cases}
			x+y&=150\\
			sx+ty &=30
		\end{cases}$$
	Es decir, despejando para $y$, se tiene: 
	$$y=150-x\implies sx+t(150-x)=30\implies sx+150t-xt=30\implies$$
	$$\implies x(s-t)=30-150t\implies x= \frac{30-150t}{s-t}.$$
	Volviendo a la ecuación inicial:
	$$ \frac{30-150t}{(s-t)}+y=150\implies y=150-\frac{30-150t}{(s-t)}=\frac{150s-30}{s-t}.$$
	Por lo tanto $x=\frac{30-150t}{s-t}$ y $y=\frac{150s-30}{s-t}$.
	\end{solution}
	\item $x y=8400, \operatorname{mcd}(x, y)=20$
	\begin{solution}
		Usamos el teorema de Bézout:
		$$\begin{cases}
			xy&=8400\\
			sx+ty &=20
		\end{cases}$$
	Es decir, despejando para $y$
	$$y=\frac{8400}{x}$$
	Evaluando en la segunda ecuación 
	$$sx+t\left(\frac{8400}{x}\right)=20\implies x^2-\frac{20}{s}+\frac{8400t}{s}=0$$
	Usando la fórmula de Dieta: 
	$$x=\frac{10\pm \sqrt{10-84st}}{s}.$$
	Volviendo a la primera ecuación, concluimos: 
	$$y=\frac{10\pm \sqrt{10-84st}}{t}.$$
	En las soluciones asumimos que $t,s\neq 0$.
	\end{solution}
\end{enumerate}

\section{Problema 7.} Calcule $\operatorname{mcd}(203,91,77)$.
\begin{solution}
	Encontrando el mcd(203,91):
	\begin{align*}
		 203 &= 91*2 + 21\\
		91 &= 21*4 + 7\\
		21 &= 7*3 + 0
	\end{align*}
El mcd(203,91)=7. Ahora bien, calculamos el mdc(7,77):
\begin{align*}
	 77 = 7*11 + 0
\end{align*}
Por lo tanto, el gcd(203,91,7) = 7.
\end{solution}

%---------------------------
\bibliographystyle{apalike}
\bibliography{sample.bib}

\end{document}