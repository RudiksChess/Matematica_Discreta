\documentclass[a4paper,12pt]{article}
\usepackage[top = 2.5cm, bottom = 2.5cm, left = 2.5cm, right = 2.5cm]{geometry}
\usepackage[T1]{fontenc}
\usepackage[utf8]{inputenc}
\usepackage{multirow} 
\usepackage{booktabs} 
\usepackage{graphicx}
\usepackage{tikz}
\usepackage[spanish]{babel}
\usepackage{setspace}
\setlength{\parindent}{0in}
\usepackage{float}
\usepackage{fancyhdr}
\usepackage{amsmath}
\usepackage{amssymb}
\usepackage{amsthm}
\usepackage{natbib}
\usepackage{graphicx}
\usepackage{subcaption}
\usepackage{booktabs}
\usepackage{etoolbox}
\usepackage{apalike}
\usepackage{minibox}
\usepackage{hyperref}
\usepackage{xcolor}
\usepackage{tcolorbox}
\AtBeginEnvironment{align}{\setcounter{equation}{0}}
\newenvironment{solution}
  {\renewcommand\qedsymbol{$\square$}\begin{proof}[\textcolor{blue}{Solución}]}
  {\end{proof}}
\usepackage{enumitem}
\pagestyle{fancy}

\fancyhf{}

\lhead{\footnotesize Matemática Discreta - }
\rhead{\footnotesize  Rudik Roberto Rompich}
\cfoot{\footnotesize \thepage}

\begin{document}
    \thispagestyle{empty} 
    \begin{tabular}{p{15.5cm}}
    \begin{tabbing}
    \textbf{Universidad del Valle de Guatemala} \\
    Departamento de Matemática\\
    Licenciatura en Matemática Aplicada\\\\
   \textbf{Estudiante:} Rudik Roberto Rompich\\
   \textbf{E-mail:} \textcolor{blue}{ \href{mailto:rom19857@uvg.edu.gt}{rom19857@uvg.edu.gt}}\\
   \textbf{Carné:} 19857
    \end{tabbing}
    \begin{center}
        MM2015 - Matemática Discreta - Catedrático: Mario Castillo\\
        \today
    \end{center}\\
    \hline
    \\
    \end{tabular} 
    \vspace*{0.3cm} 
    \begin{center} 
    {\Large \bf Tarea 8
} 
        \vspace{2mm}
    \end{center}
    \vspace{0.4cm}
    %---------------------------
%\begin{tcolorbox}[colback=gray!15,colframe=black!1!black,title=A nice heading]
%\end{tcolorbox}

%\fbox{lol}

%---------------------------

\section{Problema 1}

Calcule $3^{32} \ (\bmod\ 7)$
\begin{solution}
	Convertimos el dígito a binario: 
	$$(32)_2=100000$$
	\begin{align*}
		\begin{array}{c|l}
			(32)_{2} & c_{0}=1 \\
			\hline 1 & c_{1} \equiv 1^{2} \cdot 3^{1}=1 \cdot 3=3 \bmod 7 \\
			\hline 0 & c_{2} \equiv 3^{2} \cdot 3^{0}=9 \cdot 1=9 \equiv 2 \bmod 7 \\
			\hline 0 & c_{3} \equiv 2^{2} \cdot 3^{0}=4 \cdot 1=4 \bmod 7 \\
			\hline 0 & c_{4} \equiv 4^{2} \cdot 3^{0}=16 \cdot 1=16 \equiv 2 \bmod 7 \\
			\hline 0 & c_{5} \equiv 2^{2} \cdot 3^{0}=4 \cdot 1=4 \bmod 7 \\
			\hline 0 & c_{6} \equiv 4^{2} \cdot 3^{0}=16 \cdot 1=16 \equiv 2 \bmod 7
		\end{array}
	\end{align*}
$$\therefore \ 3^{32}\equiv 2 \ (\bmod\ 7).$$
\end{solution}
\section{Problema 2}
Calcule $2^{35}\ (\bmod\ 9)$
\begin{solution}
	Convertimos el dígito a binario: 
	$$(35)_2=100011.$$

	\begin{align*}
		\begin{array}{l|l}
			(35)_{2} & c_{0}=1 \\
			\hline 1 & c_{1} \equiv 1^{2} \cdot 2^{1}=1 \cdot 2=2 \bmod 9 \\
			\hline 0 & c_{2} \equiv 2^{2} \cdot 2^{0}=4 \cdot 1=4 \bmod 9 \\
			\hline 0 & c_{3} \equiv 4^{2} \cdot 2^{0}=16 \cdot 1=16 \equiv 7 \bmod 9 \\
			\hline 0 & c_{4} \equiv 7^{2} \cdot 2^{0}=49 \cdot 1=49 \equiv 4 \bmod 9 \\
			\hline 1 & c_{5} \equiv 4^{2} \cdot 2^{1}=16 \cdot 2=32 \equiv 5 \bmod 9 \\
			\hline 1 & c_{6} \equiv 5^{2} \cdot 2^{1}=25 \cdot 2=50 \equiv 5 \bmod 9
		\end{array}
	\end{align*}
	$$\therefore \ 2^{35}\equiv 5 \ (\bmod\ 9).$$
\end{solution}
\section{Problema 3}
Calcule $128^{129} \ (\bmod 17)$
\begin{solution}
	Convertimos el dígito a binario: 
	$$(129)_2=10000001$$
	\begin{align*}
			\begin{array}{l|l}
				(129)_{2} & c_{0}=1 \\
				\hline 1 & c_{1} \equiv 1^{2} \cdot 128^{1}=1 \cdot 128=128 \equiv 9 \bmod 17 \\
				\hline 0 & c_{2} \equiv 9^{2} \cdot 128^{0}=81 \cdot 1=81 \equiv 13 \bmod 17 \\
				\hline 0 & c_{3} \equiv 13^{2} \cdot 128^{0}=169 \cdot 1=169 \equiv 16 \bmod 17 \\
				\hline 0 & c_{4} \equiv 16^{2} \cdot 128^{0}=256 \cdot 1=256 \equiv 1 \bmod 17 \\
				\hline 0 & c_{5} \equiv 1^{2} \cdot 128^{0}=1 \cdot 1=1 \bmod 17 \\
				\hline 0 & c_{6} \equiv 1^{2} \cdot 128^{0}=1 \cdot 1=1 \bmod 17 \\
				\hline 0 & c_{7} \equiv 1^{2} \cdot 128^{0}=1 \cdot 1=1 \bmod 17 \\
				\hline 1 & c_{8} \equiv 1^{2} \cdot 128^{1}=1 \cdot 128=128 \equiv 9 \bmod 17
			\end{array}
	\end{align*}
	$$\therefore \ 128^{129} \equiv 9 \ (\bmod \ 17).$$
\end{solution}
%---------------------------
%\bibliographystyle{apalike}
%\bibliography{sample.bib}

\end{document}