\input{Configuraciones/paquetes}
\begin{document}
\input{Configuraciones/nombres}
%--------------------------
\section{Demostrar por inducción.}

$$\sum_{i=0}^n C(n,i)=2^n$$

\begin{proof}
Para la demostración de esta prueba, el único teorema que vamos a asumir verdadero es el siguiente: 

\begin{tcolorbox}[colback=gray!15,colframe=gray!1!gray,title=Teorema 2 de la Sección 6.3 de \cite{rosen2012discrete}]
El número de $r$-combinaciones de un conjunto con $n$ elementos, donde $n$ es un entero no negativo y $r$ es un entero con $0\leq r\leq n$, es igual a: 
$$C(n,r)=\frac{n!}{r!(n-r)!} = \binom{n}{r}$$
\end{tcolorbox}

\linita 

Antes de comenzar el problema, usaremos una notación más cómoda: 
$$\sum_{i=0}^n \binom{n}{i}=2^n$$
Ahora bien, comenzamos la prueba por inducción. 

\linita 

\fbox{\textbf{Paso base:}}\newline 

Se propone $n=1$, entonces: 
\begin{align*}
    \sum_{i=0}^1 \binom{1}{i}= \sum_{i=0}^1 \frac{1!}{i!(1-i)!} = \frac{1!}{0!(1-0)!} +\frac{1!}{1!(1-1)!}=1+1=2.
\end{align*}
Por lo tanto, el caso base se cumple. 

\linita 

\fbox{\textbf{Paso inductivo:}}\newline 

Asumimos que $S(k)$ es verdadera para un $k\geq 1$.
$$\sum_{i=0}^k \binom{k}{i}=2^k$$

demostramos que $S(k)\implies S(k+1)$. Es decir que es necesario probar que: 
$$\sum_{i=0}^{k+1} \binom{k+1}{i}=2^{k+1}.$$

Para la argumentación, será necesario probar la identidad de Pascal, que se define: 

\begin{tcolorbox}[colback=gray!15,colframe=gray!1!gray,title=Teorema 2 de la sección 6.4 de \cite{rosen2012discrete}.]
$$\binom{n+1}{k}=\binom{n}{k-1}+\binom{n}{k}$$
\begin{proof}
\begin{align*}
    \binom{n}{k-1}+\binom{n}{k} &= \frac{n!}{(k-1)!(n-(k-1))!} + \frac{n!}{k!(n-k)!}\\
    &= \frac{n!k+n!(n-k+1)}{k!(n-k+1)!}\\
    &= \frac{(n+1)n!}{k!((n+1)-k)!}\\
    &= \frac{(n+1)!}{k!((n+1)-k)!}\\
    &= \binom{n+1}{k}.
\end{align*}
\end{proof}
\end{tcolorbox}
\begin{tcolorbox}[colback=red!15,colframe=red!1!red,title=Casos importantes]
\begin{enumerate}
    \item \begin{align*}\binom{k+1}{0} = \binom{k}{0-1}+\binom{k}{0}= 0+\binom{k}{0} = \binom{k}{0} 
    \end{align*}
    \item Equivalencia \begin{align*}\binom{k+1}{k+1} = 1 = \binom{k}{k}
    \end{align*}
\end{enumerate}
\end{tcolorbox}
Ahora bien, 
\begin{align*}
    \sum_{i=0}^{k+1} \binom{k+1}{i} &= \binom{k+1}{0}+\binom{k+1}{1}+\binom{k+1}{2}+\cdots+ \binom{k+1}{k}+\binom{k+1}{k+1}
    \intertext{Aplicando la identidad de Pascal:}
    &= 
        \binom{k+1}{0}+ \left[\binom{k}{1-1}+\binom{k}{1}\right]+\left[\binom{k}{2-1}+\binom{k}{2}\right]+\cdots+\\
        &+ \left[\binom{k}{k-1}+\binom{k}{k}\right]+\binom{k+1}{k+1}\\
    &= \binom{k+1}{0}+ \left[\binom{k}{0}+\binom{k}{1}\right]+\left[\binom{k}{1}+\binom{k}{2}\right]+\cdots+\\
        &+ \left[\binom{k}{k-1}+\binom{k}{k}\right]+\binom{k+1}{k+1}
        \intertext{Usando los casos importantes, mencionado en el cuadro de arriba:}
        &= \binom{k}{0}+ \left[\binom{k}{0}+\binom{k}{1}\right]+\left[\binom{k}{1}+\binom{k}{2}\right]+\cdots+\\
        &+ \left[\binom{k}{k-1}+\binom{k}{k}\right]+\binom{k}{k}\\
        &= \left[\binom{k}{0} + \binom{k}{0}\right] +\left[\binom{k}{1} + \binom{k}{1}\right]+\cdots + \left[\binom{k}{k} + \binom{k}{k}\right]\\
        &= 2 \sum_{i=0}^{k}\binom{k}{i}\\
        &= 2\cdot 2^n\\
        &= 2^{n+1}
\end{align*}
$\therefore$ $S(n)$ es verdadera para todo número $n\in\mathbb{N}$.
\end{proof}


%---------------------------
\bibliographystyle{apalike}
\bibliography{sample.bib}

\end{document}