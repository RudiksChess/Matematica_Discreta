\documentclass[a4paper,12pt]{article}
\usepackage[top = 2.5cm, bottom = 2.5cm, left = 2.5cm, right = 2.5cm]{geometry}
% Unfortunately, LaTeX has a hard time interpreting German Umlaute. The following two lines and packages should help. If it doesn't work for you please let me know.
\usepackage[T1]{fontenc}
\usepackage[utf8]{inputenc}
% The following two packages - multirow and booktabs - are needed to create nice looking tables.
\usepackage{multirow} % Multirow is for tables with multiple rows within one cell.
\usepackage{booktabs} % For even nicer tables.
% As we usually want to include some plots (.pdf files) we need a package for that.
\usepackage{graphicx}
% The default setting of LaTeX is to indent new paragraphs. This is useful for articles. But not really nice for homework problem sets. The following command sets the indent to 0.
\usepackage[spanish]{babel}
\usepackage{setspace}
\setlength{\parindent}{0in}
% Package to place figures where you want them.
\usepackage{float}
% The fancyhdr package let's us create nice headers.
\usepackage{fancyhdr}
\usepackage{amsmath}
\usepackage{amssymb}
\usepackage{natbib}
\usepackage{graphicx}
\usepackage{subcaption}
\usepackage{booktabs}
\usepackage{etoolbox}
\AtBeginEnvironment{align}{\setcounter{equation}{0}}

\pagestyle{fancy}

\fancyhf{}

\lhead{\footnotesize Sesión 4}
\rhead{\footnotesize  Rompich}
\cfoot{\footnotesize \thepage}

\begin{document}
    \thispagestyle{empty} % This command disables the header on the first page.

    \begin{tabular}{p{15.5cm}} % This is a simple tabular environment to align your text nicely
    \begin{tabbing}
    Universidad del Valle de Guatemala \\ 22 de enero de 2021  \\
    Rudik R. Rompich   - Carné: 19857\\
    \end{tabbing}
    Matemática Discreta 1 - MM2015 - Mario Castillo \\
    \hline % \hline produces horizontal lines.
    \\
    \end{tabular} % Our tabular environment ends here.
    \vspace*{0.3cm} % Now we want to add some vertical space in between the line and our title.
    \begin{center} % Everything within the center environment is centered.
    {\Large \bf Sesión 4 (Asíncrona)
} % <---- Don't forget to put in the right number
        \vspace{2mm}
    \end{center}
    \vspace{0.4cm}

Instrucciones: Resuelva los siguientes ejercicios de forma clara y ordenada, dejando constancia de todo su razonamiento.
\newline \newline 
1. Escriba en lenguaje simbólico la definición de cada una de las siguientes operaciones entre conjuntos:
\begin{itemize}
    \item Unión - Dados dos conjuntos $A$ y $B$ $\to A\cup B= \{x | x\in A \lor x\in B\}$
    \item Intersección - Dados dos conjuntos $A$ y $B$ $\to A\cap B= \{x | x\in A \land x\in B \}$
    \item Diferencia - Dados dos conjuntos $A$ y $B$ $\to A-B =\{x|x \in A \land x\not\in B\}$
    \item Diferencia simétrica - Dados dos conjuntos $A$ y $B$ $\to A \Delta B=\{x \mid x \in A-B \vee x \in B-A\} = (A-B)\cup (B-A)$
    \item Complemento - Sea $A$ un subconjunto de $\mathcal{U }$, siendo $\mathcal{U}$ un conjunto universal de una teoría. $A^{0}=\{x \mid x \in \mathcal{U} \wedge x \notin A\}=\mathcal{U}-A$
\end{itemize}

2. Use la doble contención para demostrar la distributividad de la intersección respecto de la unión, es decir la igualdad:\newline 

$$ A \cap(B \cup C)=(A \cap B) \cup(A \cap C)$$
Sugerencia: Use la demostración presentada en las páginas 29 a 32 como guía.
\newline 
\textit{Demostración: }
\newline 
Por doble contención:
\begin{enumerate}
    \item $A\cap(B\cup C) \subset (A \cap B) \cup(A \cap C)$\newline 
    \begin{align}
        \to& x\in A \land x\in (B\cup C)\\
        \to& x\in A \land \{x\in B \lor x\in C \}
    \intertext{Por lo que se tiene:}
        \to& \{x\in A \land x\in B\}\lor\{x\in A\land x\in C\}\\
        \to& \{x\in A\cap B\}\lor \{x\in A\cap C\}\\
        \to& x\in (A\cap B)\cup (A\cap C) \\
        \therefore& A\cap(B\cup C) \subset (A \cap B) \cup(A \cap C)
    \end{align}
    \item $(A \cap B) \cup(A \cap C)\subset A\cap(B\cup C) $\newline 
    \begin{align}
        \to& \{x\in A \land x\in B\}\lor \{x\in A\land x\in C\}\\
        \to& x\in A\land \{x\in B\lor x\in C\}\\
        \to& x\in A\land \{x\in B\cup C\}\\
        \to& A\cap(B\cup C)
    \end{align}
\end{enumerate}

$$\therefore A \cap(B \cup C)=(A \cap B) \cup(A \cap C)$$

3. Demuestre que:
$$
A-B=A \cap B^{\prime}
$$
Nota: $B^{\prime}$ es el complemento del conjunto $B$.
\textit{Doble contención:}

\begin{enumerate}
    \item $A-B\subset A\cap B'$\newline 
    $x\in A\land x\not\in B\to x\not\in B\Longleftrightarrow x\in B'\to x\in A\land x\in B'\to A-B\subset A\cap B'$.
    \item $A\cap B'=A-B$\newline 
    $x\in A\land x\in B'\to x\in B'\Longleftrightarrow x\not\in B \to x\in A \land x\not \in B\to A\cap B'\subset A-B$. 
\end{enumerate}

$\therefore A-B=A \cap B^{\prime}$
 

\end{document}