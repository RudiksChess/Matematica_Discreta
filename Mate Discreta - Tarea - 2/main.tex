\documentclass[a4paper,12pt]{article}
\usepackage[top = 2.5cm, bottom = 2.5cm, left = 2.5cm, right = 2.5cm]{geometry}
% Unfortunately, LaTeX has a hard time interpreting German Umlaute. The following two lines and packages should help. If it doesn't work for you please let me know.
\usepackage[T1]{fontenc}
\usepackage[utf8]{inputenc}
% The following two packages - multirow and booktabs - are needed to create nice looking tables.
\usepackage{multirow} % Multirow is for tables with multiple rows within one cell.
\usepackage{booktabs} % For even nicer tables.
% As we usually want to include some plots (.pdf files) we need a package for that.
\usepackage{graphicx}
% The default setting of LaTeX is to indent new paragraphs. This is useful for articles. But not really nice for homework problem sets. The following command sets the indent to 0.
\usepackage[spanish]{babel}
\usepackage{setspace}
\setlength{\parindent}{0in}
% Package to place figures where you want them.
\usepackage{float}
% The fancyhdr package let's us create nice headers.
\usepackage{fancyhdr}
\usepackage{amsmath}
\usepackage{amssymb}
\usepackage{natbib}
\usepackage{graphicx}
\usepackage{subcaption}
\usepackage{booktabs}
\usepackage{etoolbox}
\usepackage{amsthm}
\AtBeginEnvironment{align}{\setcounter{equation}{0}}
\newenvironment{solution}
  {\renewcommand\qedsymbol{$\blacksquare$}\begin{proof}[Solución]}
  {\end{proof}}
\pagestyle{fancy}

\fancyhf{}

\lhead{\footnotesize Tarea 2}
\rhead{\footnotesize  Rompich}
\cfoot{\footnotesize \thepage}



\begin{document}
    \thispagestyle{empty} % This command disables the header on the first page.

    \begin{tabular}{p{15.5cm}} % This is a simple tabular environment to align your text nicely
    \begin{tabbing}
    Universidad del Valle de Guatemala \\ 26 de febrero de 2021  \\
    Rudik R. Rompich   - Carné: 19857\\
    \end{tabbing}
    Matemática Discreta 1 - MM2015 - Mario Castillo \\
    \hline % \hline produces horizontal lines.
    \\
    \end{tabular} % Our tabular environment ends here.
    \vspace*{0.3cm} % Now we want to add some vertical space in between the line and our title.
    \begin{center} % Everything within the center environment is centered.
    {\Large \bf Tarea 2
} % <---- Don't forget to put in the right number
        \vspace{2mm}
    \end{center}
    \vspace{0.4cm}

\section{Ejercicios del tema 2}
\subsection{Ejercicio 9}
En la siguiente igualdad de pares ordenados $(y-2,2x-1) =(x-1,y+2)$ encontrar los valores de $"x"$ y $"y"$.\begin{solution}Consideremos el sistema de ecuaciones:
$$\begin{cases}
y-2 =x-1\\
2x-1 =y+2
\end{cases} =
\begin{cases}
y-x=1\\
2x-y = 3\end{cases}$$

Es decir, si tomamos $y=1+x$ y reemplazamos la $y$ en el segundo término $2x-(1+x)=3\implies x=4$. Es decir $y-(4)=1\implies y=5$. Por lo tanto, $y=5$ y $x=4$.
\end{solution}
\subsection{Ejercicio 10}
Si $X=\{1,2,3,4\}$, expresar en forma enumerativa las siguientes relaciones definidas en $X$.
\begin{solution}
Se tienen, por definición, un conjunto $X$, tal que una relación binaria definida sobre X es todo subconjunto $X\times X$, es decir: $Rel \subseteq X\times X$, tal que:
\begin{align}
    X\times X =\{(1,1),(1,2),(1,3),(1,4),(2,1),(2,2),(2,3),(2,4),\\(3,1),(3,2),(3,3),(3,4),(4,1),(4,2),(4,3),(4,4)\}
\end{align}
\begin{enumerate}
    \item $Rel_1 =\{(x,y)|x=y\}=\{(1,1),(2,2),(3,3),(4,4),(5,5)\}$
    \item $Rel_2 =\{(x,y)|3x=y\}=\{(1,3)\}$
    \item $Rel_3 =\{(x,y)|x=5y\}=\emptyset$
    \item $Rel_4 =\{(x,y)|2x=y+1\}=\{(1,1),(2,3)\}$
    \item $Rel_5 =\{(x,y)|x\geq y\}=\{(1,1),(2,2),(3,3),(4,4),(2,1),(3,1),(3,2),(4,1),(4,2),(4,3)\}$
\end{enumerate}
\end{solution}
\subsection{Ejercicio 11}
Analice cada una de las relaciones definidas en el problema anterior e indique si son reflexivas, simétricas, transitivas, antisimétricas o no lo son.
\begin{solution}
Considerando: 
\begin{enumerate}
    \item $Rel_1$; es reflexiva ya que contiene los pares $(a,a)$; es simétrica ya que $(a,b)\in R$ y $(b,a)\in R$
    \item $Rel_2$; es antisimétrica, ya que $(a,b)\in R$ pero $(a,b)\not\in R$; es transitiva, solo tiene un elemento.
    \item $Rel_3$; no es ninguna.
    \item $Rel_4$;es antisimétrica, ya que $(a,b)\in R$ pero $(a,b)\not\in R$
    \item $Rel_5$; es reflexiva ya que contiene los pares $(a,a)$; es antisimétrica, ya que $(a,b)\in R$ pero $(a,b)\not\in R$; es transitiva.
\end{enumerate}
\end{solution}
\subsection{Ejercicio 15}
Analizar cada una de las relaciones dadas. Sea $A$ el conjunto de todos los subconjuntos de un conjunto dado $\mathcal{U}$ y las relaciones definidas en $A$: 
\begin{solution}
Tenemos:
\begin{enumerate}
    \item $Rel_1=\{(C,D)|C,D\in A \text{ y }C\subset D\}$
    \begin{solution}
    Es decir:
    \begin{itemize}
        \item Reflexiva; $\forall C\in A \implies C\subset C\implies (C, C)\in A$. Se cumple, ya que la todo conjunto está contenido en sí mismo.
        \item Simétrica; $\forall C,D \in A\implies C\subset D$. Pero entonces $\implies D\not\subset C$. Entonces no es simétrica.
        \item Antisimétrica; $\forall C,D \in A$, entonces sabemos que $C\subset D$ o $D\subset C$, es decir $(C,D)\in A$ o $(D,C)\in A$. Entonces es antisimétrica.
        \item Transitiva; $\forall C, D, E\in A$, entonces: $(C,D)\in A$ y $(D,E)\in A\implies C\subset D$ y $D\subset E\implies C\subset E\implies (C,E)\in A$
    \end{itemize}
    
    \end{solution}
    \item $Rel_2= \{(C,D)| C,D\in A\text{ y } C\cap D=\emptyset \}$
    \begin{itemize}
        \item Reflexiva; $\forall C \in A\implies C\cap C\not=\emptyset$. No es reflexiva.
        \item Simétrica $\forall C, D$, entonces $C\cap D\implies D\cap C\implies (C,D)\in A y (D,C)\in A$
        \item Antisimétrica. No, ya que la intersección es conmutativa. 
        \item Transitiva: $\forall C,D,E \in A$, entonces, $C\cap D$ y $D\cap E\implies C\cap E\implies (C,E)\in A$. Es transitiva.
    \end{itemize}
\end{enumerate}
\end{solution}
\subsection{Ejercicio 19}
Sea $\mathbf{Z}$ el conjunto de números enteros. Sea $Rel(\mathbf{Z})$ la relación definida en $\mathbf{Z}$ por: 
$$Rel(Z)=\{(a,b)| a\in Z,b\in Z \text{ y } a+b=2^c\}$$
Siendo $2^c=$ múltiplo de 2. ¿Es $Rel(\mathbf{Z})$ una relación de equivalencia?
\begin{solution}
A probar: reflexiva, simétrica y transitiva: 
\begin{enumerate}
    \item Reflexiva, $\forall a\in Z\implies a+a =2a\implies (a,a)\in Z$ es múltiplo de 2. Entonces es reflexiva. 
    \item Simétrica, $\forall a,b \in Z\implies a+b =b+a =2^c\implies (a,b)= (b,a) \in Z$ es múltiplo de 2. Entonces es simétrica. 
    \item Transitiva, $\forall a,b,c \in Z\implies a+b=2^{c_2}$ y $b+c=2^{c_1}$ $\implies a+b+b+c=2^{c_1}+2^{c_2}\implies a+c=2^{c_1+c_2}-2b$. Entonces, es transitiva. 
\end{enumerate}
$\therefore$ es una relación de equivalencia. 
\end{solution}
\subsection{Ejercicio 22}
Sea $\mathbf{Z}$ el conjunto de números enteros. Sea $Rel(\mathbf{Z})$ la relación definida en $\mathbf{Z}$ por: 
$$Rel(Z)=\{(a,b)| a\in Z,b\in Z \text{ y } a=3b\}$$
¿Es $Rel(\mathbf{Z})$ una relación de equivalencia?
\begin{solution}
A probar: reflexiva, simétrica y transitiva:
\begin{enumerate}
    \item Reflexiva, $\forall a\in Z\implies a=3a \implies (a,a)\in Z$. Es transitiva.
    \item Simétrica, $\forall a,b\in Z\implies a=3b$ pero $b=3a$ no es igual. Entonces, no es simétrica.
\end{enumerate}
$\therefore$ no es una relación de equivalencia.
\end{solution}

\subsection{Ejercicio 26}
26. Sea $\mathcal{P}(\mathrm{A})$ la familia de subconjuntos de un conjunto dado A. (Por ejemplo, si
$$
P(A)=\{\emptyset,\{a\},\{b\},\{c\},\{a, b\},\{a, c\},\{b, c\},\{a, b, c\}\}
$$
Sea $\mathrm{B}$ un subconjunto fijo de A. (Por ejemplo, sea $\mathrm{B}=\{\mathrm{a}, \mathrm{b}\}$,) Se definen las siguientes funciones:
\begin{enumerate}
\item  $ \mathrm{f}_{1}: P(\mathrm{A}) \rightarrow P(\mathrm{A}) \ni \mathrm{f}_{1}(\mathrm{X})=\mathrm{X}^{\mathrm{c}}
$
\begin{solution}
\begin{align}
    f_1(\emptyset) &=\{\{a\},\{b\},\{c\},\{a, b\},\{a, c\},\{b, c\},\{a, b, c\}\} \\
    f_1(\{a\}) &=\{\emptyset,\{b\},\{c\},\{a, b\},\{a, c\},\{b, c\},\{a, b, c\}\} \\
    f_1(\{b\}) &=\{\emptyset,\{a\},\{c\},\{a, b\},\{a, c\},\{b, c\},\{a, b, c\}\} \\
    f_1(\{c\}) &=\{\emptyset,\{a\},\{b\},\{a, b\},\{a, c\},\{b, c\},\{a, b, c\}\} \\
    f_1(\{a, b\}) &= \{\emptyset,\{a\},\{b\},\{c\},\{a, c\},\{b, c\},\{a, b, c\}\}\\
    f_1(\{a, c\}) &= \{\emptyset,\{a\},\{b\},\{c\},\{a, b\},\{b, c\},\{a, b, c\}\}\\
    f_1(\{b, c\}) &= \{\emptyset,\{a\},\{b\},\{c\},\{a, b\},\{a, c\},\{a, b, c\}\}\\
    f_1(\{a, b, c\}) &=\{\emptyset,\{a\},\{b\},\{c\},\{a, b\},\{a, c\},\{b, c\}\}
\end{align}
\end{solution}
\item  $\mathrm{f}_{2}: P(\mathrm{A}) \rightarrow P(\mathrm{A}) \ni \mathrm{f}_{2}(\mathrm{X})=\mathrm{X} \cup \mathrm{B}$
\begin{solution}
\begin{align}
\intertext{Considerando $B=\{a,b\}$}
    f_2(\emptyset) &= \{\emptyset,a,b\}\\
    f_2(\{a\}) &=\{a,b\} \\
    f_2(\{b\}) &=\{a,b\} \\
    f_2(\{c\}) &=\{a,b,c\} \\
    f_2(\{a, b\}) &=\{a,b\} \\
    f_2(\{a, c\}) &=\{a,b,c\} \\
    f_2(\{b, c\}) &=\{a,b,c\} \\
    f_2(\{a, b, c\}) &=\{a,b,c\}
\end{align}
\end{solution}
\item  $\mathrm{f}_{3}:P(\mathrm{A}) \rightarrow P(\mathrm{A}) \ni \mathrm{f}_{3}(\mathrm{X})=\mathrm{X}-\mathrm{B}\forall \mathrm{X} \in \mathcal{P}(\mathrm{A})$
\begin{solution}
\begin{align}
\intertext{Considerando $B=\{a,b\}$}
    f_3(\emptyset) &= \emptyset \\
    f_3(\{a\}) &=\{b\} \\
    f_3(\{b\}) &= \{a\} \\
    f_3(\{c\}) &=\{c\} \\
    f_3(\{a, b\}) &= \emptyset \\
    f_3(\{a, c\}) &= \{c\} \\
    f_3(\{b, c\}) &= \{c\}\\
    f_3(\{a, b, c\}) &= \{c\}
\end{align}
\end{solution}
\item  $\mathrm{f}_{4}: P(\mathrm{A}) \rightarrow P(\mathrm{A}) \ni \mathrm{f}_{4}(\mathrm{X})=B-\mathrm{X}\forall \mathrm{X} \in P(\mathrm{A})$
\begin{solution}
\begin{align}
\intertext{Considerando $B=\{a,b\}$}
    f_4(\emptyset) &= \{a,b\} \\
    f_4(\{a\}) &= \{b\}\\
    f_4(\{b\}) &= \{a\}\\
    f_4(\{c\}) &= \{a,b\}\\
    f_4(\{a, b\}) &= \emptyset\\
    f_4(\{a, c\}) &= \{b\}\\
    f_4(\{b, c\}) &= \{a\}\\
    f_4(\{a, b, c\}) &= \emptyset\\
\end{align}
\end{solution}
\item  $ \mathrm{f}_{5}: P(\mathrm{A}) \rightarrow P(\mathrm{A}) \ni \mathrm{f}_{5}(\mathrm{X})=\mathrm{B} \cap \mathrm{X}\forall\mathrm{X} \in P(\mathrm{A})
$
\begin{solution}
\begin{align}
\intertext{Considerando $B=\{a,b\}$}
    f_5(\emptyset) &=\emptyset\\ 
    f_5(\{a\}) &=\{a\} \\
    f_5(\{b\}) &= \{b\}\\
    f_5(\{c\}) &= \emptyset\\
    f_5(\{a, b\}) &= \{a,b\}\\
    f_5(\{a, c\}) &= \{a\}\\
    f_5(\{b, c\}) &= \{b\}\\
    f_5(\{a, b, c\}) &=\{a,b\}\\
\end{align}
\end{solution}
\item $f_6: P(A)\to P(A)\ni f_6(X)=(B\cap X)^c\forall X\in P(A) $
\begin{solution}
\begin{align}
\intertext{Considerando $\{\emptyset,\{a\},\{b\},\{c\},\{a, b\},\{a, c\},\{b, c\},\{a, b, c\}\} $}
    f_6(\emptyset) &= \{\{a\},\{b\},\{c\},\{a, b\},\{a, c\},\{b, c\},\{a, b, c\}\} \\
    f_6(\{a\}) &=\{\emptyset,\{b\},\{c\},\{a, b\},\{a, c\},\{b, c\},\{a, b, c\}\} \\
    f_6(\{b\}) &= \{\emptyset,\{a\},\{c\},\{a, b\},\{a, c\},\{b, c\},\{a, b, c\}\} \\
    f_6(\{c\}) &= \{\emptyset,\{a\},\{b\},\{a, b\},\{a, c\},\{b, c\},\{a, b, c\}\}\\
    f_6(\{a, b\}) &= \{\emptyset,\{a\},\{b\},\{c\},\{a, c\},\{b, c\},\{a, b, c\}\}\\
    f_6(\{a, c\}) &=  \{\emptyset,\{a\},\{b\},\{c\},\{a, b\},\{b, c\},\{a, b, c\}\}\\
    f_6(\{b, c\}) &=  \{\emptyset,\{a\},\{b\},\{c\},\{a, b\},\{a, c\},\{a, b, c\}\}\\
    f_6(\{a, b, c\}) &=\{\emptyset,\{a\},\{b\},\{c\},\{a, b\},\{a, c\},\{b, c\}\}\\
\end{align}
\end{solution}
\item $f_7: P(A)\to P(A)\ni f_7(X)=(B\cap X)^c\forall X\in P(A) $
\begin{solution}
\begin{align}
\intertext{Considerando $\{\emptyset,\{a\},\{b\},\{c\},\{a, b\},\{a, c\},\{b, c\},\{a, b, c\}\} $}
    f_6(\emptyset) &= \{\{a\},\{b\},\{c\},\{a, b\},\{a, c\},\{b, c\},\{a, b, c\}\} \\
    f_6(\{a\}) &=\{\emptyset,\{b\},\{c\},\{a, b\},\{a, c\},\{b, c\},\{a, b, c\}\} \\
    f_6(\{b\}) &= \{\emptyset,\{a\},\{c\},\{a, b\},\{a, c\},\{b, c\},\{a, b, c\}\} \\
    f_6(\{c\}) &= \{\emptyset,\{a\},\{b\},\{a, b\},\{a, c\},\{b, c\},\{a, b, c\}\}\\
    f_6(\{a, b\}) &= \{\emptyset,\{a\},\{b\},\{c\},\{a, c\},\{b, c\},\{a, b, c\}\}\\
    f_6(\{a, c\}) &=  \{\emptyset,\{a\},\{b\},\{c\},\{a, b\},\{b, c\},\{a, b, c\}\}\\
    f_6(\{b, c\}) &=  \{\emptyset,\{a\},\{b\},\{c\},\{a, b\},\{a, c\},\{a, b, c\}\}\\
    f_6(\{a, b, c\}) &=\{\emptyset,\{a\},\{b\},\{c\},\{a, b\},\{a, c\},\{b, c\}\}\\
\end{align}
\end{solution}
\end{enumerate}

Encuentre todas las imágenes correspondientes a las 7 funciones dadas para todos los valores de de $X$ perteneciente a $P(A)$ que se definió al principio de este ejercicio. 

\subsection{Ejercicio 34}
Del ejercicio 26 indique cuáles de las funciones son sobre, biyectivas y cuáles no son ningún tipo de ellas.
\begin{solution}
Tenemos:
\begin{enumerate}
    \item $f_1$ inyectiva y sobreyectiva; biyectiva.
    \item $f_2$ no es ninguna
    \item $f_3$ no es ninguna
    \item $f_4$ no es ninguna.
    \item $f_5$ no es ninguna
    \item $f_6$ inyectiva y sobreyectiva; biyectiva.
    \item $f_7$ inyectiva y sobreyectiva; biyectiva.
\end{enumerate}
\end{solution}
\subsection{Ejercicio 35}
Sean $f: R\to R\ni f(x)=x+2$\newline 
$g: R\to R\ni g(x)=x^2$\newline 
Encontrar: \begin{enumerate}
    \item $f\circ g= f(g)= (x^2)+2$
    \item $g\circ f=g(f) = (x+2)^2$
    \item $(f\circ g)(5) = (25)+2)=27$
    \item $(g\circ f)(-2)= (-2+2)^2=0$
\end{enumerate}
\section{Hoja de trabajo 2}
\subsection{Ejercicio 4}

Indíquese cuáles entre las siguientes funciones son inyectivas, cuáles sobreyectivas y cuáles biyectivas.\newline\newline

Nota: Por $P(U)$ representaremos la familia de todos los subconjuntos de un conjunto U.
\begin{enumerate}
\item  $\mathrm{f}: \mathrm{Z} \rightarrow \mathrm{Z} \ni \mathrm{f}(\mathrm{x})=\mathrm{x}+3$, inyectiva, sobreyectiva y biyectiva.
\item  $f: R \rightarrow R \ni f(x)=x^{3}$  inyectiva, sobreyectiva y biyectiva.
\item  $f: R \rightarrow R \ni f(x)=x^{2}$ ninguna.
\item  $f: Z \rightarrow Z^{+} \ni f(x)=x^{2}+1$ ninguna.
\item  $f: R \rightarrow(R^+\cup\{0\}) \ni f(x)=\operatorname{max}\{x,-x\}$, inyectiva, sobreyectiva y biyectiva.
\item  $f: P(U) \rightarrow P(U) \quad \ni f(X)=X \cup A, \forall X \in P(U),$ donde $A$ es un
subconjunto fijo de U; ninguno, por el ejericicio 34 de la primera parte.
\item $\mathrm{f}: \mathrm{P}(\mathrm{U}) \rightarrow \mathrm{P}(\mathrm{U}) \ni \mathrm{f}(\mathrm{X})=\mathrm{U}-\mathrm{X}, \forall \mathrm{X} \in \mathrm{P}(\mathrm{U})$, ninguno, por el ejericicio 34 de la primera parte.
\item $\mathrm{f}: \mathrm{P}(\mathrm{U}) \rightarrow \mathrm{P}(\mathrm{U}) \ni \mathrm{f}(\mathrm{X})=\mathrm{A}-\mathrm{X}, \forall \mathrm{X} \in \mathrm{P}(\mathrm{U})$, ninguno, por el ejericicio 34 de la primera parte.
\item $\mathrm{f}: \mathrm{P}(\mathrm{U}) \rightarrow \mathrm{P}(\mathrm{U}) \ni \mathrm{f}(\mathrm{X})=(\mathrm{X}-\mathrm{A}) \cup(\mathrm{A}-\mathrm{X}), \forall \mathrm{X} \in \mathrm{P}(\mathrm{U})$, ninguno, por el ejericicio 34 de la primera parte.
\end{enumerate}
\subsection{Ejercicio 6}
Demuéstrese las siguientes proposiciones: 
\begin{enumerate}
    \item Si $f\circ g$ está definida y $g$ y $f$ son inyectivas, entonces $f\circ g$ es inyectiva.
    \begin{proof}
    A probar que $f\circ g$ es inyectiva. Tenemos por hipótesis que $f$ y $g$ son inyectivas, es decir: $f(x)=f(x') \leftrightarrow x=x'$ y $g(x)=g(x') \leftrightarrow x=x'$. Entonces, si consideramos $f(g(x))=f(g(x'))\leftrightarrow g(x)=g(x')\leftrightarrow x=x'$. Probando que es inyectiva. 
    \end{proof}
    \item Si $g\circ f$ está definida y $g$ y $f$ son sobreyectivas, entonces $g\circ f$ es sobreyectiva.
    \begin{proof}
    A probar que $g\circ f$ es sobreyectiva. Por hipótesis sabemos que $f$ y $g$ son sobreyectivas. Entonces, supongamos que $g: Y\to Z\to \exists y\in Y$ y $z\in Z\ni g(y)=z$. Además $f: X\to Y\to \exists x\in X $ y $y\in Y\ni f(x)= y$. Entonces, $g(f(x))=g(y)=z$
    \end{proof}
    \item Si $g\circ f$ está definida y $g$ y $f$ son biyectivas, entonces $g\circ f$ es biyectiva.
    \begin{proof}
    Por la demostración (1) y la demostración (2) sabemos que la composición de funciones de dos funciones inyectivas es inyectiva y la composición de dos funciones sobreyectivas es sobreyectiva, entonces es trivial considerar que dos funciones biyectivas (inyectivas y sobreyectivas) es biyectiva. 
    \end{proof}
\end{enumerate}
\subsection{Ejercicio 8}
Sea $A=\{0,1,2,3,4,5,6,7,8,9\}$
Sea $\mathrm{R}_{1}$ y $\mathrm{R}_{2}$ las relaciones definidas en $\mathrm{A}$ :
$$
\begin{array}{l}
\mathrm{R}_{1}=\{(0,1),(1,2),(4,5),(8,9),(9,9)\} \\
\mathrm{R}_{2}=\{(0,0),(1,1),(3,4),(4,5),(8,8),(8,9)\} .
\end{array}
$$
Descríbanse en forma enumerativa las relaciones $\mathrm{R}_{1} \cup \mathrm{R}_{2}, \mathrm{R}_{1} \mathrm{R}_{2}, \mathrm{R}_{1}-\mathrm{R}_{2}, R_{2}-R_{1}
$
\begin{solution}
Entonces, tenemos:
\begin{enumerate}
    \item $R_1\cup R_2=\{(0,1),(1,2),(4,5),(8,9),(9,9),(0,0),(1,1),(3,4),(8,8)\}$
    \item $R_1R_2 =\{(0,1),(1,2),(4,5),(8,9),(9,9)\}\cdot\{(0,0),(1,1),(3,4),(4,5),(8,8),(8,9)\}$
    \item $R_1-R_2 =\{(0,1),(1,2),(9,9)\}$
    \item $R_2-R_1= \{(0,0),(1,1),(3,4),(8,8)\}$
\end{enumerate}
\end{solution}
\subsection{Ejercicio 10}
Demuéstrese que si $R_1$ y $R_2$ son equivalencias definidas en un mismo conjunto $A$, entonces $R_1 \cap R_2$ es también una equivalencia definida en $A$.
\begin{proof}
A probar: $R_1\cap R_2$ es una equivalencia. Por hipótesis, sabemos que $R_1$ y $R_2$ son equivalencias (cumplen con la reflexividad, simetría y transitividad). Entonces procederemos a demostrar que $R_1\cap R_2 \implies (R_1,R_2)\in A$ cumple con esas 3 propiedades: 
\begin{enumerate}
    \item Reflexividad, supóngase que $\forall R_1\in A\implies R_1\cap R_1= \emptyset\implies \emptyset \in A$ cumple con la reflexividad.
    \item Simetría, $\forall R_1,R_2\implies R_1\cap R_2 =R_2\cap R_1 \implies (R_1,R_2)\in A$ y $(R_2,R_1)\in A$  cumple con la simetría.
    \item Transitividad $\forall R_1,R_2,R_3\implies R_1\cap R_2$ y $R_2\cap R_3\implies R_1\cap R_3\implies (R_1,R_3)\in A$. Entonces cumple con la transitividad.
\end{enumerate}
$\therefore R_1\cap R_2$ es una equivalencia definida en $A$
\end{proof}

\subsection{Ejercicio 12}
Sea $A=\{1,2,3,4, \ldots, 13,14,15\} .$ Sea
$$
\mathrm{R}=\triangle_{\mathrm{A}}\cup\{(1,4),(4,1),(3,4),(4,3),(3,1),(1,3)\}
$$
Pruébese que $R$ es una equivalencia mostrando cuál es la partición de A que la induce.
\begin{solution}
Sea $\triangle_A=\{(1,1),(2,2),(3,3),...,(14,15),(15,15)\}$
\begin{align}
    \implies \triangle_A\cup \{(1,4),(4,1),(3,4),(4,3),(3,1),(1,3)\}=\\
    =\{(1,1),(2,2),(3,3),...,(14,14),(15,15),(1,4),(4,1),(3,4),(4,3),(3,1),(1,3)\}
\end{align}
 Mostrando que $\forall r \in R, r\neq\emptyset$: se cumple, ya que no hay ningún elemento $\emptyset$ en R.\newline\newline 
 Por otro lado, se debe cumplir que $r_i$ y $r_j\in R$ debemos demostrar que $r_i\cap r_j = \emptyset$ si $r_i\neq r_j$, es trivial darse cuenta que no hay ningún $r_i\cap R_j\neq \emptyset$. Por lo tanto, $R$ es una relación de equivalencia.
    
\end{solution}


\end{document}