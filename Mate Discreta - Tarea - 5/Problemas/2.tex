\section{Problema 2}

¿Cuántas palabras diferentes se pueden formar con las letras de la palabra MOROSO?
\begin{solution} Tomamos como referencia el término \textbf{palabras} para referirse a un conjunto ordenado de letras. Notamos que tenemos 3 O's indistinguibles,
\begin{center}
    M\textcolor{green}{O}R\textcolor{red}{O}S\textcolor{orange}{O}\\
    M\textcolor{red}{O}R\textcolor{green}{O}S\textcolor{orange}{O}\\
    M\textcolor{red}{O}R\textcolor{orange}{O}S\textcolor{green}{O}\\
    $\vdots$
\end{center}
Vamos a usar la fórmula para permutaciones de elementos indistinguibles, 
$$P(n,\{n_1,n_2,\cdots, n_k\})=\frac{n!}{n_1!n_2!\cdots n_k!}$$
Entonces, 
$$P(6,3)=\frac{6!}{3!}=120 \text{ palabras que se pueden juntar con MOROSO.}$$
\end{solution}
¿Cuántas de estas tienen las tres O’s juntas?

\begin{solution}
La estrategia consiste en considerar OOO como una letra gigante, es decir que tenemos la cadena de elementos, 
$$
    \underbrace{M}_{L_1}\underbrace{R}_{L_2}\underbrace{S}_{L_3}\underbrace{\textbf{OOO}}_{L_4}
$$
Es decir, podemos hacer una permutación: 
$$P(4,4)=4!=24 \text{ palabras tienen OOO juntas.}$$
\end{solution}