\section{Problema 7}
\begin{enumerate}
    \item ¿Cuántas cadenas de 12 bits tienen por lo menos cuatro 1’s?
    \begin{solution}
    \begin{align*}
        C(12,4) &= 495\\
        C(12,5) &= 792\\
        C(12,6) &= 924\\
        C(12,7) &= 792\\
        C(12,8) &= 495\\
        C(12,9) &= 220\\
        C(12,10) &=66 \\
        C(12,11) &=12 \\
        C(12,12) &= 1\\
                 &\quad \_\_\_+\\
                 & \quad 3797 \text{ cadenas.}
    \end{align*}
    \end{solution}
    \item ¿Cuántas cadenas binarias de longitud 10 comienzan con 111 o terminan con 101 o ambos?
    \begin{enumerate}
        \item Por el principio de la multiplicación:
        $$2^7=128 \text{ cadenas.}$$ comienzan en 111.
        \item Por el principio de la multiplicación:
        $$2^7=128\text{ cadenas.}$$ terminan en 101.
        \item Con «ambos» vamos a referirnos a que inician en 111 y terminan en 101; es decir, 
        $$2^4=16\text{ cadenas.}$$
    \end{enumerate}
\end{enumerate}

