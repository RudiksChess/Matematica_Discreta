\documentclass[a4paper,12pt]{article}
\usepackage[top = 2.5cm, bottom = 2.5cm, left = 2.5cm, right = 2.5cm]{geometry}
\usepackage[T1]{fontenc}
\usepackage[utf8]{inputenc}
\usepackage{multirow} 
\usepackage{booktabs} 
\usepackage{graphicx}
\usepackage[spanish]{babel}
\usepackage{setspace}
\usepackage{svg}
\usepackage{emoji}
\setlength{\parindent}{0in}
\usepackage{float}
\usepackage{fancyhdr}
\usepackage{amsmath}
\usepackage{amssymb}
\usepackage{amsthm}
\usepackage{natbib}
\usepackage{graphicx}
\usepackage{subcaption}
\usepackage{booktabs}
\usepackage{etoolbox}
\usepackage{apalike}
\usepackage{minibox}
\usepackage{hyperref}
\usepackage{xcolor}
\usepackage{tcolorbox}
\usepackage{tikz}
\usepackage{mathtools}
\DeclarePairedDelimiter\ceil{\lceil}{\rceil}
\DeclarePairedDelimiter\floor{\lfloor}{\rfloor}
\newcommand{\linita}{\noindent\rule{\textwidth}{1pt}}
\AtBeginEnvironment{align}{\setcounter{equation}{0}}
\newenvironment{solution}
  {\renewcommand\qedsymbol{$\square$}\begin{proof}[\textcolor{blue}{Solución}]}
  {\end{proof}}

\pagestyle{fancy}

\fancyhf{}

\lhead{\footnotesize Matemática Discreta - }
\rhead{\footnotesize  Rudik Roberto Rompich}
\cfoot{\footnotesize \thepage}

\begin{document}
    \thispagestyle{empty} 
    \begin{tabular}{p{15.5cm}}
    \begin{tabbing}
    \textbf{Universidad del Valle de Guatemala} \\
    Departamento de Matemática\\
    Licenciatura en Matemática Aplicada\\\\
   \textbf{Estudiante:} Rudik Roberto Rompich\\
   \textbf{E-mail:} \textcolor{blue}{ \href{mailto:rom19857@uvg.edu.gt}{rom19857@uvg.edu.gt}}\\
   \textbf{Carné:} 19857
    \end{tabbing}
    \begin{center}
        MM2015 - Matemática Discreta - Catedrático: Mario Castillo\\
        \today
    \end{center}\\
    \hline
    \\
    \end{tabular} 
    \vspace*{0.3cm} 
    \begin{center} 
    {\Large \bf Tarea 5
} 
        \vspace{2mm}
    \end{center}
    \vspace{0.4cm}
    %---------------------------
%\begin{tcolorbox}[colback=gray!15,colframe=black!1!black,title=A nice heading]
%\end{tcolorbox}

%\fbox{lol}

%---------------------------
\section{Problema 1}

Supongamos que no se permiten repeticiones.
\begin{enumerate}
    \item ¿Cuántos números de 3 dígitos se pueden formar con los siete dígitos 1, 2, 5, 6, 8, 9 y 0?
    \begin{solution} Tenemos el conjunto de datos: $S_0= \{0,1,2,5,6,8,9\}$ con cardinalidad 7. Notamos que es un problema de $r$-permutación. La forma esperada es la siguiente: 
    $$\underbrace{X}_{D_1}\underbrace{X}_{D_2} \underbrace{X}_{D_3}=P(n,r)=\square$$
     Sin embargo, notamos que el dígito 0 podría generar problemas, ya que podrían generarse números de 3 cifras como 017, 007, etcétera; los cuales no serían números válidos. Por lo cual, excluimos el 0 y el conjunto sería $S_1= \{1,2,5,6,8,9\}$ con cardinalidad 6 para $D_1$; por su parte $D_2$ (ya que el 0 fue eliminado en $D_1$, entonces en $D_2$ sí es posible que haya un cero) y $D_3$ pertenecen a $S$. Ahora, por el principio del producto (y como no se pueden repetir):  
    $$\underbrace{6}_{D_1}\cdot \underbrace{6}_{D_2}\cdot \underbrace{5}_{D_3}=P(6,1)\cdot P(6,2)=180 \text{ dígitos.}$$
    \end{solution}
    \item ¿Cuántos de estos son menores que 400?
    \begin{solution}
    Analizamos la situación, es decir que en la posición $D_1$ no pueden estar los números: $\{0,5,6,8,9\}$ ya que en el caso de $0$ no sería un número de 3 dígitos y en el caso de los demás números, sería un número mayor a 400. Entonces (ya que no se pueden repetir), 
    
    $$\underbrace{2}_{D_1}\cdot \underbrace{6}_{D_2}\cdot \underbrace{5}_{D_3}=P(2,1)\cdot P(6,2)=60 \text{ dígitos.}$$
    \end{solution} 
\end{enumerate}
\section{Problema 2}

¿Cuántas palabras diferentes se pueden formar con las letras de la palabra MOROSO?
\begin{solution} Tomamos como referencia el término \textbf{palabras} para referirse a un conjunto ordenado de letras. Notamos que tenemos 3 O's indistinguibles,
\begin{center}
    M\textcolor{green}{O}R\textcolor{red}{O}S\textcolor{orange}{O}\\
    M\textcolor{red}{O}R\textcolor{green}{O}S\textcolor{orange}{O}\\
    M\textcolor{red}{O}R\textcolor{orange}{O}S\textcolor{green}{O}\\
    $\vdots$
\end{center}
Vamos a usar la fórmula para permutaciones de elementos indistinguibles, 
$$P(n,\{n_1,n_2,\cdots, n_k\})=\frac{n!}{n_1!n_2!\cdots n_k!}$$
Entonces, 
$$P(6,3)=\frac{6!}{3!}=120 \text{ palabras que se pueden juntar con MOROSO.}$$
\end{solution}
¿Cuántas de estas tienen las tres O’s juntas?

\begin{solution}
La estrategia consiste en considerar OOO como una letra gigante, es decir que tenemos la cadena de elementos, 
$$
    \underbrace{M}_{L_1}\underbrace{R}_{L_2}\underbrace{S}_{L_3}\underbrace{\textbf{OOO}}_{L_4}
$$
Es decir, podemos hacer una permutación: 
$$P(4,4)=4!=24 \text{ palabras tienen OOO juntas.}$$
\end{solution}
\section{Problema 3}

Un mazo estándar de 52 cartas consta de 4 palos (corazones, diamantes, espadas y tréboles), cada uno con 13 valores diferentes (A, 2, 3, ..., 10, J, Q, K). En una mano estándar de póker (5 cartas):
  \begin{center}
      

\tikzset{every picture/.style={line width=0.75pt}} %set default line width to 0.75pt        

\begin{tikzpicture}[x=0.75pt,y=0.75pt,yscale=-1,xscale=1]
%uncomment if require: \path (0,300); %set diagram left start at 0, and has height of 300

%Rounded Rect [id:dp6128576032908483] 
\draw  [fill={rgb, 255:red, 255; green, 255; blue, 255 }  ,fill opacity=1 ] (243.85,34.05) .. controls (248.17,34.28) and (250.62,37.69) .. (249.31,41.66) -- (242.21,63.24) .. controls (240.91,67.21) and (236.34,70.25) .. (232.01,70.02) -- (206.07,68.64) .. controls (201.75,68.41) and (199.3,65) .. (200.6,61.03) -- (207.7,39.45) .. controls (209.01,35.48) and (213.58,32.44) .. (217.91,32.67) -- cycle ;
%Rounded Rect [id:dp7795928017300927] 
\draw  [fill={rgb, 255:red, 255; green, 255; blue, 255 }  ,fill opacity=1 ] (236.78,52.97) .. controls (241.11,53.2) and (243.56,56.6) .. (242.25,60.58) -- (235.15,82.16) .. controls (233.84,86.13) and (229.27,89.17) .. (224.95,88.94) -- (199.01,87.56) .. controls (194.68,87.33) and (192.23,83.92) .. (193.54,79.95) -- (200.64,58.37) .. controls (201.95,54.39) and (206.51,51.36) .. (210.84,51.59) -- cycle ;
%Rounded Rect [id:dp38096446948076257] 
\draw  [fill={rgb, 255:red, 255; green, 255; blue, 255 }  ,fill opacity=1 ] (231.68,68.1) .. controls (236,68.33) and (238.45,71.74) .. (237.15,75.71) -- (230.05,97.29) .. controls (228.74,101.26) and (224.17,104.3) .. (219.84,104.07) -- (193.9,102.69) .. controls (189.58,102.46) and (187.13,99.05) .. (188.44,95.08) -- (195.54,73.5) .. controls (196.84,69.53) and (201.41,66.49) .. (205.74,66.72) -- cycle ;
%Rounded Rect [id:dp5901650766630238] 
\draw  [fill={rgb, 255:red, 255; green, 255; blue, 255 }  ,fill opacity=1 ] (226.46,83.99) .. controls (230.78,84.22) and (233.23,87.63) .. (231.92,91.6) -- (224.82,113.18) .. controls (223.52,117.15) and (218.95,120.19) .. (214.62,119.96) -- (188.68,118.58) .. controls (184.36,118.35) and (181.91,114.94) .. (183.21,110.97) -- (190.31,89.39) .. controls (191.62,85.42) and (196.19,82.38) .. (200.52,82.61) -- cycle ;
%Rounded Rect [id:dp16189410272363214] 
\draw  [fill={rgb, 255:red, 255; green, 255; blue, 255 }  ,fill opacity=1 ] (221.35,99.12) .. controls (225.68,99.35) and (228.13,102.76) .. (226.82,106.73) -- (219.72,128.31) .. controls (218.41,132.29) and (213.85,135.32) .. (209.52,135.09) -- (183.58,133.71) .. controls (179.25,133.48) and (176.8,130.08) .. (178.11,126.1) -- (185.21,104.52) .. controls (186.52,100.55) and (191.09,97.51) .. (195.41,97.74) -- cycle ;
%Rounded Rect [id:dp42326113968840884] 
\draw  [fill={rgb, 255:red, 255; green, 255; blue, 255 }  ,fill opacity=1 ] (214.76,115.01) .. controls (219.09,115.24) and (221.53,118.65) .. (220.23,122.62) -- (213.13,144.2) .. controls (211.82,148.18) and (207.25,151.21) .. (202.92,150.98) -- (176.99,149.6) .. controls (172.66,149.37) and (170.21,145.97) .. (171.52,141.99) -- (178.62,120.41) .. controls (179.92,116.44) and (184.49,113.4) .. (188.82,113.63) -- cycle ;
%Rounded Rect [id:dp4185585313318074] 
\draw  [fill={rgb, 255:red, 255; green, 255; blue, 255 }  ,fill opacity=1 ] (207.69,133.93) .. controls (212.02,134.16) and (214.47,137.56) .. (213.16,141.54) -- (206.06,163.12) .. controls (204.75,167.09) and (200.19,170.13) .. (195.86,169.9) -- (169.92,168.52) .. controls (165.59,168.29) and (163.14,164.88) .. (164.45,160.91) -- (171.55,139.33) .. controls (172.86,135.35) and (177.43,132.32) .. (181.75,132.55) -- cycle ;
%Rounded Rect [id:dp2920259914907978] 
\draw  [fill={rgb, 255:red, 255; green, 255; blue, 255 }  ,fill opacity=1 ] (202.59,149.06) .. controls (206.92,149.29) and (209.37,152.7) .. (208.06,156.67) -- (200.96,178.25) .. controls (199.65,182.23) and (195.08,185.26) .. (190.76,185.03) -- (164.82,183.65) .. controls (160.49,183.42) and (158.04,180.02) .. (159.35,176.04) -- (166.45,154.46) .. controls (167.76,150.49) and (172.32,147.45) .. (176.65,147.68) -- cycle ;
%Rounded Rect [id:dp7997610326122214] 
\draw  [fill={rgb, 255:red, 255; green, 255; blue, 255 }  ,fill opacity=1 ] (197.37,164.95) .. controls (201.7,165.18) and (204.14,168.59) .. (202.84,172.56) -- (195.74,194.14) .. controls (194.43,198.12) and (189.86,201.15) .. (185.53,200.92) -- (159.6,199.54) .. controls (155.27,199.31) and (152.82,195.91) .. (154.13,191.93) -- (161.23,170.35) .. controls (162.53,166.38) and (167.1,163.34) .. (171.43,163.57) -- cycle ;
%Rounded Rect [id:dp9067963707796727] 
\draw  [fill={rgb, 255:red, 255; green, 255; blue, 255 }  ,fill opacity=1 ] (192.27,180.08) .. controls (196.59,180.31) and (199.04,183.72) .. (197.73,187.69) -- (190.63,209.28) .. controls (189.33,213.25) and (184.76,216.28) .. (180.43,216.05) -- (154.49,214.68) .. controls (150.17,214.45) and (147.72,211.04) .. (149.02,207.07) -- (156.12,185.48) .. controls (157.43,181.51) and (162,178.48) .. (166.33,178.71) -- cycle ;
%Rounded Rect [id:dp21577273271752628] 
\draw  [fill={rgb, 255:red, 255; green, 255; blue, 255 }  ,fill opacity=1 ] (185.32,198.24) .. controls (189.64,198.47) and (192.09,201.88) .. (190.79,205.85) -- (183.69,227.43) .. controls (182.38,231.41) and (177.81,234.44) .. (173.48,234.21) -- (147.55,232.84) .. controls (143.22,232.61) and (140.77,229.2) .. (142.08,225.22) -- (149.18,203.64) .. controls (150.48,199.67) and (155.05,196.64) .. (159.38,196.86) -- cycle ;
%Rounded Rect [id:dp20615672324075307] 
\draw  [fill={rgb, 255:red, 255; green, 255; blue, 255 }  ,fill opacity=1 ] (180.21,213.38) .. controls (184.54,213.61) and (186.99,217.01) .. (185.68,220.99) -- (178.58,242.57) .. controls (177.28,246.54) and (172.71,249.58) .. (168.38,249.35) -- (142.44,247.97) .. controls (138.11,247.74) and (135.67,244.33) .. (136.97,240.36) -- (144.07,218.78) .. controls (145.38,214.8) and (149.95,211.77) .. (154.28,212) -- cycle ;
%Rounded Rect [id:dp3546346081182864] 
\draw  [fill={rgb, 255:red, 255; green, 255; blue, 255 }  ,fill opacity=1 ] (174.99,229.27) .. controls (179.32,229.5) and (181.77,232.9) .. (180.46,236.88) -- (173.36,258.46) .. controls (172.05,262.43) and (167.49,265.47) .. (163.16,265.24) -- (137.22,263.86) .. controls (132.89,263.63) and (130.44,260.22) .. (131.75,256.25) -- (138.85,234.67) .. controls (140.16,230.69) and (144.73,227.66) .. (149.05,227.89) -- cycle ;
%Rounded Rect [id:dp8917377453290999] 
\draw  [fill={rgb, 255:red, 255; green, 255; blue, 255 }  ,fill opacity=1 ] (302.63,38.92) .. controls (306.96,39.15) and (309.41,42.56) .. (308.1,46.53) -- (301,68.12) .. controls (299.69,72.09) and (295.13,75.12) .. (290.8,74.89) -- (264.86,73.52) .. controls (260.53,73.29) and (258.08,69.88) .. (259.39,65.91) -- (266.49,44.32) .. controls (267.8,40.35) and (272.36,37.32) .. (276.69,37.55) -- cycle ;
%Rounded Rect [id:dp2208950823157233] 
\draw  [fill={rgb, 255:red, 255; green, 255; blue, 255 }  ,fill opacity=1 ] (295.57,57.84) .. controls (299.89,58.07) and (302.34,61.48) .. (301.03,65.45) -- (293.93,87.03) .. controls (292.63,91) and (288.06,94.04) .. (283.73,93.81) -- (257.79,92.43) .. controls (253.47,92.2) and (251.02,88.79) .. (252.32,84.82) -- (259.42,63.24) .. controls (260.73,59.27) and (265.3,56.23) .. (269.63,56.46) -- cycle ;
%Rounded Rect [id:dp7590972467883634] 
\draw  [fill={rgb, 255:red, 255; green, 255; blue, 255 }  ,fill opacity=1 ] (290.46,72.97) .. controls (294.79,73.2) and (297.24,76.61) .. (295.93,80.58) -- (288.83,102.16) .. controls (287.52,106.14) and (282.96,109.17) .. (278.63,108.94) -- (252.69,107.57) .. controls (248.36,107.34) and (245.91,103.93) .. (247.22,99.95) -- (254.32,78.37) .. controls (255.63,74.4) and (260.2,71.37) .. (264.52,71.59) -- cycle ;
%Rounded Rect [id:dp2711627981575816] 
\draw  [fill={rgb, 255:red, 255; green, 255; blue, 255 }  ,fill opacity=1 ] (285.24,88.86) .. controls (289.57,89.09) and (292.02,92.5) .. (290.71,96.47) -- (283.61,118.05) .. controls (282.3,122.03) and (277.74,125.06) .. (273.41,124.83) -- (247.47,123.45) .. controls (243.14,123.22) and (240.69,119.82) .. (242,115.84) -- (249.1,94.26) .. controls (250.41,90.29) and (254.98,87.25) .. (259.3,87.48) -- cycle ;
%Rounded Rect [id:dp23825909007295665] 
\draw  [fill={rgb, 255:red, 255; green, 255; blue, 255 }  ,fill opacity=1 ] (280.14,104) .. controls (284.47,104.22) and (286.91,107.63) .. (285.61,111.61) -- (278.51,133.19) .. controls (277.2,137.16) and (272.63,140.2) .. (268.3,139.97) -- (242.37,138.59) .. controls (238.04,138.36) and (235.59,134.95) .. (236.9,130.98) -- (244,109.4) .. controls (245.3,105.42) and (249.87,102.39) .. (254.2,102.62) -- cycle ;
%Rounded Rect [id:dp07362516789869411] 
\draw  [fill={rgb, 255:red, 255; green, 255; blue, 255 }  ,fill opacity=1 ] (273.54,119.88) .. controls (277.87,120.11) and (280.32,123.52) .. (279.01,127.5) -- (271.91,149.08) .. controls (270.61,153.05) and (266.04,156.08) .. (261.71,155.85) -- (235.77,154.48) .. controls (231.44,154.25) and (229,150.84) .. (230.3,146.87) -- (237.4,125.29) .. controls (238.71,121.31) and (243.28,118.28) .. (247.61,118.51) -- cycle ;
%Rounded Rect [id:dp5777585743427922] 
\draw  [fill={rgb, 255:red, 255; green, 255; blue, 255 }  ,fill opacity=1 ] (266.48,138.8) .. controls (270.81,139.03) and (273.25,142.44) .. (271.95,146.41) -- (264.85,167.99) .. controls (263.54,171.97) and (258.97,175) .. (254.65,174.77) -- (228.71,173.39) .. controls (224.38,173.16) and (221.93,169.76) .. (223.24,165.78) -- (230.34,144.2) .. controls (231.64,140.23) and (236.21,137.19) .. (240.54,137.42) -- cycle ;
%Rounded Rect [id:dp32566653068254103] 
\draw  [fill={rgb, 255:red, 255; green, 255; blue, 255 }  ,fill opacity=1 ] (261.38,153.93) .. controls (265.7,154.16) and (268.15,157.57) .. (266.84,161.54) -- (259.74,183.13) .. controls (258.44,187.1) and (253.87,190.13) .. (249.54,189.9) -- (223.6,188.53) .. controls (219.28,188.3) and (216.83,184.89) .. (218.13,180.92) -- (225.23,159.33) .. controls (226.54,155.36) and (231.11,152.33) .. (235.44,152.56) -- cycle ;
%Rounded Rect [id:dp6945029046945058] 
\draw  [fill={rgb, 255:red, 255; green, 255; blue, 255 }  ,fill opacity=1 ] (256.15,169.82) .. controls (260.48,170.05) and (262.93,173.46) .. (261.62,177.43) -- (254.52,199.02) .. controls (253.22,202.99) and (248.65,206.02) .. (244.32,205.79) -- (218.38,204.42) .. controls (214.05,204.19) and (211.61,200.78) .. (212.91,196.81) -- (220.01,175.22) .. controls (221.32,171.25) and (225.89,168.22) .. (230.22,168.45) -- cycle ;
%Rounded Rect [id:dp4709488794155279] 
\draw  [fill={rgb, 255:red, 255; green, 255; blue, 255 }  ,fill opacity=1 ] (251.05,184.96) .. controls (255.38,185.19) and (257.83,188.59) .. (256.52,192.57) -- (249.42,214.15) .. controls (248.11,218.12) and (243.55,221.16) .. (239.22,220.93) -- (213.28,219.55) .. controls (208.95,219.32) and (206.5,215.91) .. (207.81,211.94) -- (214.91,190.36) .. controls (216.22,186.38) and (220.78,183.35) .. (225.11,183.58) -- cycle ;
%Rounded Rect [id:dp16409079980236996] 
\draw  [fill={rgb, 255:red, 255; green, 255; blue, 255 }  ,fill opacity=1 ] (244.1,203.12) .. controls (248.43,203.35) and (250.88,206.75) .. (249.57,210.73) -- (242.47,232.31) .. controls (241.17,236.28) and (236.6,239.32) .. (232.27,239.09) -- (206.33,237.71) .. controls (202,237.48) and (199.55,234.07) .. (200.86,230.1) -- (207.96,208.52) .. controls (209.27,204.54) and (213.84,201.51) .. (218.16,201.74) -- cycle ;
%Rounded Rect [id:dp3388631409567209] 
\draw  [fill={rgb, 255:red, 255; green, 255; blue, 255 }  ,fill opacity=1 ] (239,218.25) .. controls (243.33,218.48) and (245.78,221.89) .. (244.47,225.86) -- (237.37,247.44) .. controls (236.06,251.41) and (231.49,254.45) .. (227.17,254.22) -- (201.23,252.84) .. controls (196.9,252.61) and (194.45,249.21) .. (195.76,245.23) -- (202.86,223.65) .. controls (204.17,219.68) and (208.73,216.64) .. (213.06,216.87) -- cycle ;
%Rounded Rect [id:dp43204958512816616] 
\draw  [fill={rgb, 255:red, 255; green, 255; blue, 255 }  ,fill opacity=1 ] (233.78,234.14) .. controls (238.11,234.37) and (240.56,237.78) .. (239.25,241.75) -- (232.15,263.33) .. controls (230.84,267.3) and (226.27,270.34) .. (221.95,270.11) -- (196.01,268.73) .. controls (191.68,268.5) and (189.23,265.09) .. (190.54,261.12) -- (197.64,239.54) .. controls (198.94,235.57) and (203.51,232.53) .. (207.84,232.76) -- cycle ;
%Rounded Rect [id:dp7661890445479147] 
\draw  [fill={rgb, 255:red, 255; green, 255; blue, 255 }  ,fill opacity=1 ] (359.58,41.66) .. controls (363.91,41.89) and (366.36,45.3) .. (365.05,49.27) -- (357.95,70.85) .. controls (356.65,74.83) and (352.08,77.86) .. (347.75,77.63) -- (321.81,76.25) .. controls (317.49,76.02) and (315.04,72.62) .. (316.34,68.64) -- (323.44,47.06) .. controls (324.75,43.09) and (329.32,40.05) .. (333.65,40.28) -- cycle ;
%Rounded Rect [id:dp41633103341228417] 
\draw  [fill={rgb, 255:red, 255; green, 255; blue, 255 }  ,fill opacity=1 ] (352.52,60.58) .. controls (356.85,60.81) and (359.3,64.21) .. (357.99,68.19) -- (350.89,89.77) .. controls (349.58,93.74) and (345.01,96.78) .. (340.69,96.55) -- (314.75,95.17) .. controls (310.42,94.94) and (307.97,91.53) .. (309.28,87.56) -- (316.38,65.98) .. controls (317.69,62) and (322.25,58.97) .. (326.58,59.2) -- cycle ;
%Rounded Rect [id:dp7210888085066917] 
\draw  [fill={rgb, 255:red, 255; green, 255; blue, 255 }  ,fill opacity=1 ] (347.42,75.71) .. controls (351.74,75.94) and (354.19,79.35) .. (352.89,83.32) -- (345.79,104.9) .. controls (344.48,108.87) and (339.91,111.91) .. (335.58,111.68) -- (309.64,110.3) .. controls (305.32,110.07) and (302.87,106.66) .. (304.17,102.69) -- (311.27,81.11) .. controls (312.58,77.14) and (317.15,74.1) .. (321.48,74.33) -- cycle ;
%Rounded Rect [id:dp9483720676140747] 
\draw  [fill={rgb, 255:red, 255; green, 255; blue, 255 }  ,fill opacity=1 ] (342.2,91.6) .. controls (346.52,91.83) and (348.97,95.24) .. (347.66,99.21) -- (340.56,120.79) .. controls (339.26,124.76) and (334.69,127.8) .. (330.36,127.57) -- (304.42,126.19) .. controls (300.1,125.96) and (297.65,122.55) .. (298.95,118.58) -- (306.05,97) .. controls (307.36,93.03) and (311.93,89.99) .. (316.26,90.22) -- cycle ;
%Rounded Rect [id:dp8664078580240527] 
\draw  [fill={rgb, 255:red, 255; green, 255; blue, 255 }  ,fill opacity=1 ] (337.09,106.73) .. controls (341.42,106.96) and (343.87,110.37) .. (342.56,114.34) -- (335.46,135.92) .. controls (334.15,139.9) and (329.59,142.93) .. (325.26,142.7) -- (299.32,141.32) .. controls (294.99,141.09) and (292.54,137.69) .. (293.85,133.71) -- (300.95,112.13) .. controls (302.26,108.16) and (306.83,105.12) .. (311.15,105.35) -- cycle ;
%Rounded Rect [id:dp43580550170987564] 
\draw  [fill={rgb, 255:red, 255; green, 255; blue, 255 }  ,fill opacity=1 ] (330.5,122.62) .. controls (334.83,122.85) and (337.27,126.26) .. (335.97,130.23) -- (328.87,151.81) .. controls (327.56,155.79) and (322.99,158.82) .. (318.66,158.59) -- (292.73,157.21) .. controls (288.4,156.98) and (285.95,153.58) .. (287.26,149.6) -- (294.36,128.02) .. controls (295.66,124.05) and (300.23,121.01) .. (304.56,121.24) -- cycle ;
%Rounded Rect [id:dp5584197815926703] 
\draw  [fill={rgb, 255:red, 255; green, 255; blue, 255 }  ,fill opacity=1 ] (323.43,141.54) .. controls (327.76,141.77) and (330.21,145.17) .. (328.9,149.15) -- (321.8,170.73) .. controls (320.49,174.7) and (315.93,177.74) .. (311.6,177.51) -- (285.66,176.13) .. controls (281.33,175.9) and (278.88,172.49) .. (280.19,168.52) -- (287.29,146.94) .. controls (288.6,142.96) and (293.17,139.93) .. (297.49,140.16) -- cycle ;
%Rounded Rect [id:dp8661838019763981] 
\draw  [fill={rgb, 255:red, 255; green, 255; blue, 255 }  ,fill opacity=1 ] (318.33,156.67) .. controls (322.66,156.9) and (325.11,160.31) .. (323.8,164.28) -- (316.7,185.86) .. controls (315.39,189.84) and (310.82,192.87) .. (306.5,192.64) -- (280.56,191.26) .. controls (276.23,191.03) and (273.78,187.63) .. (275.09,183.65) -- (282.19,162.07) .. controls (283.49,158.1) and (288.06,155.06) .. (292.39,155.29) -- cycle ;
%Rounded Rect [id:dp9508845146912172] 
\draw  [fill={rgb, 255:red, 255; green, 255; blue, 255 }  ,fill opacity=1 ] (313.11,172.56) .. controls (317.44,172.79) and (319.88,176.2) .. (318.58,180.17) -- (311.48,201.75) .. controls (310.17,205.73) and (305.6,208.76) .. (301.27,208.53) -- (275.34,207.15) .. controls (271.01,206.92) and (268.56,203.52) .. (269.87,199.54) -- (276.97,177.96) .. controls (278.27,173.99) and (282.84,170.95) .. (287.17,171.18) -- cycle ;
%Rounded Rect [id:dp32216690729736597] 
\draw  [fill={rgb, 255:red, 255; green, 255; blue, 255 }  ,fill opacity=1 ] (308,187.69) .. controls (312.33,187.92) and (314.78,191.33) .. (313.47,195.3) -- (306.37,216.89) .. controls (305.07,220.86) and (300.5,223.89) .. (296.17,223.66) -- (270.23,222.29) .. controls (265.9,222.06) and (263.46,218.65) .. (264.76,214.68) -- (271.86,193.09) .. controls (273.17,189.12) and (277.74,186.09) .. (282.07,186.32) -- cycle ;
%Rounded Rect [id:dp08943760761685893] 
\draw  [fill={rgb, 255:red, 255; green, 255; blue, 255 }  ,fill opacity=1 ] (301.06,205.85) .. controls (305.38,206.08) and (307.83,209.49) .. (306.53,213.46) -- (299.43,235.05) .. controls (298.12,239.02) and (293.55,242.05) .. (289.22,241.82) -- (263.28,240.45) .. controls (258.96,240.22) and (256.51,236.81) .. (257.82,232.84) -- (264.92,211.25) .. controls (266.22,207.28) and (270.79,204.25) .. (275.12,204.48) -- cycle ;
%Rounded Rect [id:dp805662074972429] 
\draw  [fill={rgb, 255:red, 255; green, 255; blue, 255 }  ,fill opacity=1 ] (295.95,220.99) .. controls (300.28,221.22) and (302.73,224.62) .. (301.42,228.6) -- (294.32,250.18) .. controls (293.02,254.15) and (288.45,257.19) .. (284.12,256.96) -- (258.18,255.58) .. controls (253.85,255.35) and (251.41,251.94) .. (252.71,247.97) -- (259.81,226.39) .. controls (261.12,222.41) and (265.69,219.38) .. (270.01,219.61) -- cycle ;
%Rounded Rect [id:dp6144131939434834] 
\draw  [fill={rgb, 255:red, 255; green, 255; blue, 255 }  ,fill opacity=1 ] (290.73,236.88) .. controls (295.06,237.11) and (297.51,240.51) .. (296.2,244.49) -- (289.1,266.07) .. controls (287.79,270.04) and (283.23,273.08) .. (278.9,272.85) -- (252.96,271.47) .. controls (248.63,271.24) and (246.18,267.83) .. (247.49,263.86) -- (254.59,242.28) .. controls (255.9,238.3) and (260.47,235.27) .. (264.79,235.5) -- cycle ;
%Rounded Rect [id:dp2586813936455584] 
\draw  [fill={rgb, 255:red, 255; green, 255; blue, 255 }  ,fill opacity=1 ] (417.62,44.4) .. controls (421.95,44.63) and (424.4,48.03) .. (423.09,52.01) -- (415.99,73.59) .. controls (414.68,77.56) and (410.12,80.6) .. (405.79,80.37) -- (379.85,78.99) .. controls (375.52,78.76) and (373.07,75.35) .. (374.38,71.38) -- (381.48,49.8) .. controls (382.79,45.82) and (387.35,42.79) .. (391.68,43.02) -- cycle ;
%Rounded Rect [id:dp39485740837645134] 
\draw  [fill={rgb, 255:red, 255; green, 255; blue, 255 }  ,fill opacity=1 ] (410.56,63.31) .. controls (414.88,63.54) and (417.33,66.95) .. (416.02,70.92) -- (408.92,92.5) .. controls (407.62,96.48) and (403.05,99.51) .. (398.72,99.28) -- (372.78,97.9) .. controls (368.46,97.68) and (366.01,94.27) .. (367.31,90.29) -- (374.41,68.71) .. controls (375.72,64.74) and (380.29,61.71) .. (384.62,61.93) -- cycle ;
%Rounded Rect [id:dp5643654743213579] 
\draw  [fill={rgb, 255:red, 255; green, 255; blue, 255 }  ,fill opacity=1 ] (405.45,78.45) .. controls (409.78,78.68) and (412.23,82.08) .. (410.92,86.06) -- (403.82,107.64) .. controls (402.51,111.61) and (397.95,114.65) .. (393.62,114.42) -- (367.68,113.04) .. controls (363.35,112.81) and (360.9,109.4) .. (362.21,105.43) -- (369.31,83.85) .. controls (370.62,79.87) and (375.19,76.84) .. (379.51,77.07) -- cycle ;
%Rounded Rect [id:dp031209723571941073] 
\draw  [fill={rgb, 255:red, 255; green, 255; blue, 255 }  ,fill opacity=1 ] (400.23,94.34) .. controls (404.56,94.56) and (407.01,97.97) .. (405.7,101.95) -- (398.6,123.53) .. controls (397.29,127.5) and (392.73,130.54) .. (388.4,130.31) -- (362.46,128.93) .. controls (358.13,128.7) and (355.68,125.29) .. (356.99,121.32) -- (364.09,99.74) .. controls (365.4,95.76) and (369.96,92.73) .. (374.29,92.96) -- cycle ;
%Rounded Rect [id:dp732750954296393] 
\draw  [fill={rgb, 255:red, 255; green, 255; blue, 255 }  ,fill opacity=1 ] (395.13,109.47) .. controls (399.46,109.7) and (401.9,113.11) .. (400.6,117.08) -- (393.5,138.66) .. controls (392.19,142.63) and (387.62,145.67) .. (383.29,145.44) -- (357.36,144.06) .. controls (353.03,143.83) and (350.58,140.42) .. (351.89,136.45) -- (358.99,114.87) .. controls (360.29,110.9) and (364.86,107.86) .. (369.19,108.09) -- cycle ;
%Rounded Rect [id:dp08454114971378346] 
\draw  [fill={rgb, 255:red, 255; green, 255; blue, 255 }  ,fill opacity=1 ] (388.53,125.36) .. controls (392.86,125.59) and (395.31,128.99) .. (394,132.97) -- (386.9,154.55) .. controls (385.6,158.52) and (381.03,161.56) .. (376.7,161.33) -- (350.76,159.95) .. controls (346.43,159.72) and (343.99,156.31) .. (345.29,152.34) -- (352.39,130.76) .. controls (353.7,126.79) and (358.27,123.75) .. (362.6,123.98) -- cycle ;
%Rounded Rect [id:dp9403590024297529] 
\draw  [fill={rgb, 255:red, 255; green, 255; blue, 255 }  ,fill opacity=1 ] (381.47,144.27) .. controls (385.8,144.5) and (388.24,147.91) .. (386.94,151.88) -- (379.84,173.47) .. controls (378.53,177.44) and (373.96,180.47) .. (369.64,180.24) -- (343.7,178.87) .. controls (339.37,178.64) and (336.92,175.23) .. (338.23,171.26) -- (345.33,149.67) .. controls (346.63,145.7) and (351.2,142.67) .. (355.53,142.9) -- cycle ;
%Rounded Rect [id:dp770965513589922] 
\draw  [fill={rgb, 255:red, 255; green, 255; blue, 255 }  ,fill opacity=1 ] (376.37,159.41) .. controls (380.69,159.64) and (383.14,163.04) .. (381.83,167.02) -- (374.73,188.6) .. controls (373.43,192.57) and (368.86,195.61) .. (364.53,195.38) -- (338.59,194) .. controls (334.27,193.77) and (331.82,190.36) .. (333.12,186.39) -- (340.22,164.81) .. controls (341.53,160.83) and (346.1,157.8) .. (350.43,158.03) -- cycle ;
%Rounded Rect [id:dp7870087247791839] 
\draw  [fill={rgb, 255:red, 255; green, 255; blue, 255 }  ,fill opacity=1 ] (371.14,175.3) .. controls (375.47,175.53) and (377.92,178.93) .. (376.61,182.91) -- (369.51,204.49) .. controls (368.21,208.46) and (363.64,211.5) .. (359.31,211.27) -- (333.37,209.89) .. controls (329.04,209.66) and (326.6,206.25) .. (327.9,202.28) -- (335,180.7) .. controls (336.31,176.72) and (340.88,173.69) .. (345.21,173.92) -- cycle ;
%Rounded Rect [id:dp6067583391886342] 
\draw  [fill={rgb, 255:red, 255; green, 255; blue, 255 }  ,fill opacity=1 ] (366.04,190.43) .. controls (370.37,190.66) and (372.82,194.07) .. (371.51,198.04) -- (364.41,219.62) .. controls (363.1,223.6) and (358.54,226.63) .. (354.21,226.4) -- (328.27,225.02) .. controls (323.94,224.79) and (321.49,221.39) .. (322.8,217.41) -- (329.9,195.83) .. controls (331.21,191.86) and (335.77,188.82) .. (340.1,189.05) -- cycle ;
%Rounded Rect [id:dp9836114697741176] 
\draw  [fill={rgb, 255:red, 255; green, 255; blue, 255 }  ,fill opacity=1 ] (359.09,208.59) .. controls (363.42,208.82) and (365.87,212.23) .. (364.56,216.2) -- (357.46,237.78) .. controls (356.16,241.75) and (351.59,244.79) .. (347.26,244.56) -- (321.32,243.18) .. controls (316.99,242.95) and (314.54,239.55) .. (315.85,235.57) -- (322.95,213.99) .. controls (324.26,210.02) and (328.83,206.98) .. (333.15,207.21) -- cycle ;
%Rounded Rect [id:dp4871634892255601] 
\draw  [fill={rgb, 255:red, 255; green, 255; blue, 255 }  ,fill opacity=1 ] (353.99,223.72) .. controls (358.32,223.95) and (360.77,227.36) .. (359.46,231.33) -- (352.36,252.91) .. controls (351.05,256.89) and (346.48,259.92) .. (342.16,259.69) -- (316.22,258.32) .. controls (311.89,258.09) and (309.44,254.68) .. (310.75,250.71) -- (317.85,229.12) .. controls (319.16,225.15) and (323.72,222.12) .. (328.05,222.35) -- cycle ;
%Rounded Rect [id:dp5672151170096975] 
\draw  [fill={rgb, 255:red, 255; green, 255; blue, 255 }  ,fill opacity=1 ] (348.77,239.61) .. controls (353.1,239.84) and (355.54,243.25) .. (354.24,247.22) -- (347.14,268.8) .. controls (345.83,272.78) and (341.26,275.81) .. (336.94,275.58) -- (311,274.2) .. controls (306.67,273.98) and (304.22,270.57) .. (305.53,266.59) -- (312.63,245.01) .. controls (313.93,241.04) and (318.5,238) .. (322.83,238.23) -- cycle ;

% Text Node
\draw (233.12,36.44) node [anchor=north west][inner sep=0.75pt]  [font=\footnotesize] [align=left] {A};
% Text Node
\draw (226.05,55.36) node [anchor=north west][inner sep=0.75pt]  [font=\footnotesize] [align=left] {2};
% Text Node
\draw (220.95,70.49) node [anchor=north west][inner sep=0.75pt]  [font=\footnotesize] [align=left] {3};
% Text Node
\draw (215.73,86.38) node [anchor=north west][inner sep=0.75pt]  [font=\footnotesize] [align=left] {4};
% Text Node
\draw (210.62,101.51) node [anchor=north west][inner sep=0.75pt]  [font=\footnotesize] [align=left] {5};
% Text Node
\draw (205.17,118.41) node [anchor=north west][inner sep=0.75pt]  [font=\footnotesize] [align=left] {6};
% Text Node
\draw (196.96,136.32) node [anchor=north west][inner sep=0.75pt]  [font=\footnotesize] [align=left] {7};
% Text Node
\draw (191.86,151.45) node [anchor=north west][inner sep=0.75pt]  [font=\footnotesize] [align=left] {8};
% Text Node
\draw (186.64,167.34) node [anchor=north west][inner sep=0.75pt]  [font=\footnotesize] [align=left] {9};
% Text Node
\draw (180.48,185.03) node [anchor=north west][inner sep=0.75pt]  [font=\footnotesize] [align=left] {10};
% Text Node
\draw (169.48,215.77) node [anchor=north west][inner sep=0.75pt]  [font=\footnotesize] [align=left] {Q};
% Text Node
\draw (164.26,231.66) node [anchor=north west][inner sep=0.75pt]  [font=\footnotesize] [align=left] {K};
% Text Node
\draw (176.08,199.37) node [anchor=north west][inner sep=0.75pt]  [font=\footnotesize] [align=left] {J};
% Text Node
\draw (291.9,41.31) node [anchor=north west][inner sep=0.75pt]  [font=\footnotesize] [align=left] {A};
% Text Node
\draw (284.84,60.23) node [anchor=north west][inner sep=0.75pt]  [font=\footnotesize] [align=left] {2};
% Text Node
\draw (279.73,75.36) node [anchor=north west][inner sep=0.75pt]  [font=\footnotesize] [align=left] {3};
% Text Node
\draw (274.51,91.25) node [anchor=north west][inner sep=0.75pt]  [font=\footnotesize] [align=left] {4};
% Text Node
\draw (269.41,106.39) node [anchor=north west][inner sep=0.75pt]  [font=\footnotesize] [align=left] {5};
% Text Node
\draw (263.95,123.29) node [anchor=north west][inner sep=0.75pt]  [font=\footnotesize] [align=left] {6};
% Text Node
\draw (255.75,141.19) node [anchor=north west][inner sep=0.75pt]  [font=\footnotesize] [align=left] {7};
% Text Node
\draw (250.65,156.33) node [anchor=north west][inner sep=0.75pt]  [font=\footnotesize] [align=left] {8};
% Text Node
\draw (245.42,172.22) node [anchor=north west][inner sep=0.75pt]  [font=\footnotesize] [align=left] {9};
% Text Node
\draw (239.27,189.91) node [anchor=north west][inner sep=0.75pt]  [font=\footnotesize] [align=left] {10};
% Text Node
\draw (228.27,220.64) node [anchor=north west][inner sep=0.75pt]  [font=\footnotesize] [align=left] {Q};
% Text Node
\draw (223.05,236.53) node [anchor=north west][inner sep=0.75pt]  [font=\footnotesize] [align=left] {K};
% Text Node
\draw (234.87,204.25) node [anchor=north west][inner sep=0.75pt]  [font=\footnotesize] [align=left] {J};
% Text Node
\draw (348.85,44.05) node [anchor=north west][inner sep=0.75pt]  [font=\footnotesize] [align=left] {A};
% Text Node
\draw (341.79,62.97) node [anchor=north west][inner sep=0.75pt]  [font=\footnotesize] [align=left] {2};
% Text Node
\draw (336.69,78.1) node [anchor=north west][inner sep=0.75pt]  [font=\footnotesize] [align=left] {3};
% Text Node
\draw (331.47,93.99) node [anchor=north west][inner sep=0.75pt]  [font=\footnotesize] [align=left] {4};
% Text Node
\draw (326.36,109.12) node [anchor=north west][inner sep=0.75pt]  [font=\footnotesize] [align=left] {5};
% Text Node
\draw (320.91,126.02) node [anchor=north west][inner sep=0.75pt]  [font=\footnotesize] [align=left] {6};
% Text Node
\draw (312.7,143.93) node [anchor=north west][inner sep=0.75pt]  [font=\footnotesize] [align=left] {7};
% Text Node
\draw (307.6,159.06) node [anchor=north west][inner sep=0.75pt]  [font=\footnotesize] [align=left] {8};
% Text Node
\draw (302.38,174.95) node [anchor=north west][inner sep=0.75pt]  [font=\footnotesize] [align=left] {9};
% Text Node
\draw (296.22,192.64) node [anchor=north west][inner sep=0.75pt]  [font=\footnotesize] [align=left] {10};
% Text Node
\draw (285.22,223.38) node [anchor=north west][inner sep=0.75pt]  [font=\footnotesize] [align=left] {Q};
% Text Node
\draw (280,239.27) node [anchor=north west][inner sep=0.75pt]  [font=\footnotesize] [align=left] {K};
% Text Node
\draw (291.82,206.98) node [anchor=north west][inner sep=0.75pt]  [font=\footnotesize] [align=left] {J};
% Text Node
\draw (406.89,46.79) node [anchor=north west][inner sep=0.75pt]  [font=\footnotesize] [align=left] {A};
% Text Node
\draw (399.83,65.7) node [anchor=north west][inner sep=0.75pt]  [font=\footnotesize] [align=left] {2};
% Text Node
\draw (394.72,80.84) node [anchor=north west][inner sep=0.75pt]  [font=\footnotesize] [align=left] {3};
% Text Node
\draw (389.5,96.73) node [anchor=north west][inner sep=0.75pt]  [font=\footnotesize] [align=left] {4};
% Text Node
\draw (384.4,111.86) node [anchor=north west][inner sep=0.75pt]  [font=\footnotesize] [align=left] {5};
% Text Node
\draw (378.94,128.76) node [anchor=north west][inner sep=0.75pt]  [font=\footnotesize] [align=left] {6};
% Text Node
\draw (370.74,146.67) node [anchor=north west][inner sep=0.75pt]  [font=\footnotesize] [align=left] {7};
% Text Node
\draw (365.64,161.8) node [anchor=north west][inner sep=0.75pt]  [font=\footnotesize] [align=left] {8};
% Text Node
\draw (360.41,177.69) node [anchor=north west][inner sep=0.75pt]  [font=\footnotesize] [align=left] {9};
% Text Node
\draw (350.48,191.82) node [anchor=north west][inner sep=0.75pt]  [font=\footnotesize] [align=left] {10};
% Text Node
\draw (343.26,226.11) node [anchor=north west][inner sep=0.75pt]  [font=\footnotesize] [align=left] {Q};
% Text Node
\draw (338.04,242) node [anchor=north west][inner sep=0.75pt]  [font=\footnotesize] [align=left] {K};
% Text Node
\draw (349.86,209.72) node [anchor=north west][inner sep=0.75pt]  [font=\footnotesize] [align=left] {J};
% Text Node
\draw (230.41,14.79) node [anchor=north west][inner sep=0.75pt]  [xslant=0.42] [align=left] {\emoji{heart-suit}};
% Text Node
\draw (347.16,19.79) node [anchor=north west][inner sep=0.75pt]  [xslant=0.51] [align=left] {\emoji{spade-suit}};
% Text Node
\draw (289.91,17.79) node [anchor=north west][inner sep=0.75pt]  [xslant=0.6] [align=left] {\emoji{diamond-suit}};
% Text Node
\draw (403.41,22.79) node [anchor=north west][inner sep=0.75pt]  [xslant=0.42] [align=left] {\emoji{club-suit}};


\end{tikzpicture}
  \end{center}
\begin{enumerate}
    \item ¿De cuántas maneras diferentes podemos sacar tres espadas y dos cartas rojas (diamantes y/o corazones)?

  \begin{solution}
    Notamos que tenemos un problema de combinatoria en dos etapas. Primero, obtenemos las maneras que se pueden sacar 3 espadas de 13 cartas de espadas que tenemos:
    $$C(13,3)=286 \text{ maneras.}$$
    Ahora bien, calculamos las maneras que se pueden obtener 2 cartas rojas, de las 26 cartas rojas que tenemos, es decir: 
    $$C(26,2)=325\text{ maneras.}$$
    Ahora aplicamos el principio del producto: 
    $$C(13,3)\cdot C(26,2)= 286\cdot 325 = 92950 \text{ maneras de sacar 3 espadas y 2 cartas rojas.}$$
  \end{solution} 
    \item ¿De cuántas maneras diferentes podemos sacar un flush (cinco cartas del mismo palo, sin necesariamente ser consecutivas)?
      \begin{solution}
      Ya que el orden no importa, es un problema de combinatoria, es decir: 
      $$C(13,5)= 1287 \text{ maneras diferentes de sacar un flush para cada palo.}$$
      Si en caso nos piden la solución para los 4 palos, entonces simplemente sería: 
      $$4\cdot C(13,5)= 4\cdot 1287 = 5148 \text{ maneras diferentes de sacar un flush.}$$
  \end{solution}
\end{enumerate}
\section{Problema 4}
Supongamos que hay 4 líneas de buses entre A y B; y 3 líneas de buses entre B y C. ¿De cuántas maneras puede una persona viajar en viaje redondo (ida y vuelta) de A a C, sin usar ninguna línea de bus más de una vez?

\begin{solution}
Visualicemos el problema: 
\begin{center}
    

\tikzset{every picture/.style={line width=0.75pt}} %set default line width to 0.75pt        

\begin{tikzpicture}[x=0.75pt,y=0.75pt,yscale=-1,xscale=1]
%uncomment if require: \path (0,300); %set diagram left start at 0, and has height of 300

%Shape: Cube [id:dp19630639075995027] 
\draw  [fill={rgb, 255:red, 255; green, 1; blue, 1 }  ,fill opacity=0.22 ] (245.66,99.96) -- (264.2,89.11) -- (394.18,146.03) -- (388.05,187.61) -- (369.51,198.46) -- (239.53,141.54) -- cycle ; \draw   (394.18,146.03) -- (375.64,156.88) -- (245.66,99.96) ; \draw   (375.64,156.88) -- (369.51,198.46) ;
%Shape: Cube [id:dp5121656042070563] 
\draw  [fill={rgb, 255:red, 74; green, 144; blue, 226 }  ,fill opacity=0.21 ] (370.66,12.96) -- (389.2,2.11) -- (519.18,59.03) -- (513.05,100.61) -- (494.51,111.46) -- (364.53,54.54) -- cycle ; \draw   (519.18,59.03) -- (500.64,69.88) -- (370.66,12.96) ; \draw   (500.64,69.88) -- (494.51,111.46) ;
%Shape: Cube [id:dp9223007421491981] 
\draw  [fill={rgb, 255:red, 236; green, 150; blue, 9 }  ,fill opacity=0.31 ] (108.73,183.45) -- (127.5,172.46) -- (264.43,232.43) -- (258.23,274.52) -- (239.46,285.5) -- (102.53,225.54) -- cycle ; \draw   (264.43,232.43) -- (245.66,243.41) -- (108.73,183.45) ; \draw   (245.66,243.41) -- (239.46,285.5) ;
%Curve Lines [id:da9236988327654716] 
\draw [color={rgb, 255:red, 208; green, 2; blue, 27 }  ,draw opacity=1 ]   (169.86,186.83) .. controls (183.9,151.93) and (197.75,135.69) .. (268.36,140.36) ;
\draw [shift={(269.43,140.43)}, rotate = 183.92] [color={rgb, 255:red, 208; green, 2; blue, 27 }  ,draw opacity=1 ][line width=0.75]    (10.93,-3.29) .. controls (6.95,-1.4) and (3.31,-0.3) .. (0,0) .. controls (3.31,0.3) and (6.95,1.4) .. (10.93,3.29)   ;
\draw [shift={(169,189)}, rotate = 291.53] [color={rgb, 255:red, 208; green, 2; blue, 27 }  ,draw opacity=1 ][line width=0.75]    (10.93,-3.29) .. controls (6.95,-1.4) and (3.31,-0.3) .. (0,0) .. controls (3.31,0.3) and (6.95,1.4) .. (10.93,3.29)   ;
%Curve Lines [id:da8519542966027018] 
\draw [color={rgb, 255:red, 255; green, 3; blue, 3 }  ,draw opacity=1 ]   (225.17,212.49) .. controls (271.7,180.74) and (303.54,206.06) .. (309.19,170.11) ;
\draw [shift={(309.43,168.43)}, rotate = 457.31] [color={rgb, 255:red, 255; green, 3; blue, 3 }  ,draw opacity=1 ][line width=0.75]    (10.93,-3.29) .. controls (6.95,-1.4) and (3.31,-0.3) .. (0,0) .. controls (3.31,0.3) and (6.95,1.4) .. (10.93,3.29)   ;
\draw [shift={(223,214)}, rotate = 324.48] [color={rgb, 255:red, 255; green, 3; blue, 3 }  ,draw opacity=1 ][line width=0.75]    (10.93,-3.29) .. controls (6.95,-1.4) and (3.31,-0.3) .. (0,0) .. controls (3.31,0.3) and (6.95,1.4) .. (10.93,3.29)   ;
%Curve Lines [id:da8967339679578482] 
\draw [color={rgb, 255:red, 208; green, 2; blue, 27 }  ,draw opacity=1 ]   (249.34,223.98) .. controls (299.43,223.63) and (316.32,229.44) .. (330.98,174.12) ;
\draw [shift={(331.43,172.43)}, rotate = 464.5] [color={rgb, 255:red, 208; green, 2; blue, 27 }  ,draw opacity=1 ][line width=0.75]    (10.93,-3.29) .. controls (6.95,-1.4) and (3.31,-0.3) .. (0,0) .. controls (3.31,0.3) and (6.95,1.4) .. (10.93,3.29)   ;
\draw [shift={(247,224)}, rotate = 359.38] [color={rgb, 255:red, 208; green, 2; blue, 27 }  ,draw opacity=1 ][line width=0.75]    (10.93,-3.29) .. controls (6.95,-1.4) and (3.31,-0.3) .. (0,0) .. controls (3.31,0.3) and (6.95,1.4) .. (10.93,3.29)   ;
%Curve Lines [id:da8526320801010947] 
\draw [color={rgb, 255:red, 74; green, 144; blue, 226 }  ,draw opacity=1 ]   (304.86,102.83) .. controls (318.9,67.93) and (332.75,51.69) .. (403.36,56.36) ;
\draw [shift={(404.43,56.43)}, rotate = 183.92] [color={rgb, 255:red, 74; green, 144; blue, 226 }  ,draw opacity=1 ][line width=0.75]    (10.93,-3.29) .. controls (6.95,-1.4) and (3.31,-0.3) .. (0,0) .. controls (3.31,0.3) and (6.95,1.4) .. (10.93,3.29)   ;
\draw [shift={(304,105)}, rotate = 291.53] [color={rgb, 255:red, 74; green, 144; blue, 226 }  ,draw opacity=1 ][line width=0.75]    (10.93,-3.29) .. controls (6.95,-1.4) and (3.31,-0.3) .. (0,0) .. controls (3.31,0.3) and (6.95,1.4) .. (10.93,3.29)   ;
%Curve Lines [id:da5716803692726303] 
\draw [color={rgb, 255:red, 74; green, 144; blue, 226 }  ,draw opacity=1 ]   (365.49,132.85) .. controls (431.53,114.17) and (417.17,115.1) .. (444.17,86.74) ;
\draw [shift={(445.43,85.43)}, rotate = 494.03] [color={rgb, 255:red, 74; green, 144; blue, 226 }  ,draw opacity=1 ][line width=0.75]    (10.93,-3.29) .. controls (6.95,-1.4) and (3.31,-0.3) .. (0,0) .. controls (3.31,0.3) and (6.95,1.4) .. (10.93,3.29)   ;
\draw [shift={(363.43,133.43)}, rotate = 344.26] [color={rgb, 255:red, 74; green, 144; blue, 226 }  ,draw opacity=1 ][line width=0.75]    (10.93,-3.29) .. controls (6.95,-1.4) and (3.31,-0.3) .. (0,0) .. controls (3.31,0.3) and (6.95,1.4) .. (10.93,3.29)   ;
%Curve Lines [id:da161419448202348] 
\draw [color={rgb, 255:red, 255; green, 3; blue, 3 }  ,draw opacity=1 ]   (199.75,198.87) .. controls (227.71,165.42) and (209.78,159.43) .. (285.28,155.49) ;
\draw [shift={(286.43,155.43)}, rotate = 537.06] [color={rgb, 255:red, 255; green, 3; blue, 3 }  ,draw opacity=1 ][line width=0.75]    (10.93,-3.29) .. controls (6.95,-1.4) and (3.31,-0.3) .. (0,0) .. controls (3.31,0.3) and (6.95,1.4) .. (10.93,3.29)   ;
\draw [shift={(198.43,200.43)}, rotate = 310.6] [color={rgb, 255:red, 255; green, 3; blue, 3 }  ,draw opacity=1 ][line width=0.75]    (10.93,-3.29) .. controls (6.95,-1.4) and (3.31,-0.3) .. (0,0) .. controls (3.31,0.3) and (6.95,1.4) .. (10.93,3.29)   ;
%Curve Lines [id:da5871528552964772] 
\draw [color={rgb, 255:red, 74; green, 144; blue, 226 }  ,draw opacity=1 ]   (338.02,118.72) .. controls (363.27,91.69) and (366.45,91.02) .. (417.86,71.04) ;
\draw [shift={(419.43,70.43)}, rotate = 518.75] [color={rgb, 255:red, 74; green, 144; blue, 226 }  ,draw opacity=1 ][line width=0.75]    (10.93,-3.29) .. controls (6.95,-1.4) and (3.31,-0.3) .. (0,0) .. controls (3.31,0.3) and (6.95,1.4) .. (10.93,3.29)   ;
\draw [shift={(336.43,120.43)}, rotate = 312.95] [color={rgb, 255:red, 74; green, 144; blue, 226 }  ,draw opacity=1 ][line width=0.75]    (10.93,-3.29) .. controls (6.95,-1.4) and (3.31,-0.3) .. (0,0) .. controls (3.31,0.3) and (6.95,1.4) .. (10.93,3.29)   ;

% Text Node
\draw (161,224.21) node [anchor=north west][inner sep=0.75pt]    {$A$};
% Text Node
\draw (301,141.21) node [anchor=north west][inner sep=0.75pt]    {$B$};
% Text Node
\draw (423,53.21) node [anchor=north west][inner sep=0.75pt]    {$C$};
% Text Node
\draw (173.6,138.87) node [anchor=north west][inner sep=0.75pt]  [color={rgb, 255:red, 244; green, 0; blue, 0 }  ,opacity=1 ,rotate=-322.11]  {$a_{1}$};
% Text Node
\draw (198.6,158.87) node [anchor=north west][inner sep=0.75pt]  [color={rgb, 255:red, 244; green, 0; blue, 0 }  ,opacity=1 ,rotate=-322.11]  {$a_{2}$};
% Text Node
\draw (248.64,177.47) node [anchor=north west][inner sep=0.75pt]  [color={rgb, 255:red, 244; green, 0; blue, 0 }  ,opacity=1 ,rotate=-339.89]  {$a_{3}$};
% Text Node
\draw (283.6,206.87) node [anchor=north west][inner sep=0.75pt]  [color={rgb, 255:red, 244; green, 0; blue, 0 }  ,opacity=1 ,rotate=-322.11]  {$a_{4}$};
% Text Node
\draw (308.6,51.87) node [anchor=north west][inner sep=0.75pt]  [color={rgb, 255:red, 74; green, 144; blue, 226 }  ,opacity=1 ,rotate=-322.11]  {$b_{1}$};
% Text Node
\draw (343.6,79.87) node [anchor=north west][inner sep=0.75pt]  [color={rgb, 255:red, 74; green, 144; blue, 226 }  ,opacity=1 ,rotate=-322.11]  {$b_{2}$};
% Text Node
\draw (386.6,106.87) node [anchor=north west][inner sep=0.75pt]  [color={rgb, 255:red, 74; green, 144; blue, 226 }  ,opacity=1 ,rotate=-322.11]  {$b_{3}$};


\end{tikzpicture}
\end{center}

Es decir, tenemos un problema de combinatoria. Para el tramo $A-B$: 
$$C(4,1)= 4 \text{ formas.}$$
Para el tramo $B-C$: 
$$C(3,1)= 3 \text{ formas.}$$
Ahora de regreso, tramo $C-B$: 
$$C(2,1)= 2 \text{ formas.}$$
Para el tramo $B-A$:
$$C(3,1)= 3 \text{ formas.}$$

Por lo tanto, las maneras que se puede hacer un viaje redondo son (aplicando el principio del producto):
$$C(4,1)\cdot C(3,1)\cdot C(2,1)\cdot  C(3,1)=4\cdot 3 \cdot 2\cdot 3 = 72 \text{ formas.}$$
\end{solution}
\section{Problema 5}

¿Cuántos números entre 300 y 900 son múltiplos de 2 o 5, pero no de 6?
\begin{solution}
El problema lo podemos visualizar como: 
\begin{center}
    

\tikzset{every picture/.style={line width=0.75pt}} %set default line width to 0.75pt        

\begin{tikzpicture}[x=0.75pt,y=0.75pt,yscale=-1,xscale=1]
%uncomment if require: \path (0,300); %set diagram left start at 0, and has height of 300

%Shape: Rectangle [id:dp7203045602613047] 
\draw  [fill={rgb, 255:red, 0; green, 0; blue, 0 }  ,fill opacity=0.02 ][dash pattern={on 0.84pt off 2.51pt}] (125.43,17) -- (503.43,17) -- (503.43,290) -- (125.43,290) -- cycle ;
%Shape: Ellipse [id:dp05198513027857288] 
\draw  [color={rgb, 255:red, 65; green, 117; blue, 5 }  ,draw opacity=1 ][fill={rgb, 255:red, 126; green, 211; blue, 33 }  ,fill opacity=0.62 ][dash pattern={on 4.5pt off 4.5pt}] (231.55,119.17) .. controls (231.55,79.73) and (264.4,47.76) .. (304.93,47.76) .. controls (345.45,47.76) and (378.31,79.73) .. (378.31,119.17) .. controls (378.31,158.61) and (345.45,190.58) .. (304.93,190.58) .. controls (264.4,190.58) and (231.55,158.61) .. (231.55,119.17) -- cycle ;
%Shape: Ellipse [id:dp12947926731616155] 
\draw  [color={rgb, 255:red, 255; green, 0; blue, 31 }  ,draw opacity=1 ][fill={rgb, 255:red, 208; green, 2; blue, 27 }  ,fill opacity=0 ][dash pattern={on 4.5pt off 4.5pt}] (243.97,206.69) .. controls (243.97,166.34) and (277.58,133.63) .. (319.04,133.63) .. controls (360.5,133.63) and (394.11,166.34) .. (394.11,206.69) .. controls (394.11,247.04) and (360.5,279.75) .. (319.04,279.75) .. controls (277.58,279.75) and (243.97,247.04) .. (243.97,206.69) -- cycle ;
%Shape: Ellipse [id:dp04922743105309446] 
\draw  [color={rgb, 255:red, 144; green, 19; blue, 254 }  ,draw opacity=1 ][fill={rgb, 255:red, 189; green, 16; blue, 224 }  ,fill opacity=0.37 ][dash pattern={on 4.5pt off 4.5pt}] (321.11,152.49) .. controls (321.11,113.06) and (353.96,81.08) .. (394.49,81.08) .. controls (435.01,81.08) and (467.87,113.06) .. (467.87,152.49) .. controls (467.87,191.93) and (435.01,223.9) .. (394.49,223.9) .. controls (353.96,223.9) and (321.11,191.93) .. (321.11,152.49) -- cycle ;

% Text Node
\draw  [fill={rgb, 255:red, 248; green, 231; blue, 28 }  ,fill opacity=0.24 ]  (160.62,29.35) -- (236.62,29.35) -- (236.62,53.35) -- (160.62,53.35) -- cycle  ;
\draw (163.62,33.57) node [anchor=north west][inner sep=0.75pt]    {$300-900$};
% Text Node
\draw (274.08,88.68) node [anchor=north west][inner sep=0.75pt]    {$A=2$};
% Text Node
\draw (402.84,141.23) node [anchor=north west][inner sep=0.75pt]    {$B=5$};
% Text Node
\draw (294.25,220.69) node [anchor=north west][inner sep=0.75pt]    {$C=6$};
% Text Node
\draw (347,113.21) node [anchor=north west][inner sep=0.75pt]    {$10$};
% Text Node
\draw (284,154.21) node [anchor=north west][inner sep=0.75pt]    {$12$};
% Text Node
\draw (362,184.21) node [anchor=north west][inner sep=0.75pt]    {$30$};
% Text Node
\draw (332,151.21) node [anchor=north west][inner sep=0.75pt]    {$60$};


\end{tikzpicture}
\end{center}
Es decir, que nos interesa conocer los números del círculo verde y purpura; evitando \textbf{todos} los elementos del círculo punteado rojo. Ya que tenemos 3 conjuntos, vamos a usar la generalización del principio de inclusión-exclusión centrado para 3 conjuntos. Entonces,

$$\left|\bigcup_{i=1}^n A_i\right| = \sum_{i=1}^n |A_i| - \sum_{1 \leqslant i < j \leqslant n} |A_i\cap A_j| + \sum_{1 \leqslant i < j < k \leqslant n} |A_i \cap A_j\cap A_k| - \cdots + (-1)^{n-1} \left|A_1\cap\cdots\cap A_n\right|.$$
En el caso específico de n=3 (3 conjuntos), 

$$|A \cup B \cup C| = |A| + |B| + |C| - |A \cap B| - |A \cap C| - |B \cap C| + |A \cap B \cap C|.$$


\linita 

Elementos del conjunto $A$: 

$$A=\floor*{\frac{900}{2}}-\floor*{\frac{300}{2}}=300$$

Elementos del conjunto $B$: 

$$B=\floor*{\frac{900}{5}}-\floor*{\frac{300}{5}}=120$$

Elementos del conjunto $C$: 

$$C=\floor*{\frac{900}{6}}-\floor*{\frac{300}{6}}=100$$

\linita 

Elementos de $|A\cap B|$: 

$$A\cap B=\floor*{\frac{900}{2*5}}-\floor*{\frac{300}{2*5}}=60$$

Elementos de $|A\cap C|$: 
$$A\cap C=\floor*{\frac{900}{2*6}}-\floor*{\frac{300}{2*6}}=50$$
\newpage
Elementos de $|B\cap C|$: 

$$B\cap C=\floor*{\frac{900}{5*6}}-\floor*{\frac{300}{5*6}}=20$$

Elementos de $|A\cap B\cap C|$: 
$$A\cap B\cap C=\floor*{\frac{900}{2*5*6}}-\floor*{\frac{300}{2*5*6}}=10$$

\linita

Entonces, por el principio de inclusión-exclusión (excluyendo el conjunto $C$ porque no nos interesa), 
$$|A\cup B\cup C|= 300+120+0-60-50-20+10=300 \text{ números.}$$



\end{solution}
\section{Problema 6}

Supongamos que una tienda ofrece seis sabores de donas: chocolate, glaseada, crema bavaria, café, cajeta y fresa.

\begin{enumerate}
    \item ¿De cuántas maneras diferentes podemos escoger ocho de ellas, si hay por lo menos ocho donas de cada sabor?
    \begin{center}
        

\tikzset{every picture/.style={line width=0.75pt}} %set default line width to 0.75pt        

\begin{tikzpicture}[x=0.75pt,y=0.75pt,yscale=-1,xscale=1]
%uncomment if require: \path (0,300); %set diagram left start at 0, and has height of 300

%Shape: Parallelogram [id:dp8936185554499697] 
\draw   (243.63,72.43) -- (499.43,72.43) -- (389.8,233.43) -- (134,233.43) -- cycle ;
%Straight Lines [id:da33224811057421144] 
\draw    (389.8,233.43) -- (389.8,271.43) ;
%Straight Lines [id:da6768937678830187] 
\draw    (499.43,110.43) -- (389.8,271.43) ;
%Straight Lines [id:da9562599472933878] 
\draw    (134,233.43) -- (134,271.43) ;
%Straight Lines [id:da7467639533039396] 
\draw    (389.8,271.43) -- (134,271.43) ;

%Straight Lines [id:da4356800696440639] 
\draw    (499.43,72.43) -- (499.43,110.43) ;
%Straight Lines [id:da9727141871179429] 
\draw    (134,233.43) -- (389.8,233.43) (177,229.43) -- (177,237.43)(220,229.43) -- (220,237.43)(263,229.43) -- (263,237.43)(306,229.43) -- (306,237.43)(349,229.43) -- (349,237.43) ;
%Straight Lines [id:da6970568553893579] 
\draw    (243.63,72.43) -- (499.43,72.43) (286.63,68.43) -- (286.63,76.43)(329.63,68.43) -- (329.63,76.43)(372.63,68.43) -- (372.63,76.43)(415.63,68.43) -- (415.63,76.43)(458.63,68.43) -- (458.63,76.43) ;
%Straight Lines [id:da12917242218329472] 
\draw [color={rgb, 255:red, 255; green, 0; blue, 0 }  ,draw opacity=1 ]   (177,233) -- (286.43,72.43) ;
%Straight Lines [id:da5826111049341247] 
\draw [color={rgb, 255:red, 255; green, 0; blue, 0 }  ,draw opacity=1 ]   (221,232) -- (330.43,71.43) ;
%Straight Lines [id:da9643596433649732] 
\draw [color={rgb, 255:red, 255; green, 0; blue, 0 }  ,draw opacity=1 ]   (261.9,233.43) -- (371.33,72.86) ;
%Straight Lines [id:da8578053102302399] 
\draw [color={rgb, 255:red, 255; green, 0; blue, 0 }  ,draw opacity=1 ]   (306,233) -- (415.43,72.43) ;
%Straight Lines [id:da6124498886544114] 
\draw [color={rgb, 255:red, 255; green, 0; blue, 0 }  ,draw opacity=1 ]   (349,232) -- (458.43,71.43) ;
%Shape: Donut [id:dp5987741154052165] 
\draw  [fill={rgb, 255:red, 139; green, 87; blue, 42 }  ,fill opacity=0.72 ,even odd rule] (220.7,123.94) .. controls (221.71,122.06) and (227.45,120.54) .. (233.52,120.54) .. controls (239.59,120.54) and (243.69,122.06) .. (242.68,123.94) .. controls (241.67,125.82) and (235.94,127.35) .. (229.87,127.35) .. controls (223.8,127.35) and (219.69,125.82) .. (220.7,123.94)(215.59,123.94) .. controls (218.11,119.24) and (227.36,115.43) .. (236.25,115.43) .. controls (245.14,115.43) and (250.31,119.24) .. (247.79,123.94) .. controls (245.27,128.64) and (236.02,132.46) .. (227.13,132.46) .. controls (218.24,132.46) and (213.07,128.64) .. (215.59,123.94) ;
%Shape: Donut [id:dp9908124462937663] 
\draw  [fill={rgb, 255:red, 139; green, 87; blue, 42 }  ,fill opacity=0.72 ,even odd rule] (206.79,142.22) .. controls (207.8,140.34) and (213.54,138.81) .. (219.61,138.81) .. controls (225.68,138.81) and (229.78,140.34) .. (228.77,142.22) .. controls (227.77,144.1) and (222.03,145.62) .. (215.96,145.62) .. controls (209.89,145.62) and (205.79,144.1) .. (206.79,142.22)(201.68,142.22) .. controls (204.2,137.52) and (213.45,133.71) .. (222.34,133.71) .. controls (231.24,133.71) and (236.4,137.52) .. (233.88,142.22) .. controls (231.36,146.92) and (222.11,150.73) .. (213.22,150.73) .. controls (204.33,150.73) and (199.17,146.92) .. (201.68,142.22) ;
%Shape: Donut [id:dp9781775094593524] 
\draw  [fill={rgb, 255:red, 139; green, 87; blue, 42 }  ,fill opacity=0.72 ,even odd rule] (193.7,161.29) .. controls (194.71,159.41) and (200.45,157.88) .. (206.52,157.88) .. controls (212.59,157.88) and (216.69,159.41) .. (215.68,161.29) .. controls (214.67,163.17) and (208.94,164.7) .. (202.87,164.7) .. controls (196.8,164.7) and (192.69,163.17) .. (193.7,161.29)(188.59,161.29) .. controls (191.11,156.59) and (200.36,152.78) .. (209.25,152.78) .. controls (218.14,152.78) and (223.31,156.59) .. (220.79,161.29) .. controls (218.27,165.99) and (209.02,169.8) .. (200.13,169.8) .. controls (191.24,169.8) and (186.07,165.99) .. (188.59,161.29) ;
%Shape: Donut [id:dp7257619832282152] 
\draw  [fill={rgb, 255:red, 139; green, 87; blue, 42 }  ,fill opacity=0.72 ,even odd rule] (180.61,179.57) .. controls (181.62,177.69) and (187.36,176.16) .. (193.43,176.16) .. controls (199.5,176.16) and (203.6,177.69) .. (202.59,179.57) .. controls (201.58,181.45) and (195.85,182.97) .. (189.78,182.97) .. controls (183.71,182.97) and (179.6,181.45) .. (180.61,179.57)(175.5,179.57) .. controls (178.02,174.86) and (187.27,171.05) .. (196.16,171.05) .. controls (205.05,171.05) and (210.22,174.86) .. (207.7,179.57) .. controls (205.18,184.27) and (195.93,188.08) .. (187.04,188.08) .. controls (178.15,188.08) and (172.98,184.27) .. (175.5,179.57) ;
%Shape: Donut [id:dp4608811373504348] 
\draw  [fill={rgb, 255:red, 139; green, 87; blue, 42 }  ,fill opacity=0.72 ,even odd rule] (166.7,197.84) .. controls (167.71,195.96) and (173.45,194.44) .. (179.52,194.44) .. controls (185.59,194.44) and (189.69,195.96) .. (188.68,197.84) .. controls (187.68,199.72) and (181.94,201.25) .. (175.87,201.25) .. controls (169.8,201.25) and (165.69,199.72) .. (166.7,197.84)(161.59,197.84) .. controls (164.11,193.14) and (173.36,189.33) .. (182.25,189.33) .. controls (191.14,189.33) and (196.31,193.14) .. (193.79,197.84) .. controls (191.27,202.55) and (182.02,206.36) .. (173.13,206.36) .. controls (164.24,206.36) and (159.07,202.55) .. (161.59,197.84) ;
%Shape: Donut [id:dp8271653555691039] 
\draw  [fill={rgb, 255:red, 139; green, 87; blue, 42 }  ,fill opacity=0.72 ,even odd rule] (153.61,216.91) .. controls (154.62,215.03) and (160.36,213.51) .. (166.43,213.51) .. controls (172.5,213.51) and (176.6,215.03) .. (175.59,216.91) .. controls (174.58,218.8) and (168.85,220.32) .. (162.78,220.32) .. controls (156.71,220.32) and (152.6,218.8) .. (153.61,216.91)(148.5,216.91) .. controls (151.02,212.21) and (160.27,208.4) .. (169.16,208.4) .. controls (178.05,208.4) and (183.22,212.21) .. (180.7,216.91) .. controls (178.18,221.62) and (168.93,225.43) .. (160.04,225.43) .. controls (151.15,225.43) and (145.98,221.62) .. (148.5,216.91) ;
%Shape: Donut [id:dp07666481046059437] 
\draw  [fill={rgb, 255:red, 139; green, 87; blue, 42 }  ,fill opacity=0.72 ,even odd rule] (246.7,86.94) .. controls (247.71,85.06) and (253.45,83.54) .. (259.52,83.54) .. controls (265.59,83.54) and (269.69,85.06) .. (268.68,86.94) .. controls (267.67,88.82) and (261.94,90.35) .. (255.87,90.35) .. controls (249.8,90.35) and (245.69,88.82) .. (246.7,86.94)(241.59,86.94) .. controls (244.11,82.24) and (253.36,78.43) .. (262.25,78.43) .. controls (271.14,78.43) and (276.31,82.24) .. (273.79,86.94) .. controls (271.27,91.64) and (262.02,95.46) .. (253.13,95.46) .. controls (244.24,95.46) and (239.07,91.64) .. (241.59,86.94) ;
%Shape: Donut [id:dp4406455531178006] 
\draw  [fill={rgb, 255:red, 139; green, 87; blue, 42 }  ,fill opacity=0.72 ,even odd rule] (232.79,105.22) .. controls (233.8,103.34) and (239.54,101.81) .. (245.61,101.81) .. controls (251.68,101.81) and (255.78,103.34) .. (254.77,105.22) .. controls (253.77,107.1) and (248.03,108.62) .. (241.96,108.62) .. controls (235.89,108.62) and (231.79,107.1) .. (232.79,105.22)(227.68,105.22) .. controls (230.2,100.52) and (239.45,96.71) .. (248.34,96.71) .. controls (257.24,96.71) and (262.4,100.52) .. (259.88,105.22) .. controls (257.36,109.92) and (248.11,113.73) .. (239.22,113.73) .. controls (230.33,113.73) and (225.17,109.92) .. (227.68,105.22) ;

%Shape: Donut [id:dp08489965041095582] 
\draw  [fill={rgb, 255:red, 248; green, 231; blue, 28 }  ,fill opacity=0.12 ,even odd rule] (262.7,123.94) .. controls (263.71,122.06) and (269.45,120.54) .. (275.52,120.54) .. controls (281.59,120.54) and (285.69,122.06) .. (284.68,123.94) .. controls (283.67,125.82) and (277.94,127.35) .. (271.87,127.35) .. controls (265.8,127.35) and (261.69,125.82) .. (262.7,123.94)(257.59,123.94) .. controls (260.11,119.24) and (269.36,115.43) .. (278.25,115.43) .. controls (287.14,115.43) and (292.31,119.24) .. (289.79,123.94) .. controls (287.27,128.64) and (278.02,132.46) .. (269.13,132.46) .. controls (260.24,132.46) and (255.07,128.64) .. (257.59,123.94) ;
%Shape: Donut [id:dp6040532681201994] 
\draw  [fill={rgb, 255:red, 248; green, 231; blue, 28 }  ,fill opacity=0.12 ,even odd rule] (248.79,142.22) .. controls (249.8,140.34) and (255.54,138.81) .. (261.61,138.81) .. controls (267.68,138.81) and (271.78,140.34) .. (270.77,142.22) .. controls (269.77,144.1) and (264.03,145.62) .. (257.96,145.62) .. controls (251.89,145.62) and (247.79,144.1) .. (248.79,142.22)(243.68,142.22) .. controls (246.2,137.52) and (255.45,133.71) .. (264.34,133.71) .. controls (273.24,133.71) and (278.4,137.52) .. (275.88,142.22) .. controls (273.36,146.92) and (264.11,150.73) .. (255.22,150.73) .. controls (246.33,150.73) and (241.17,146.92) .. (243.68,142.22) ;
%Shape: Donut [id:dp6838075402006836] 
\draw  [fill={rgb, 255:red, 248; green, 231; blue, 28 }  ,fill opacity=0.12 ,even odd rule] (235.7,161.29) .. controls (236.71,159.41) and (242.45,157.88) .. (248.52,157.88) .. controls (254.59,157.88) and (258.69,159.41) .. (257.68,161.29) .. controls (256.67,163.17) and (250.94,164.7) .. (244.87,164.7) .. controls (238.8,164.7) and (234.69,163.17) .. (235.7,161.29)(230.59,161.29) .. controls (233.11,156.59) and (242.36,152.78) .. (251.25,152.78) .. controls (260.14,152.78) and (265.31,156.59) .. (262.79,161.29) .. controls (260.27,165.99) and (251.02,169.8) .. (242.13,169.8) .. controls (233.24,169.8) and (228.07,165.99) .. (230.59,161.29) ;
%Shape: Donut [id:dp755350096761324] 
\draw  [fill={rgb, 255:red, 248; green, 231; blue, 28 }  ,fill opacity=0.12 ,even odd rule] (222.61,179.57) .. controls (223.62,177.69) and (229.36,176.16) .. (235.43,176.16) .. controls (241.5,176.16) and (245.6,177.69) .. (244.59,179.57) .. controls (243.58,181.45) and (237.85,182.97) .. (231.78,182.97) .. controls (225.71,182.97) and (221.6,181.45) .. (222.61,179.57)(217.5,179.57) .. controls (220.02,174.86) and (229.27,171.05) .. (238.16,171.05) .. controls (247.05,171.05) and (252.22,174.86) .. (249.7,179.57) .. controls (247.18,184.27) and (237.93,188.08) .. (229.04,188.08) .. controls (220.15,188.08) and (214.98,184.27) .. (217.5,179.57) ;
%Shape: Donut [id:dp7187048217890918] 
\draw  [fill={rgb, 255:red, 248; green, 231; blue, 28 }  ,fill opacity=0.12 ,even odd rule] (208.7,197.84) .. controls (209.71,195.96) and (215.45,194.44) .. (221.52,194.44) .. controls (227.59,194.44) and (231.69,195.96) .. (230.68,197.84) .. controls (229.68,199.72) and (223.94,201.25) .. (217.87,201.25) .. controls (211.8,201.25) and (207.69,199.72) .. (208.7,197.84)(203.59,197.84) .. controls (206.11,193.14) and (215.36,189.33) .. (224.25,189.33) .. controls (233.14,189.33) and (238.31,193.14) .. (235.79,197.84) .. controls (233.27,202.55) and (224.02,206.36) .. (215.13,206.36) .. controls (206.24,206.36) and (201.07,202.55) .. (203.59,197.84) ;
%Shape: Donut [id:dp39073666900044846] 
\draw  [fill={rgb, 255:red, 248; green, 231; blue, 28 }  ,fill opacity=0.12 ,even odd rule] (195.61,216.91) .. controls (196.62,215.03) and (202.36,213.51) .. (208.43,213.51) .. controls (214.5,213.51) and (218.6,215.03) .. (217.59,216.91) .. controls (216.58,218.8) and (210.85,220.32) .. (204.78,220.32) .. controls (198.71,220.32) and (194.6,218.8) .. (195.61,216.91)(190.5,216.91) .. controls (193.02,212.21) and (202.27,208.4) .. (211.16,208.4) .. controls (220.05,208.4) and (225.22,212.21) .. (222.7,216.91) .. controls (220.18,221.62) and (210.93,225.43) .. (202.04,225.43) .. controls (193.15,225.43) and (187.98,221.62) .. (190.5,216.91) ;
%Shape: Donut [id:dp8861872030226708] 
\draw  [fill={rgb, 255:red, 248; green, 231; blue, 28 }  ,fill opacity=0.12 ,even odd rule] (288.7,86.94) .. controls (289.71,85.06) and (295.45,83.54) .. (301.52,83.54) .. controls (307.59,83.54) and (311.69,85.06) .. (310.68,86.94) .. controls (309.67,88.82) and (303.94,90.35) .. (297.87,90.35) .. controls (291.8,90.35) and (287.69,88.82) .. (288.7,86.94)(283.59,86.94) .. controls (286.11,82.24) and (295.36,78.43) .. (304.25,78.43) .. controls (313.14,78.43) and (318.31,82.24) .. (315.79,86.94) .. controls (313.27,91.64) and (304.02,95.46) .. (295.13,95.46) .. controls (286.24,95.46) and (281.07,91.64) .. (283.59,86.94) ;
%Shape: Donut [id:dp7456328457888742] 
\draw  [fill={rgb, 255:red, 248; green, 231; blue, 28 }  ,fill opacity=0.12 ,even odd rule] (274.79,105.22) .. controls (275.8,103.34) and (281.54,101.81) .. (287.61,101.81) .. controls (293.68,101.81) and (297.78,103.34) .. (296.77,105.22) .. controls (295.77,107.1) and (290.03,108.62) .. (283.96,108.62) .. controls (277.89,108.62) and (273.79,107.1) .. (274.79,105.22)(269.68,105.22) .. controls (272.2,100.52) and (281.45,96.71) .. (290.34,96.71) .. controls (299.24,96.71) and (304.4,100.52) .. (301.88,105.22) .. controls (299.36,109.92) and (290.11,113.73) .. (281.22,113.73) .. controls (272.33,113.73) and (267.17,109.92) .. (269.68,105.22) ;

%Shape: Donut [id:dp35548577155814665] 
\draw  [fill={rgb, 255:red, 245; green, 166; blue, 35 }  ,fill opacity=0.38 ,even odd rule] (306.7,122.94) .. controls (307.71,121.06) and (313.45,119.54) .. (319.52,119.54) .. controls (325.59,119.54) and (329.69,121.06) .. (328.68,122.94) .. controls (327.67,124.82) and (321.94,126.35) .. (315.87,126.35) .. controls (309.8,126.35) and (305.69,124.82) .. (306.7,122.94)(301.59,122.94) .. controls (304.11,118.24) and (313.36,114.43) .. (322.25,114.43) .. controls (331.14,114.43) and (336.31,118.24) .. (333.79,122.94) .. controls (331.27,127.64) and (322.02,131.46) .. (313.13,131.46) .. controls (304.24,131.46) and (299.07,127.64) .. (301.59,122.94) ;
%Shape: Donut [id:dp5430092094972596] 
\draw  [fill={rgb, 255:red, 245; green, 166; blue, 35 }  ,fill opacity=0.38 ,even odd rule] (292.79,141.22) .. controls (293.8,139.34) and (299.54,137.81) .. (305.61,137.81) .. controls (311.68,137.81) and (315.78,139.34) .. (314.77,141.22) .. controls (313.77,143.1) and (308.03,144.62) .. (301.96,144.62) .. controls (295.89,144.62) and (291.79,143.1) .. (292.79,141.22)(287.68,141.22) .. controls (290.2,136.52) and (299.45,132.71) .. (308.34,132.71) .. controls (317.24,132.71) and (322.4,136.52) .. (319.88,141.22) .. controls (317.36,145.92) and (308.11,149.73) .. (299.22,149.73) .. controls (290.33,149.73) and (285.17,145.92) .. (287.68,141.22) ;
%Shape: Donut [id:dp47096654104393976] 
\draw  [fill={rgb, 255:red, 245; green, 166; blue, 35 }  ,fill opacity=0.38 ,even odd rule] (279.7,160.29) .. controls (280.71,158.41) and (286.45,156.88) .. (292.52,156.88) .. controls (298.59,156.88) and (302.69,158.41) .. (301.68,160.29) .. controls (300.67,162.17) and (294.94,163.7) .. (288.87,163.7) .. controls (282.8,163.7) and (278.69,162.17) .. (279.7,160.29)(274.59,160.29) .. controls (277.11,155.59) and (286.36,151.78) .. (295.25,151.78) .. controls (304.14,151.78) and (309.31,155.59) .. (306.79,160.29) .. controls (304.27,164.99) and (295.02,168.8) .. (286.13,168.8) .. controls (277.24,168.8) and (272.07,164.99) .. (274.59,160.29) ;
%Shape: Donut [id:dp585347730216048] 
\draw  [fill={rgb, 255:red, 245; green, 166; blue, 35 }  ,fill opacity=0.38 ,even odd rule] (266.61,178.57) .. controls (267.62,176.69) and (273.36,175.16) .. (279.43,175.16) .. controls (285.5,175.16) and (289.6,176.69) .. (288.59,178.57) .. controls (287.58,180.45) and (281.85,181.97) .. (275.78,181.97) .. controls (269.71,181.97) and (265.6,180.45) .. (266.61,178.57)(261.5,178.57) .. controls (264.02,173.86) and (273.27,170.05) .. (282.16,170.05) .. controls (291.05,170.05) and (296.22,173.86) .. (293.7,178.57) .. controls (291.18,183.27) and (281.93,187.08) .. (273.04,187.08) .. controls (264.15,187.08) and (258.98,183.27) .. (261.5,178.57) ;
%Shape: Donut [id:dp28451843417801626] 
\draw  [fill={rgb, 255:red, 245; green, 166; blue, 35 }  ,fill opacity=0.38 ,even odd rule] (252.7,196.84) .. controls (253.71,194.96) and (259.45,193.44) .. (265.52,193.44) .. controls (271.59,193.44) and (275.69,194.96) .. (274.68,196.84) .. controls (273.68,198.72) and (267.94,200.25) .. (261.87,200.25) .. controls (255.8,200.25) and (251.69,198.72) .. (252.7,196.84)(247.59,196.84) .. controls (250.11,192.14) and (259.36,188.33) .. (268.25,188.33) .. controls (277.14,188.33) and (282.31,192.14) .. (279.79,196.84) .. controls (277.27,201.55) and (268.02,205.36) .. (259.13,205.36) .. controls (250.24,205.36) and (245.07,201.55) .. (247.59,196.84) ;
%Shape: Donut [id:dp8060604821763232] 
\draw  [fill={rgb, 255:red, 245; green, 166; blue, 35 }  ,fill opacity=0.38 ,even odd rule] (239.61,215.91) .. controls (240.62,214.03) and (246.36,212.51) .. (252.43,212.51) .. controls (258.5,212.51) and (262.6,214.03) .. (261.59,215.91) .. controls (260.58,217.8) and (254.85,219.32) .. (248.78,219.32) .. controls (242.71,219.32) and (238.6,217.8) .. (239.61,215.91)(234.5,215.91) .. controls (237.02,211.21) and (246.27,207.4) .. (255.16,207.4) .. controls (264.05,207.4) and (269.22,211.21) .. (266.7,215.91) .. controls (264.18,220.62) and (254.93,224.43) .. (246.04,224.43) .. controls (237.15,224.43) and (231.98,220.62) .. (234.5,215.91) ;
%Shape: Donut [id:dp9994219685636467] 
\draw  [fill={rgb, 255:red, 245; green, 166; blue, 35 }  ,fill opacity=0.38 ,even odd rule] (332.7,85.94) .. controls (333.71,84.06) and (339.45,82.54) .. (345.52,82.54) .. controls (351.59,82.54) and (355.69,84.06) .. (354.68,85.94) .. controls (353.67,87.82) and (347.94,89.35) .. (341.87,89.35) .. controls (335.8,89.35) and (331.69,87.82) .. (332.7,85.94)(327.59,85.94) .. controls (330.11,81.24) and (339.36,77.43) .. (348.25,77.43) .. controls (357.14,77.43) and (362.31,81.24) .. (359.79,85.94) .. controls (357.27,90.64) and (348.02,94.46) .. (339.13,94.46) .. controls (330.24,94.46) and (325.07,90.64) .. (327.59,85.94) ;
%Shape: Donut [id:dp6012320798518307] 
\draw  [fill={rgb, 255:red, 245; green, 166; blue, 35 }  ,fill opacity=0.38 ,even odd rule] (318.79,104.22) .. controls (319.8,102.34) and (325.54,100.81) .. (331.61,100.81) .. controls (337.68,100.81) and (341.78,102.34) .. (340.77,104.22) .. controls (339.77,106.1) and (334.03,107.62) .. (327.96,107.62) .. controls (321.89,107.62) and (317.79,106.1) .. (318.79,104.22)(313.68,104.22) .. controls (316.2,99.52) and (325.45,95.71) .. (334.34,95.71) .. controls (343.24,95.71) and (348.4,99.52) .. (345.88,104.22) .. controls (343.36,108.92) and (334.11,112.73) .. (325.22,112.73) .. controls (316.33,112.73) and (311.17,108.92) .. (313.68,104.22) ;

%Shape: Donut [id:dp0270048895653342] 
\draw  [fill={rgb, 255:red, 139; green, 87; blue, 42 }  ,fill opacity=1 ,even odd rule] (348.7,123.94) .. controls (349.71,122.06) and (355.45,120.54) .. (361.52,120.54) .. controls (367.59,120.54) and (371.69,122.06) .. (370.68,123.94) .. controls (369.67,125.82) and (363.94,127.35) .. (357.87,127.35) .. controls (351.8,127.35) and (347.69,125.82) .. (348.7,123.94)(343.59,123.94) .. controls (346.11,119.24) and (355.36,115.43) .. (364.25,115.43) .. controls (373.14,115.43) and (378.31,119.24) .. (375.79,123.94) .. controls (373.27,128.64) and (364.02,132.46) .. (355.13,132.46) .. controls (346.24,132.46) and (341.07,128.64) .. (343.59,123.94) ;
%Shape: Donut [id:dp9827202866531394] 
\draw  [fill={rgb, 255:red, 139; green, 87; blue, 42 }  ,fill opacity=1 ,even odd rule] (334.79,142.22) .. controls (335.8,140.34) and (341.54,138.81) .. (347.61,138.81) .. controls (353.68,138.81) and (357.78,140.34) .. (356.77,142.22) .. controls (355.77,144.1) and (350.03,145.62) .. (343.96,145.62) .. controls (337.89,145.62) and (333.79,144.1) .. (334.79,142.22)(329.68,142.22) .. controls (332.2,137.52) and (341.45,133.71) .. (350.34,133.71) .. controls (359.24,133.71) and (364.4,137.52) .. (361.88,142.22) .. controls (359.36,146.92) and (350.11,150.73) .. (341.22,150.73) .. controls (332.33,150.73) and (327.17,146.92) .. (329.68,142.22) ;
%Shape: Donut [id:dp5569572224851507] 
\draw  [fill={rgb, 255:red, 139; green, 87; blue, 42 }  ,fill opacity=1 ,even odd rule] (321.7,161.29) .. controls (322.71,159.41) and (328.45,157.88) .. (334.52,157.88) .. controls (340.59,157.88) and (344.69,159.41) .. (343.68,161.29) .. controls (342.67,163.17) and (336.94,164.7) .. (330.87,164.7) .. controls (324.8,164.7) and (320.69,163.17) .. (321.7,161.29)(316.59,161.29) .. controls (319.11,156.59) and (328.36,152.78) .. (337.25,152.78) .. controls (346.14,152.78) and (351.31,156.59) .. (348.79,161.29) .. controls (346.27,165.99) and (337.02,169.8) .. (328.13,169.8) .. controls (319.24,169.8) and (314.07,165.99) .. (316.59,161.29) ;
%Shape: Donut [id:dp3335888372928081] 
\draw  [fill={rgb, 255:red, 139; green, 87; blue, 42 }  ,fill opacity=1 ,even odd rule] (308.61,179.57) .. controls (309.62,177.69) and (315.36,176.16) .. (321.43,176.16) .. controls (327.5,176.16) and (331.6,177.69) .. (330.59,179.57) .. controls (329.58,181.45) and (323.85,182.97) .. (317.78,182.97) .. controls (311.71,182.97) and (307.6,181.45) .. (308.61,179.57)(303.5,179.57) .. controls (306.02,174.86) and (315.27,171.05) .. (324.16,171.05) .. controls (333.05,171.05) and (338.22,174.86) .. (335.7,179.57) .. controls (333.18,184.27) and (323.93,188.08) .. (315.04,188.08) .. controls (306.15,188.08) and (300.98,184.27) .. (303.5,179.57) ;
%Shape: Donut [id:dp5879206705122096] 
\draw  [fill={rgb, 255:red, 139; green, 87; blue, 42 }  ,fill opacity=1 ,even odd rule] (294.7,197.84) .. controls (295.71,195.96) and (301.45,194.44) .. (307.52,194.44) .. controls (313.59,194.44) and (317.69,195.96) .. (316.68,197.84) .. controls (315.68,199.72) and (309.94,201.25) .. (303.87,201.25) .. controls (297.8,201.25) and (293.69,199.72) .. (294.7,197.84)(289.59,197.84) .. controls (292.11,193.14) and (301.36,189.33) .. (310.25,189.33) .. controls (319.14,189.33) and (324.31,193.14) .. (321.79,197.84) .. controls (319.27,202.55) and (310.02,206.36) .. (301.13,206.36) .. controls (292.24,206.36) and (287.07,202.55) .. (289.59,197.84) ;
%Shape: Donut [id:dp24801658784669922] 
\draw  [fill={rgb, 255:red, 139; green, 87; blue, 42 }  ,fill opacity=1 ,even odd rule] (281.61,216.91) .. controls (282.62,215.03) and (288.36,213.51) .. (294.43,213.51) .. controls (300.5,213.51) and (304.6,215.03) .. (303.59,216.91) .. controls (302.58,218.8) and (296.85,220.32) .. (290.78,220.32) .. controls (284.71,220.32) and (280.6,218.8) .. (281.61,216.91)(276.5,216.91) .. controls (279.02,212.21) and (288.27,208.4) .. (297.16,208.4) .. controls (306.05,208.4) and (311.22,212.21) .. (308.7,216.91) .. controls (306.18,221.62) and (296.93,225.43) .. (288.04,225.43) .. controls (279.15,225.43) and (273.98,221.62) .. (276.5,216.91) ;
%Shape: Donut [id:dp5730054593593082] 
\draw  [fill={rgb, 255:red, 139; green, 87; blue, 42 }  ,fill opacity=1 ,even odd rule] (374.7,86.94) .. controls (375.71,85.06) and (381.45,83.54) .. (387.52,83.54) .. controls (393.59,83.54) and (397.69,85.06) .. (396.68,86.94) .. controls (395.67,88.82) and (389.94,90.35) .. (383.87,90.35) .. controls (377.8,90.35) and (373.69,88.82) .. (374.7,86.94)(369.59,86.94) .. controls (372.11,82.24) and (381.36,78.43) .. (390.25,78.43) .. controls (399.14,78.43) and (404.31,82.24) .. (401.79,86.94) .. controls (399.27,91.64) and (390.02,95.46) .. (381.13,95.46) .. controls (372.24,95.46) and (367.07,91.64) .. (369.59,86.94) ;
%Shape: Donut [id:dp1406074413334435] 
\draw  [fill={rgb, 255:red, 139; green, 87; blue, 42 }  ,fill opacity=1 ,even odd rule] (360.79,105.22) .. controls (361.8,103.34) and (367.54,101.81) .. (373.61,101.81) .. controls (379.68,101.81) and (383.78,103.34) .. (382.77,105.22) .. controls (381.77,107.1) and (376.03,108.62) .. (369.96,108.62) .. controls (363.89,108.62) and (359.79,107.1) .. (360.79,105.22)(355.68,105.22) .. controls (358.2,100.52) and (367.45,96.71) .. (376.34,96.71) .. controls (385.24,96.71) and (390.4,100.52) .. (387.88,105.22) .. controls (385.36,109.92) and (376.11,113.73) .. (367.22,113.73) .. controls (358.33,113.73) and (353.17,109.92) .. (355.68,105.22) ;

%Shape: Donut [id:dp898762632476243] 
\draw  [fill={rgb, 255:red, 80; green, 227; blue, 194 }  ,fill opacity=0.53 ,even odd rule] (390.7,124.94) .. controls (391.71,123.06) and (397.45,121.54) .. (403.52,121.54) .. controls (409.59,121.54) and (413.69,123.06) .. (412.68,124.94) .. controls (411.67,126.82) and (405.94,128.35) .. (399.87,128.35) .. controls (393.8,128.35) and (389.69,126.82) .. (390.7,124.94)(385.59,124.94) .. controls (388.11,120.24) and (397.36,116.43) .. (406.25,116.43) .. controls (415.14,116.43) and (420.31,120.24) .. (417.79,124.94) .. controls (415.27,129.64) and (406.02,133.46) .. (397.13,133.46) .. controls (388.24,133.46) and (383.07,129.64) .. (385.59,124.94) ;
%Shape: Donut [id:dp5434160389902921] 
\draw  [fill={rgb, 255:red, 80; green, 227; blue, 194 }  ,fill opacity=0.53 ,even odd rule] (376.79,143.22) .. controls (377.8,141.34) and (383.54,139.81) .. (389.61,139.81) .. controls (395.68,139.81) and (399.78,141.34) .. (398.77,143.22) .. controls (397.77,145.1) and (392.03,146.62) .. (385.96,146.62) .. controls (379.89,146.62) and (375.79,145.1) .. (376.79,143.22)(371.68,143.22) .. controls (374.2,138.52) and (383.45,134.71) .. (392.34,134.71) .. controls (401.24,134.71) and (406.4,138.52) .. (403.88,143.22) .. controls (401.36,147.92) and (392.11,151.73) .. (383.22,151.73) .. controls (374.33,151.73) and (369.17,147.92) .. (371.68,143.22) ;
%Shape: Donut [id:dp4130875921492253] 
\draw  [fill={rgb, 255:red, 80; green, 227; blue, 194 }  ,fill opacity=0.53 ,even odd rule] (363.7,162.29) .. controls (364.71,160.41) and (370.45,158.88) .. (376.52,158.88) .. controls (382.59,158.88) and (386.69,160.41) .. (385.68,162.29) .. controls (384.67,164.17) and (378.94,165.7) .. (372.87,165.7) .. controls (366.8,165.7) and (362.69,164.17) .. (363.7,162.29)(358.59,162.29) .. controls (361.11,157.59) and (370.36,153.78) .. (379.25,153.78) .. controls (388.14,153.78) and (393.31,157.59) .. (390.79,162.29) .. controls (388.27,166.99) and (379.02,170.8) .. (370.13,170.8) .. controls (361.24,170.8) and (356.07,166.99) .. (358.59,162.29) ;
%Shape: Donut [id:dp2726498638322501] 
\draw  [fill={rgb, 255:red, 80; green, 227; blue, 194 }  ,fill opacity=0.53 ,even odd rule] (350.61,180.57) .. controls (351.62,178.69) and (357.36,177.16) .. (363.43,177.16) .. controls (369.5,177.16) and (373.6,178.69) .. (372.59,180.57) .. controls (371.58,182.45) and (365.85,183.97) .. (359.78,183.97) .. controls (353.71,183.97) and (349.6,182.45) .. (350.61,180.57)(345.5,180.57) .. controls (348.02,175.86) and (357.27,172.05) .. (366.16,172.05) .. controls (375.05,172.05) and (380.22,175.86) .. (377.7,180.57) .. controls (375.18,185.27) and (365.93,189.08) .. (357.04,189.08) .. controls (348.15,189.08) and (342.98,185.27) .. (345.5,180.57) ;
%Shape: Donut [id:dp5825796624648208] 
\draw  [fill={rgb, 255:red, 80; green, 227; blue, 194 }  ,fill opacity=0.53 ,even odd rule] (336.7,198.84) .. controls (337.71,196.96) and (343.45,195.44) .. (349.52,195.44) .. controls (355.59,195.44) and (359.69,196.96) .. (358.68,198.84) .. controls (357.68,200.72) and (351.94,202.25) .. (345.87,202.25) .. controls (339.8,202.25) and (335.69,200.72) .. (336.7,198.84)(331.59,198.84) .. controls (334.11,194.14) and (343.36,190.33) .. (352.25,190.33) .. controls (361.14,190.33) and (366.31,194.14) .. (363.79,198.84) .. controls (361.27,203.55) and (352.02,207.36) .. (343.13,207.36) .. controls (334.24,207.36) and (329.07,203.55) .. (331.59,198.84) ;
%Shape: Donut [id:dp2226766680540604] 
\draw  [fill={rgb, 255:red, 80; green, 227; blue, 194 }  ,fill opacity=0.53 ,even odd rule] (323.61,217.91) .. controls (324.62,216.03) and (330.36,214.51) .. (336.43,214.51) .. controls (342.5,214.51) and (346.6,216.03) .. (345.59,217.91) .. controls (344.58,219.8) and (338.85,221.32) .. (332.78,221.32) .. controls (326.71,221.32) and (322.6,219.8) .. (323.61,217.91)(318.5,217.91) .. controls (321.02,213.21) and (330.27,209.4) .. (339.16,209.4) .. controls (348.05,209.4) and (353.22,213.21) .. (350.7,217.91) .. controls (348.18,222.62) and (338.93,226.43) .. (330.04,226.43) .. controls (321.15,226.43) and (315.98,222.62) .. (318.5,217.91) ;
%Shape: Donut [id:dp4896952853363695] 
\draw  [fill={rgb, 255:red, 80; green, 227; blue, 194 }  ,fill opacity=0.53 ,even odd rule] (416.7,87.94) .. controls (417.71,86.06) and (423.45,84.54) .. (429.52,84.54) .. controls (435.59,84.54) and (439.69,86.06) .. (438.68,87.94) .. controls (437.67,89.82) and (431.94,91.35) .. (425.87,91.35) .. controls (419.8,91.35) and (415.69,89.82) .. (416.7,87.94)(411.59,87.94) .. controls (414.11,83.24) and (423.36,79.43) .. (432.25,79.43) .. controls (441.14,79.43) and (446.31,83.24) .. (443.79,87.94) .. controls (441.27,92.64) and (432.02,96.46) .. (423.13,96.46) .. controls (414.24,96.46) and (409.07,92.64) .. (411.59,87.94) ;
%Shape: Donut [id:dp5594867641914353] 
\draw  [fill={rgb, 255:red, 80; green, 227; blue, 194 }  ,fill opacity=0.53 ,even odd rule] (402.79,106.22) .. controls (403.8,104.34) and (409.54,102.81) .. (415.61,102.81) .. controls (421.68,102.81) and (425.78,104.34) .. (424.77,106.22) .. controls (423.77,108.1) and (418.03,109.62) .. (411.96,109.62) .. controls (405.89,109.62) and (401.79,108.1) .. (402.79,106.22)(397.68,106.22) .. controls (400.2,101.52) and (409.45,97.71) .. (418.34,97.71) .. controls (427.24,97.71) and (432.4,101.52) .. (429.88,106.22) .. controls (427.36,110.92) and (418.11,114.73) .. (409.22,114.73) .. controls (400.33,114.73) and (395.17,110.92) .. (397.68,106.22) ;

%Shape: Donut [id:dp5322206436213677] 
\draw  [fill={rgb, 255:red, 208; green, 2; blue, 27 }  ,fill opacity=1 ,even odd rule] (432.7,124.94) .. controls (433.71,123.06) and (439.45,121.54) .. (445.52,121.54) .. controls (451.59,121.54) and (455.69,123.06) .. (454.68,124.94) .. controls (453.67,126.82) and (447.94,128.35) .. (441.87,128.35) .. controls (435.8,128.35) and (431.69,126.82) .. (432.7,124.94)(427.59,124.94) .. controls (430.11,120.24) and (439.36,116.43) .. (448.25,116.43) .. controls (457.14,116.43) and (462.31,120.24) .. (459.79,124.94) .. controls (457.27,129.64) and (448.02,133.46) .. (439.13,133.46) .. controls (430.24,133.46) and (425.07,129.64) .. (427.59,124.94) ;
%Shape: Donut [id:dp8416638751039649] 
\draw  [fill={rgb, 255:red, 208; green, 2; blue, 27 }  ,fill opacity=1 ,even odd rule] (418.79,143.22) .. controls (419.8,141.34) and (425.54,139.81) .. (431.61,139.81) .. controls (437.68,139.81) and (441.78,141.34) .. (440.77,143.22) .. controls (439.77,145.1) and (434.03,146.62) .. (427.96,146.62) .. controls (421.89,146.62) and (417.79,145.1) .. (418.79,143.22)(413.68,143.22) .. controls (416.2,138.52) and (425.45,134.71) .. (434.34,134.71) .. controls (443.24,134.71) and (448.4,138.52) .. (445.88,143.22) .. controls (443.36,147.92) and (434.11,151.73) .. (425.22,151.73) .. controls (416.33,151.73) and (411.17,147.92) .. (413.68,143.22) ;
%Shape: Donut [id:dp7125302517124127] 
\draw  [fill={rgb, 255:red, 208; green, 2; blue, 27 }  ,fill opacity=1 ,even odd rule] (405.7,162.29) .. controls (406.71,160.41) and (412.45,158.88) .. (418.52,158.88) .. controls (424.59,158.88) and (428.69,160.41) .. (427.68,162.29) .. controls (426.67,164.17) and (420.94,165.7) .. (414.87,165.7) .. controls (408.8,165.7) and (404.69,164.17) .. (405.7,162.29)(400.59,162.29) .. controls (403.11,157.59) and (412.36,153.78) .. (421.25,153.78) .. controls (430.14,153.78) and (435.31,157.59) .. (432.79,162.29) .. controls (430.27,166.99) and (421.02,170.8) .. (412.13,170.8) .. controls (403.24,170.8) and (398.07,166.99) .. (400.59,162.29) ;
%Shape: Donut [id:dp6432151916399159] 
\draw  [fill={rgb, 255:red, 208; green, 2; blue, 27 }  ,fill opacity=1 ,even odd rule] (392.61,180.57) .. controls (393.62,178.69) and (399.36,177.16) .. (405.43,177.16) .. controls (411.5,177.16) and (415.6,178.69) .. (414.59,180.57) .. controls (413.58,182.45) and (407.85,183.97) .. (401.78,183.97) .. controls (395.71,183.97) and (391.6,182.45) .. (392.61,180.57)(387.5,180.57) .. controls (390.02,175.86) and (399.27,172.05) .. (408.16,172.05) .. controls (417.05,172.05) and (422.22,175.86) .. (419.7,180.57) .. controls (417.18,185.27) and (407.93,189.08) .. (399.04,189.08) .. controls (390.15,189.08) and (384.98,185.27) .. (387.5,180.57) ;
%Shape: Donut [id:dp9611534419831996] 
\draw  [fill={rgb, 255:red, 208; green, 2; blue, 27 }  ,fill opacity=1 ,even odd rule] (378.7,198.84) .. controls (379.71,196.96) and (385.45,195.44) .. (391.52,195.44) .. controls (397.59,195.44) and (401.69,196.96) .. (400.68,198.84) .. controls (399.68,200.72) and (393.94,202.25) .. (387.87,202.25) .. controls (381.8,202.25) and (377.69,200.72) .. (378.7,198.84)(373.59,198.84) .. controls (376.11,194.14) and (385.36,190.33) .. (394.25,190.33) .. controls (403.14,190.33) and (408.31,194.14) .. (405.79,198.84) .. controls (403.27,203.55) and (394.02,207.36) .. (385.13,207.36) .. controls (376.24,207.36) and (371.07,203.55) .. (373.59,198.84) ;
%Shape: Donut [id:dp34197931495620393] 
\draw  [fill={rgb, 255:red, 208; green, 2; blue, 27 }  ,fill opacity=1 ,even odd rule] (365.61,217.91) .. controls (366.62,216.03) and (372.36,214.51) .. (378.43,214.51) .. controls (384.5,214.51) and (388.6,216.03) .. (387.59,217.91) .. controls (386.58,219.8) and (380.85,221.32) .. (374.78,221.32) .. controls (368.71,221.32) and (364.6,219.8) .. (365.61,217.91)(360.5,217.91) .. controls (363.02,213.21) and (372.27,209.4) .. (381.16,209.4) .. controls (390.05,209.4) and (395.22,213.21) .. (392.7,217.91) .. controls (390.18,222.62) and (380.93,226.43) .. (372.04,226.43) .. controls (363.15,226.43) and (357.98,222.62) .. (360.5,217.91) ;
%Shape: Donut [id:dp26457004886950675] 
\draw  [fill={rgb, 255:red, 208; green, 2; blue, 27 }  ,fill opacity=1 ,even odd rule] (458.7,87.94) .. controls (459.71,86.06) and (465.45,84.54) .. (471.52,84.54) .. controls (477.59,84.54) and (481.69,86.06) .. (480.68,87.94) .. controls (479.67,89.82) and (473.94,91.35) .. (467.87,91.35) .. controls (461.8,91.35) and (457.69,89.82) .. (458.7,87.94)(453.59,87.94) .. controls (456.11,83.24) and (465.36,79.43) .. (474.25,79.43) .. controls (483.14,79.43) and (488.31,83.24) .. (485.79,87.94) .. controls (483.27,92.64) and (474.02,96.46) .. (465.13,96.46) .. controls (456.24,96.46) and (451.07,92.64) .. (453.59,87.94) ;
%Shape: Donut [id:dp01801300639247405] 
\draw  [fill={rgb, 255:red, 208; green, 2; blue, 27 }  ,fill opacity=1 ,even odd rule] (444.79,106.22) .. controls (445.8,104.34) and (451.54,102.81) .. (457.61,102.81) .. controls (463.68,102.81) and (467.78,104.34) .. (466.77,106.22) .. controls (465.77,108.1) and (460.03,109.62) .. (453.96,109.62) .. controls (447.89,109.62) and (443.79,108.1) .. (444.79,106.22)(439.68,106.22) .. controls (442.2,101.52) and (451.45,97.71) .. (460.34,97.71) .. controls (469.24,97.71) and (474.4,101.52) .. (471.88,106.22) .. controls (469.36,110.92) and (460.11,114.73) .. (451.22,114.73) .. controls (442.33,114.73) and (437.17,110.92) .. (439.68,106.22) ;





\end{tikzpicture}
    \end{center}
    \begin{solution}
    Vamos a usar la fórmula para combinar elementos indistinguibles. En donde: $n=$ opciones a elegir, $r=$ elementos indistinguibles. Entonces, 
    \begin{align*}
        C([n-1]+r,r) &= C([8-1]+8,8)\\
        &= C(15,8)\\
        &= 6435 \text{ formas.}
    \end{align*}
    \end{solution}
    \item ¿De cuántas maneras diferentes podemos escoger diez donas, si debemos elegir al menos dos de chocolate y dos de café?
    \begin{solution}
    Nombramos, $n=$ sabores, $r=$ opciones a elegir y $d=$ opciones a elegir distinguibles. 
    \begin{align*}
        C([n-1]+[r-d],r-d) &= C([6-1]+[10-4],10-4)\\
        &= C(11,6)\\
        &=  462 \text{ formas.}
    \end{align*}
    \end{solution}
\end{enumerate}

\section{Problema 7}
\begin{enumerate}
    \item ¿Cuántas cadenas de 12 bits tienen por lo menos cuatro 1’s?
    \begin{solution}
    \begin{align*}
        C(12,4) &= 495\\
        C(12,5) &= 792\\
        C(12,6) &= 924\\
        C(12,7) &= 792\\
        C(12,8) &= 495\\
        C(12,9) &= 220\\
        C(12,10) &=66 \\
        C(12,11) &=12 \\
        C(12,12) &= 1\\
                 &\quad \_\_\_+\\
                 & \quad 3797 \text{ cadenas.}
    \end{align*}
    \end{solution}
    \item ¿Cuántas cadenas binarias de longitud 10 comienzan con 111 o terminan con 101 o ambos?
    \begin{enumerate}
        \item Por el principio de la multiplicación:
        $$2^7=128 \text{ cadenas.}$$ comienzan en 111.
        \item Por el principio de la multiplicación:
        $$2^7=128\text{ cadenas.}$$ terminan en 101.
        \item Con «ambos» vamos a referirnos a que inician en 111 y terminan en 101; es decir, 
        $$2^4=16\text{ cadenas.}$$
    \end{enumerate}
\end{enumerate}


\section{Problema 8}
\begin{enumerate}
    \item a. Encuentre el coeficiente de $x^4$ en la expansión $(1+3x)^6.$
    \begin{solution}
    Usando el teorema binomial:  
    \begin{align*}
        (1+3x)^6 &= \sum_{j=0}^6\binom{6}{j}(3x)^{6-j}(1)^j
        \intertext{Como nos piden el coeficiente de $x^4$, entonces proponemos que $j=2$, es decir:}
        &= \binom{6}{2}(3x)^{6-2}(1)^2\\
        &= \binom{6}{2}(3^4)x^{4}\\
        &= 1215x^4.
    \end{align*}
    \end{solution}
    \item b. Encuentre el coeficiente de $x^2$ en la expansión $(1-4x)^5.$
    \begin{solution}
    Usando el teorema binomial:  
    \begin{align*}
        (1+(-4x))^5 &= \sum_{j=0}^5\binom{5}{j}(-4x)^{5-j}(1)^j
        \intertext{Como nos piden el coeficiente de $x^2$, entonces proponemos que $j=3$, es decir:}
        &= \binom{5}{3}(-4x)^{5-3}(1)^3\\
        &= \binom{5}{3}(-4)^2x^{2}\\
        &= 160x^2
    \end{align*}
    \end{solution}
    \item c. Encuentre el coeficiente de $x^3$ en la expansión $(1+3x)^6(1-4x)^5.$
    \begin{solution}
    Usando el teorema binomial:  
    \begin{align*}
        (1+3x)^6(1+(-4x))^5 &=\sum_{j=0}^6\binom{6}{j}(3x)^{6-j}(1)^j\cdot \sum_{j=0}^5\binom{5}{j}(-4x)^{5-j}(1)^j
        \intertext{Como nos piden el coeficiente de $x^3$, entonces proponemos que $j=4$, es decir:}
        &=\binom{6}{4}(3x)^{6-4}\cdot \binom{5}{4}(-4x)^{5-4} \\
        &= \binom{6}{4}(3x)^{2}\cdot \binom{5}{4}(-4x)^{1}\\
        &=  \binom{6}{4}\cdot \binom{5}{4} 9\cdot (-4) x^3\\
        &=-2700x^3
    \end{align*}
    \end{solution}
\end{enumerate}

%---------------------------
%\bibliographystyle{apalike}
%\bibliography{sample.bib}

\end{document}